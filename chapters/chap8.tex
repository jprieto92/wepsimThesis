\lhead[\thepage]{CAPÍTULO \thechapter. CONCLUSIONES Y TRABAJOS FUTUROS}
\chead[]{}
\rhead[WepSIM: Simulador de un procesador elemental con unidad de control microprogramada\leftmark]{\thepage}
\renewcommand{\headrulewidth}{0.5pt}

\lfoot[]{}
\cfoot[]{}
\rfoot[]{}
\renewcommand{\footrulewidth}{0pt}

%% This is an example first chapter.  You should put chapter/appendix that you
%% write into a separate file, and add a line \include{yourfilename} to
%% main.tex, where `yourfilename.tex' is the name of the chapter/appendix file.
%% You can process specific files by typing their names in at the 
%% \files=
%% prompt when you run the file main.tex through LaTeX.
\chapter{Conclusiones y trabajos futuros}
\label{ch:conclusions_and_future_work}
\markboth{}{CONCLUSIONS AND FUTURE WORK}

En este capítulo se presentan las conclusiones del trabajo, se revisan los objetivos establecidos al principio de este documento, y se incluyen algunas conclusiones personales. Además, se discuten las principales contribuciones de nuestro trabajo, indicando también las publicaciones resultantes de este trabajo. Finalmente, se discute el trabajo futuro.

\section{Conclusiones}

En este trabajo se ha descrito el diseño de \acrshort{wepsim}, un simulador de un procesador elemental con unidad de control microprogramada. Este trabajo presenta un nuevo simulador que resulta intuitivo, portable y extensible, sirviendo como complemento docente para la docencia en Estructura de Computadores. Este simulador, permite definir diferentes juegos de instrucciones y ejecutar y depurar código fuente que use el conjunto de instrucciones definido. También permite definir el comportamiento del procesador mediante microprogramación.

\acrshort{wepsim}, permite a los estudiante entender el funcionamiento de un procesador elemental de una forma sencilla, pudiendo ser usado desde un dispositivo móvil o un ordenador con un navegador Web moderno, sin la necesidad de ser instalado. De esta forma, los estudiantes pueden interactuar con el simulador aprendiendo y comprendiendo el funcionamiento del procesador elemental \acrshort{wepsim}, incluyendo los mecanismos de interacción con el \gls{software} de sistema e integrando en una misma herramienta tanto la microprogramación como la programación en \gls{ensamblador}.

El objetivo principal de este proyecto era desarrollar un simulador, que a diferencia de los existentes, pudiera simular de forma completa el comportamiento de un procesador elemental permitiendo comprobar el estado de los componentes en cada ciclo de reloj, de manera que ayudáse a los estudiantes a comprender y asimilar de forma sencilla y visual el funcionamiento de un procesador.  También hemos cumplido con todos los demás objetivos presentados en la introducción del documento:

\begin{itemize}

\item \textbf{O1}, Se ha diseñado una herramienta que simula la ejecución del juego de instrucciones especificado en un computador llamado \acrshort{wepsim}, desde el punto de vista de la microprogramación y la programación en \gls{ensamblador}.

\item \textbf{O2}, La herramienta permite la especificación de diferentes juegos de instrucciones.

\item \textbf{O3}, La herramienta unifica la microprogramación de un computador y la programación en lenguaje \gls{ensamblador}.

\item \textbf{O4}, La herramienta permite al usuario visualizar en cada ciclo de reloj el estado y el comportamiento del computador simulado.

\end{itemize}

A nivel personal, este trabajo me ha ayudado a adentrarme en el mundo de la investigación científica. He logrado aplicar una gran cantidad de los conocimientos adquiridos a lo largo del grado. Además, he aprendido importantes técnicas de modelado de \gls{hardware}, compilación y simulación, las cuáles tienen una gran utilidad y complejidad y me han servido parar profundizar aún más en los conocimientos adquiridos en el grado. Por todo ello, es muy satisfactorio ver el resultado final obtenido, puesto que he logrado superar todos los problemas que han surgido a lo largo del proyecto.

\subsection{Contribuciones}

El proyecto llevado a cabo durante este Trabajo Fin de Grado encaja con muchas de las asignaturas estudiadas en el Grado en Ingeniería Informática de la Universidad Carlos III de Madrid, destacando los siguientes temas en particular:

\begin{itemize}

\item \textbf{Tecnología de Computadores} (asignatura obligatoria, Primer curso) en donde se introducen los componentes \gls{hardware} y la lógica binaria.

\item \textbf{Estructura de Computadores} (asignatura obligatoria, Segundo curso) en donde se introducen las bases de la estructura y funcionamiento de un computador.

\item \textbf{Teoría de Autómatas y Lenguajes Formales} (asignatura obligatoria, Segundo curso) en donde se introducen las bases acerca de los lenguajes y gramáticas formales.

\item \textbf{Sistemas Operativos} (asignatura obligatoria, Segundo curso) en donde se introducen las bases del funcionamiento del sistema operativo.

\item \textbf{Arquitectura de Computadores} (asignatura obligatoria, Tercer curso) en donde se introducen las bases de la arquitectura de un computador.

\item \textbf{Diseño de Sistemas Operativos} (asignatura obligatoria, Tercer curso) en donde se introducen las bases del diseño de los distintos módulos de un sistema operativo.

\item \textbf{Dirección de proyectos de desarrollo de \gls{software}} (asignatura obligatoria, Tercer curso) en donde se introducen las bases para la dirección y gestión de un proyecto de desarrollo de \gls{software}.

\end{itemize}

\subsection{Publicaciones}

Este Trabajo Fin de Grado ha permitido realizar una importante contribución al campo de la docencia en Estructura y Arquitectura de Computadores. Además, se han conseguido publicar los siguientes artículos científicos:

\begin{itemize}

\item \textbf{A. Calderón, F. García-Carballeira, and J. Prieto}, “WepSIM: Simulador modular e interactivo de un procesador elemental para facilitar una visión integrada de la microprogramación y la programación en ensamblador”, \textit{Enseñanza y aprendizaje de ingeniería de computadores}, vol. 6, 35-53,2016. \cite{mateos2016wepsim}

\item \textbf{J. Prieto, A. Calderón, F. García-Carballeira, and S. Alonso-Monsalve}, “WepSIM: simulador integrado de microprogramación y programación en ensamblador”, \textit{Jornadas sarteco 2016}. \cite{arcos2032}

\end{itemize}

Además, en el momento de la entrega de este documento, también otro artículo enviado a la espera de su aceptación.

\vspace{1cm}

\section{Trabajos futuros}

Actualmente, hay varias líneas de trabajos futuros en las cuáles estamos trabajando.

\begin{itemize}

\item En cuanto a mejoras en el modelo \gls{hardware}:

\begin{itemize}

\item[1.] Introducir más elementos \gls{hardware}, como por ejemplo una caché, de forma que se amplíen los contenidos de la asignatura incluídos en la herramienta.

\item[2.] Introducir un modelo \gls{hardware} basado en \textit{\gls{pipeline}}, permitiendo el uso de la herramienta en aquellas asignaturas que utilizan este modelo de arquitectura. 

\end{itemize}

\item En cuanto a mejoras en el modelo \gls{software}:

\begin{itemize}

\item[3.] Añadir revisión semántica del código, permitiendo identificar y notificar los errores de programación al usuario.

\item[4.] Añadir nuevos juegos de instrucciones a la herramienta como por ejemplo el \gls{ensamblador} ARM, permitiendo la utilización de diferentes lenguajes en la herramienta.

\item[5.] Estudiar el \gls{ensamblador} de MIPS/ARM generado con GCC/Clang de forma que pueda ser usado directamente en \acrshort{wepsim}.

\end{itemize}

\item En cuanto a mejoras en la herramienta:

\begin{itemize}

\item[6.] Añadir un módulo de corrección automática de prácticas a la herramienta, de forma que los estudiantes puedan practicar con ella y comprobar la validez de sus ejercicios.

\item[7.] Migrar la herramienta como aplicación móvil mediante el \emph{plugin} Apache Cordova, de forma que la herramienta no quede ligada al uso mediante navegador web.

\end{itemize}

\end{itemize}

