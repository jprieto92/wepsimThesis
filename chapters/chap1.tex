\lhead[\thepage]{CHAPTER \thechapter. INTRODUCTION}
\chead[]{}
\rhead[A Complete Simulator for Volunteer Computing Environments\leftmark]{\thepage}
\renewcommand{\headrulewidth}{0.5pt}

\lfoot[]{}
\cfoot[]{}
\rfoot[]{}
\renewcommand{\footrulewidth}{0pt}

%% This is an example first chapter.  You should put chapter/appendix that you
%% write into a separate file, and add a line \include{yourfilename} to
%% main.tex, where `yourfilename.tex' is the name of the chapter/appendix file.
%% You can process specific files by typing their names in at the 
%% \files=
%% prompt when you run the file main.tex through LaTeX.
\chapter{Introducción}
\label{ch:introduction}
\markboth{}{INTRODUCTION}

El primer capítulo introduce brevemente el objetivo del proyecto, ncluyendo the key characteristics of a the project and its motivation (Section \ref{sec:background_and_motivation}, \textit{\nameref{sec:background_and_motivation}}), the project objectives (Section \ref{sec:objectives}, \textit{\nameref{sec:objectives}}), and the entire structure of the document (Section \ref{sec:document_structure}, \textit{\nameref{sec:document_structure}}).

\section{Motivación}
\label{sec:background_and_motivation}
Explicamos la motivación de realizar un simulador interactivo.



\section{Objetivos}
\label{sec:objectives}

Explicamos los objetivos que ha tenido el proyecto:

\begin{itemize}

\item Providing a complete specification of the simulation parameters in each execution.

\item Optimizing the simulator to be efficient in terms of time.

\item Allowing simulations with hundreds of thousands of volunteer clients.

\item Breaking down the simulator structure into discrete modules that allow to easily add new functionalities in the future.

\item Implementing an scheduler on the client side that is close to the actual planning of the \gls{boinc} client.

\item Creating a generator that allows the user to compile and generate the files (executables, platform, and deployment) required for each simulation.

\end{itemize}

\section{Estructura del documento}
\label{sec:document_structure}

Explicamos la estructura que tendrá el documento, lo que se explicará en cada capítulo.
The document contains the following chapters:

\begin{itemize}

\item Chapter \ref{ch:introduction}, \textit{\nameref{ch:introduction}}, presents a brief description of the document contents. It also includes the motivation and the objectives of the project.

\item Chapter \ref{ch:state_of_the_art}, \textit{\nameref{ch:state_of_the_art}}, includes a description of the different types of current distributed computing and presents the related work.

\item Chapter \ref{ch:analysis}, \textit{\nameref{ch:analysis}}, briefly describes the project, explains the chosen solution, sets the requirements, and presents the regulatory \gls{framework} of the project.

\item Chapter \ref{ch:design}, \textit{\nameref{ch:design}}, details the design of the system, including all of its components.

\item Chapter \ref{ch:implementation_and_deployment}, \textit{\nameref{ch:implementation_and_deployment}}, includes the implementation details of the main parts of the developed software and the necessary features for the application deployment.

\item Chapter \ref{ch:verification_validation_and_evaluation}, \textit{\nameref{ch:verification_validation_and_evaluation}}, details a complete verification and validation of the project. It also shows an evaluation of different test cases using  the simulator.

\item Chapter \ref{ch:planning_and_budget}, \textit{\nameref{ch:planning_and_budget}}, presents the concepts related to the followed planning, breaks down all the project costs, and describes the socio-economic environment.

\item Chapter \ref{ch:conclusions_and_future_work}, \textit{\nameref{ch:conclusions_and_future_work}}, includes the contributions of the project, explains the main conclusions of the project and presents future work.

\item Appendix \ref{ch:user_manual}, \textit{\nameref{ch:user_manual}}, includes a complete user manual for the application. It contains a tutorial for the installation of the tools used, and a number of practical and educational examples to learn how to perform simulations using the developed software. 

\end{itemize}

