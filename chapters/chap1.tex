\lhead[\thepage]{CHAPTER \thechapter. INTRODUCTION}
\chead[]{}
\rhead[A Complete Simulator for Volunteer Computing Environments\leftmark]{\thepage}
\renewcommand{\headrulewidth}{0.5pt}

\lfoot[]{}
\cfoot[]{}
\rfoot[]{}
\renewcommand{\footrulewidth}{0pt}

%% This is an example first chapter.  You should put chapter/appendix that you
%% write into a separate file, and add a line \include{yourfilename} to
%% main.tex, where `yourfilename.tex' is the name of the chapter/appendix file.
%% You can process specific files by typing their names in at the 
%% \files=
%% prompt when you run the file main.tex through LaTeX.
\chapter{Introducción}
\label{ch:introduction}
\markboth{}{INTRODUCTION}

El primer capítulo introduce brevemente el objetivo del proyecto, incluyendo las características clave del proyecto y su motivación (Section \ref{sec:background_and_motivation}, \textit{\nameref{sec:background_and_motivation}}), los objetivos del proyecto (Section \ref{sec:objectives}, \textit{\nameref{sec:objectives}}), y toda la estructura del documento(Section \ref{sec:document_structure}, \textit{\nameref{sec:document_structure}}).

\section{Motivación}
\label{sec:background_and_motivation}
Explicamos la motivación de realizar un simulador interactivo.



\section{Objetivos}
\label{sec:objectives}

El objetivo principal de este proyecto, es desarrollar un simulador, que a diferencia de los existentes, pueda simular de forma completa el comportamiento de un procesador elemental permitiendo comprobar el estado de los componentes en cada ciclo de reloj, de manera que ayude a los alumnos a comprender y asimilar de forma sencilla y visual el funcionamiento de un procesador. Los objetivos secundarios son:

\begin{itemize}

\item Diseñar la especificación del juego de instrucciones que permita la creación de un lenguaje ensamblador adaptado a la arquitectura del simulador.

\item Diseñar e implementar el compilador del juego de instrucciones para la generación del firmware del simulador.

\item Diseñar e implementar el compilador genérico de ensamblador que permita la generación del binario correspondiente al juego de instrucciones diseñado.

\item Diseñar e implementar el motor del simulador permitiendo ejecuciones reales del código ensamblador correspondiente al juego de instrucciones compilado.

\item Diseñar una interfaz que proporcione en todo momento la información necesaria en relación al estado de la ejecución del código, de forma sencilla y visual.

\item Permitir que los usuarios puedan importar/exportar tanto la especificación del juego de instrucciones como el código ensamblador.

\item Crear un mecanismo de "modificación en caliente" que permita en mitad de una ejecución modificar el juego de instrucciones, de forma que se puedan realizar pruebas sin necesidad de reiniciar la ejecución.

\end{itemize}

\section{Estructura del documento}
\label{sec:document_structure}

El documento contiene los siguientes capítulos:

\begin{itemize}

\item Capítulo \ref{ch:introduction}, \textit{\nameref{ch:introduction}}, presenta una breve descripción del contenido del documento. También incluye la motivación y los objetivos del proyecto.

\item Capítulo \ref{ch:state_of_the_art}, \textit{\nameref{ch:state_of_the_art}}, incluye una descripción de los diferentes tipos de procesadores y compiladores actuales y presenta el trabajo relacionado.

\item Capítulo \ref{ch:analysis}, \textit{\nameref{ch:analysis}}, describe brevemente el proyecto, explica la solución elegida, establece los requisitos y presenta el marco regulador del proyecto.

\item Capítulo \ref{ch:design}, \textit{\nameref{ch:design}}, detalla el diseño del sistema, incluyendo todos sus componentes.

\item Capítulo \ref{ch:implementation_and_deployment}, \textit{\nameref{ch:implementation_and_deployment}}, incluye los detalles de implementación de las partes principales del software desarrollado y las características necesarias para la implementación de la aplicación.

\item Capítulo \ref{ch:verification_validation_and_evaluation}, \textit{\nameref{ch:verification_validation_and_evaluation}}, detalla una verificación y validación completa del proyecto. También muestra una evaluación de diferentes casos de prueba utilizando el simulador.

\item Capítulo \ref{ch:planning_and_budget}, \textit{\nameref{ch:planning_and_budget}}, presenta los conceptos relacionados con la planificación seguida, descompone todos los costes del proyecto y describe el entorno socio-económico.

\item Capítulo \ref{ch:conclusions_and_future_work}, \textit{\nameref{ch:conclusions_and_future_work}}, incluye las contribuciones del proyecto, explica las principales conclusiones del proyecto y presenta los trabajos futuros.

\item Appendix \ref{ch:user_manual}, \textit{\nameref{ch:user_manual}}, incluye un manual de usuario completo para la aplicación. Contiene un tutorial que guía al usuario desde la creación de un nuevo juego de instrucciones hasta una ejecución completa paso a paso, y una serie de ejemplos educativos para aprender el funcionamiento de un procesador mediante el uso de simulaciones utilizando el software desarrollado. 

\end{itemize}

