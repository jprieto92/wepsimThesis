\lhead[\thepage]{CHAPTER \thechapter. INTRODUCTION}
\chead[]{}
\rhead[WepSIM: Simulador de procesador elemental con unidad de control microprogramada\leftmark]{\thepage}
\renewcommand{\headrulewidth}{0.5pt}

\lfoot[]{}
\cfoot[]{}
\rfoot[]{}
\renewcommand{\footrulewidth}{0pt}

%% This is an example first chapter.  You should put chapter/appendix that you
%% write into a separate file, and add a line \include{yourfilename} to
%% main.tex, where `yourfilename.tex' is the name of the chapter/appendix file.
%% You can process specific files by typing their names in at the 
%% \files=
%% prompt when you run the file main.tex through LaTeX.
\chapter{Introducción}
\label{ch:introduction}
\markboth{}{INTRODUCTION}

El primer capítulo introduce brevemente el objetivo del proyecto, incluyendo las características clave del proyecto y su motivación (Section \ref{sec:background_and_motivation}, \textit{\nameref{sec:background_and_motivation}}), los objetivos del proyecto (Section \ref{sec:objectives}, \textit{\nameref{sec:objectives}}), y toda la estructura del documento(Section \ref{sec:document_structure}, \textit{\nameref{sec:document_structure}}).

\section{Motivación}
\label{sec:background_and_motivation}

La enseñanza de la arquitectura de un computador es una parte básica y fundamental en la formación de los estudiantes de Ingeniería Informática mediante la cual los alumnos logran obtener una visión y comprensión del comportamiento a bajo nivel de la máquina. Para lograr que los estudiantes comprendan y asienten correctamente los fundamentos teóricos, es necesario el uso de clases prácticas, en donde el alumno pueda ser capaz de interactuar con un computador de arquitectura igual o similar a la explicada en teoría y logre extrapolar los fundamentos teóricos al comportamiento real de la máquina.

Uno de los principales problemas a la hora de diseñar estas clases prácticas es lograr obtener los medios necesarios para que los alumnos puedan hacer uso de un computador similar al visto en las clases teóricas, debido al coste que supone tener un número suficiente de computadores para el número de alumnos que deben hacer uso de ellos y su mantenimiento, la limitación de movilidad a la hora de realizar las practicas debido a la necesidad física del computador, etc. Para evitar estos problemas, actualmente se hace uso de simuladores y emuladores que proporcionan las funciones necesarias para las clases prácticas, evitando los problemas anteriormente comentados.

Hay distintos simuladores que se pueden utilizar para trabajar con los principales aspectos que se tratan en las asignaturas de Estructura y Arquitectura de Computadores: ensamblador, caché, etc. Aunque la idea de usar distintos simuladores cae dentro de la estrategia de "divide y vencerás", hay dos principales problemas con estos simuladores: cuanto más realistas son más compleja se hace la enseñanza (tanto del simulador como de la tarea simulada), y cuantos más simuladores se usan más se pierde la visión de conjunto.

Hay otro problema no menos importante: la mayoría de los simuladores están pensados para PC. Uno de los objetivos que nos planteamos con WepSIM es que pudiera ser utilizado en dispositivos móviles (smartphones o tablets), para ofrecer al estudiante una mayor flexibilidad en su uso.

Además de tener un simulador portable a distintas plataformas, el simulador ha de ser lo más autocontenido posible de manera que integre la ayuda principal para su uso (no como un documento separado que sirva de manual de uso para ser impreso) permitiendo al usuario hacer un uso completo de la aplicación sin la necesidad de salir de ella.

Por todo ello, nos hemos planteado cómo ofrecer un simulador que sea simple y modular, y que permita integrar la enseñanza de la microprogramación con la programación en ensamblador. En concreto, puede utilizarse para microprogramar un juego de instrucciones y ver el funcionamiento básico de un procesador, y para crear programas en ensamblador basados en el ensamblador definido por el anterior microcódigo. Esto es de gran ayuda, por ejemplo, para la programación de sistemas dado que es posible ver cómo interactúa el software en ensamblador con el hardware en el tratamiento de interrupciones. La idea es ofrecer un simulador que ofrezca una visión global de lo que pasa en hardware y software, evitando además el tiempo extra que supone el aprendizaje de distintas herramientas.


\section{Objetivos}
\label{sec:objectives}

El objetivo principal de este proyecto, es desarrollar un simulador, que a diferencia de los existentes, pueda simular de forma completa el comportamiento de un procesador elemental permitiendo comprobar el estado de los componentes en cada ciclo de reloj, de manera que ayude a los alumnos a comprender y asimilar de forma sencilla y visual el funcionamiento de un procesador. Los objetivos secundarios son:

\begin{itemize}

\item Diseñar la especificación del juego de instrucciones que permita la creación de un lenguaje ensamblador adaptado a la arquitectura del simulador.

\item Diseñar e implementar el compilador del juego de instrucciones para la generación del firmware del simulador.

\item Diseñar e implementar el compilador genérico de ensamblador que permita la generación del binario correspondiente al juego de instrucciones diseñado.

\item Diseñar e implementar el motor del simulador permitiendo ejecuciones reales del código ensamblador correspondiente al juego de instrucciones compilado.

\item Diseñar una interfaz que proporcione en todo momento la información necesaria en relación al estado de la ejecución del código, de forma sencilla y visual.

\item Permitir que los usuarios puedan importar/exportar tanto la especificación del juego de instrucciones como el código ensamblador.

\item Crear un mecanismo de "modificación en caliente" que permita en mitad de una ejecución modificar el juego de instrucciones, de forma que se puedan realizar pruebas sin necesidad de reiniciar la ejecución.

\end{itemize}

\section{Estructura del documento}
\label{sec:document_structure}

El documento contiene los siguientes capítulos:

\begin{itemize}

\item Capítulo \ref{ch:introduction}, \textit{\nameref{ch:introduction}}, presenta una breve descripción del contenido del documento. También incluye la motivación y los objetivos del proyecto.

\item Capítulo \ref{ch:state_of_the_art}, \textit{\nameref{ch:state_of_the_art}}, incluye una descripción de los diferentes tipos de simuladores de lenguaje ensamblador y simuladores de microcódigo y presenta el trabajo relacionado.

\item Capítulo \ref{ch:analysis}, \textit{\nameref{ch:analysis}}, describe brevemente el proyecto, explica la solución elegida, establece los requisitos y presenta el marco regulador del proyecto.

\item Capítulo \ref{ch:design}, \textit{\nameref{ch:design}}, detalla el diseño del sistema, incluyendo todos sus componentes.

\item Capítulo \ref{ch:implementation_and_deployment}, \textit{\nameref{ch:implementation_and_deployment}}, incluye los detalles de implementación de las partes principales del software desarrollado y las características necesarias para la implementación de la aplicación.

\item Capítulo \ref{ch:verification_validation_and_evaluation}, \textit{\nameref{ch:verification_validation_and_evaluation}}, detalla una verificación y validación completa del proyecto. También muestra una evaluación de diferentes casos de prueba utilizando el simulador.

\item Capítulo \ref{ch:planning_and_budget}, \textit{\nameref{ch:planning_and_budget}}, presenta los conceptos relacionados con la planificación seguida, descompone todos los costes del proyecto y describe el entorno socio-económico.

\item Capítulo \ref{ch:conclusions_and_future_work}, \textit{\nameref{ch:conclusions_and_future_work}}, incluye las contribuciones del proyecto, explica las principales conclusiones del proyecto y presenta los trabajos futuros.

\item Appendix \ref{ch:user_manual}, \textit{\nameref{ch:user_manual}}, incluye un manual de usuario completo para la aplicación. Contiene un tutorial que guía al usuario desde la creación de un nuevo juego de instrucciones hasta una ejecución completa paso a paso, y una serie de ejemplos educativos para aprender el funcionamiento de un procesador mediante el uso de simulaciones utilizando el software desarrollado. 

\end{itemize}

