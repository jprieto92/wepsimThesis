\lhead[\thepage]{CHAPTER \thechapter. STATE OF THE ART}
\chead[]{}
\rhead[A Complete Simulator for Volunteer Computing Environments\leftmark]{\thepage}
\renewcommand{\headrulewidth}{0.5pt}

\lfoot[]{}
\cfoot[]{}
\rfoot[]{}
\renewcommand{\footrulewidth}{0pt}

%% This is an example first chapter.  You should put chapter/appendix that you
%% write into a separate file, and add a line \include{yourfilename} to
%% main.tex, where `yourfilename.tex' is the name of the chapter/appendix file.
%% You can process specific files by typing their names in at the 
%% \files=
%% prompt when you run the file main.tex through LaTeX.
\chapter{Estado del arte}
\label{ch:state_of_the_art}
\markboth{}{STATE OF THE ART}

This chapter presents the state of the art, the latest and most advanced stage of the technologies related to our application. First, we discuss the different types of large-scale distributed systems (Section \ref{sec:large_scale_distributed_systems}). After this, we present the  current methods and tools for the simulation of distributed systems (Section \ref{sec:simulation_of_distributed_systems}). Finally, we deal with the existing volunteer computing simulators (Section \ref{sec:related_work}).

\section{Procesadores}
\label{sec:large_scale_distributed_systems}


\section{Compiladores}
\label{sec:related_work}


\section{Simulación de procesador elemental}
\label{sec:Processor_simulator}

