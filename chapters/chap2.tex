\lhead[\thepage]{CHAPTER \thechapter. STATE OF THE ART}
\chead[]{}
\rhead[WepSIM: Simulador de procesador elemental con unidad de control microprogramada\leftmark]{\thepage}
\renewcommand{\headrulewidth}{0.5pt}

\lfoot[]{}
\cfoot[]{}
\rfoot[]{}
\renewcommand{\footrulewidth}{0pt}

%% This is an example first chapter.  You should put chapter/appendix that you
%% write into a separate file, and add a line \include{yourfilename} to
%% main.tex, where `yourfilename.tex' is the name of the chapter/appendix file.
%% You can process specific files by typing their names in at the 
%% \files=
%% prompt when you run the file main.tex through LaTeX.
\chapter{Estado del arte}
\label{ch:state_of_the_art}
\markboth{}{STATE OF THE ART}

Este capítulo presenta el estado del arte, la última y más avanzada etapa de las tecnologías relacionadas con nuestra aplicación. Primero, se presentan los diferentes simuladores existentes para microprogramación (Section \ref{sec:simuladores_microprogramacion}). Después, se presentan los diferentes simuladores existentes para la programación en código ensamblador (Section \ref{sec:simuladores_ensamblador}). Por último, realizamos una comparación de nuestro trabajo con el contexto actual de los distintos simuladores expuestos previamente (Section \ref{sec:propuesta_simulacion}).

\section{Simuladores para microprogramación}
\label{sec:simuladores_microprogramacion}


\section{Simuladores para programación en ensamblador}
\label{sec:simuladores_ensamblador}


\section{Propuesta de simulación unificada}
\label{sec:propuesta_simulacion}

