\lhead[\thepage]{CHAPTER \thechapter. ANALYSIS}
\chead[]{}
\rhead[A Complete Simulator for Volunteer Computing Environments\leftmark]{\thepage}
\renewcommand{\headrulewidth}{0.5pt}

\lfoot[]{}
\cfoot[]{}
\rfoot[]{}
\renewcommand{\footrulewidth}{0pt}

%% This is an example first chapter.  You should put chapter/appendix that you
%% write into a separate file, and add a line \include{yourfilename} to
%% main.tex, where `yourfilename.tex' is the name of the chapter/appendix file.
%% You can process specific files by typing their names in at the 
%% \files=
%% prompt when you run the file main.tex through LaTeX.
\chapter{Análisis}
\label{ch:analysis}
\markboth{}{ANALYSIS}

El objetivo principal de este capitulo, es describir el proyecto mediante la obtención y especificación de los requisitos del simulador, que puede proporcionar información suficiente para un análisis detallado que, por lo tanto, puede servir para continuar diseñando e implementando (Capítulos \ref{ch:design}, \textit{\nameref{ch:design}}; and \ref{ch:implementation_and_deployment}, \textit{\nameref{ch:implementation_and_deployment}}) un software que cumpla con esos requisitos. 

Con el fin de obtener los requisitos del sistema, el tutor ha desempeñado el papel del cliente en diferentes reuniones, mientras que el alumno ha desempeñado los roles de analista, diseñador, programador y probador.

Section \ref{sec:project_description} briefly summarizes the project description. Section \ref{sec:solution_selection} discusses the chosen solution and compares it to the alternatives considered. Section \ref{sec:requirements} specifies the system requirements, starting with the user requirements, and ending with the functional and non-functional requirements. Finally, Section \ref{sec:regulatory_framework} indicates the set of laws and regulations for the management of the software.

\section{Descripción del proyecto}
\label{sec:project_description}

Explicamos lo que se pretende obtener del simulador.

\section{Solución elegida}
\label{sec:solution_selection}

Explicamos el porque hemos elegido esta solución, y nos comparamos.

\begin{table}[htbp]
\ra{1.2}
\centering
%\resizebox{\textwidth}{%
\resizebox{\textwidth}{!}{
\begin{tabular}{@{}llllll@{}}
\toprule
Features & SimGrid & PVMsim & Virtual-GEMS & MDCSim & SPECI-2 \\ 
\midrule
Languages				& C/C++/Java/Ruby                            & C                   & C/C++/Ruby           & C++/Java             & Java\\
Open Source				& \ding{51}                            & \ding{51}                   & \ding{51}           & \ding{55}             & \ding{51}\\
\midrule
Models & & & & &\\
\midrule
Communication		& \ding{51}               & \ding{55}           & \ding{55}  & \ding{51} & \ding{51}                  \\
Energy				& \ding{51}               & \ding{55}           & \ding{55}  & \ding{51} & \ding{55}                  \\
Hardware				& \ding{51}               & \ding{51}           & \ding{51}  & \ding{51} & \ding{51}                  \\
\Gls{scheduling}			& \ding{51}               & \ding{51}           & \ding{55}  & \ding{55} & \ding{51}                  \\
Users				& \ding{51}               & \ding{55}           & \ding{55}  & \ding{51} & \ding{55}                  \\
\bottomrule
\end{tabular}
}
\caption{Comparison of simulation \gls{framework}s for distributed computing systems.}
\label{tab:comparison_frameworks}
\end{table}


\section{Requisitos}
\label{sec:requirements}

This section provides a detailed description of the application requirements. For the requirement specification task, the IEEE recommended practices \cite{ieee1998} were followed. According to these practices, a good specification must address the software functionality, performance issues, the external interfaces, other non-functional features and design or implementation constraints. Moreover, the requirements specification must be:

\begin{itemize}

\item \textbf{Complete:} the document reflects all significant software requirements.

\item \textbf{Consistent:} requirements must not generate conflicts with each other.

\item \textbf{Correct:} every requirement is one that the software shall meet according to the user needs.

\item \textbf{Modifiable:} the structure of the specification allows changes to the requirements in a simple, complete and consistent way.

\item \textbf{Ranked based on importance and stability:} every requirement must indicate its importance and its stability.

\item \textbf{Traceable:} the origin of every requirement is clear and it can be easily referenced in further stages.

\item \textbf{Unambiguous:} every requirement has a single interpretation.

\item \textbf{Verifiable:} every requirement must be verifiable, that is, there exists some process to verify that the software complies with every single requirement.


\end{itemize}

Starting from the user requirements, which constitute an informal reference to the product performance that the client expects, we derived the software requirements (in this case, functional requirements and non-functional requirements) that guided the design process with specific information on the functionality of the system and other characteristics. The retrieved requirements were structured according with the following schema:

\begin{itemize}
\item[1.] \textbf{User Requirements} 
	\begin{itemize}
		\item[(a)] \textbf{Capacity:} the requirement describes the expected system functionality as in use cases.
		\item[(b)] \textbf{Restriction:} the requirement specifies constraints or conditions the system must fulfil.	
	\end{itemize}	
\end{itemize}

\begin{itemize}
\item[2.] \textbf{Software Requirements}
	\begin{itemize}
	\item[(a)] 	\textbf{Functional}
		\begin{itemize}
		\item[i.] 	\textbf{Functional:} the requirement describes the basic system functionality and purpose while minimizing ambiguity.
		\item[ii.] 	\textbf{Inverse:} the requirement limits the functionality of the application to clarify its scope.
		\end{itemize}

	\item[(b)] 	\textbf{Non-Functional}
		\begin{itemize}
		\item[i.] 	\textbf{Performance:} the requirement is related to the minimum required performance of the resulting system.
		\item[ii.] 	\textbf{Interface:} the requirement is related to the user interface of the application.
		\item[iii.] 	\textbf{Scalability:} the requirement is related to the ability of the system to adapt to increasing workloads.
		\item[iv.] 	\textbf{Platform:} the requirement specifies the underlying software and hardware platforms in which the system will operate.
		\end{itemize}			
	\end{itemize}
\end{itemize}

Table \ref{tab:requirements_template} provides the template used for requirements specification. Note that for user requirements, the ID format will be UR-XYY, where X indicates the requirement subtype: capacity requirements (C), or restrictions (R). YY corresponds to the requirement number under its subcategory. For software requirements, the ID format SR-X-YZZ will be used, where X indicates if it is a functional (F) or non-functional (NF) requirement, and Y represents its subcategory: functional (F), inverse (I), performance (P), interface (UI), scalability (S), or platform (PL). ZZ corresponds to the requirement number under its subcategory.

\begin{center}
\begin{table*}[htbp]
\centering
\begin{tabular}{@{}p{2.5cm} p{9cm}@{}} 
\toprule
\textbf{ID} 				& Requirement ID. \\
\midrule
\textbf{Name} 			& Requirement name. \\
\midrule
\textbf{Type} 			& Indicates the category in which the requirement would be placed according to the previously described schema. \\
\midrule
\textbf{Origin} 			& Constitutes the requirement source. It might be the user, another requirement or other stakeholders involved in the project. \\
\midrule
\textbf{Priority}		& Indicates the requirement priority according to its importance. A requirement can be identified either as \textit{essential}, \textit{conditional} or \textit{optional}. \\
\midrule
\textbf{Stability} 		& Indicates the requirement variability through the development process, defined as \textit{stable} or \textit{unstable}. \\
\midrule
\textbf{Description} 	& Detailed explanation of the requirement. \\
\bottomrule
\end{tabular}
\caption{Template for requirements specification.}
\label{tab:requirements_template}
\end{table*}
\end{center}

\subsection{Requisitos de Usuario}

This subsection specifies the user requirements.

\begin{center}
\begin{table*}[htbp]
\centering
\begin{tabular}{@{}p{2.5cm} p{9cm}@{}} 
\toprule
\textbf{ID} 				& UR-C01\\
\midrule
\textbf{Name} 			& BOINC projects simulation \\
\midrule
\textbf{Type} 			& Capacity \\
\midrule
\textbf{Origin} 			& User \\
\midrule
\textbf{Priority}		& Essential \\
\midrule
\textbf{Stability} 		& Stable \\
\midrule
\textbf{Description} 	& The application shall simulate real BOINC projects. \\
\bottomrule
\end{tabular}
\caption{User requirement UR-C01.}
\label{tab:urc01}
\end{table*}
\end{center}

\begin{center}
\begin{table*}[htbp]
\centering
\begin{tabular}{@{}p{2.5cm} p{9cm}@{}} 
\toprule
\textbf{ID} 				& UR-C02\\
\midrule
\textbf{Name} 			& Client \gls{scheduling} \\
\midrule
\textbf{Type} 			& Capacity \\
\midrule
\textbf{Origin} 			& User \\
\midrule
\textbf{Priority}		& Essential \\
\midrule
\textbf{Stability} 		& Stable \\
\midrule
\textbf{Description} 	& The client scheduler of the simulator shall follow the actual BOINC client \gls{scheduling}. \\
\bottomrule
\end{tabular}
\caption{User requirement UR-C02.}
\label{tab:urc02}
\end{table*}
\end{center}

\begin{center}
\begin{table*}[htbp]
\centering
\begin{tabular}{@{}p{2.5cm} p{9cm}@{}} 
\toprule
\textbf{ID} 				& UR-C03\\
\midrule
\textbf{Name} 			& Simulation components \\
\midrule
\textbf{Type} 			& Capacity \\
\midrule
\textbf{Origin} 			& User \\
\midrule
\textbf{Priority}		& Essential \\
\midrule
\textbf{Stability} 		& Stable \\
\midrule
\textbf{Description} 	& The simulations shall cover all the elements present in the BOINC infrastructure. \\
\bottomrule
\end{tabular}
\caption{User requirement UR-C03.}
\label{tab:urc03}
\end{table*}
\end{center}

\begin{center}
\begin{table*}[htbp]
\centering
\begin{tabular}{@{}p{2.5cm} p{9cm}@{}} 
\toprule
\textbf{ID} 				& UR-R01\\
\midrule
\textbf{Name} 			& Linux as underlying OS \\
\midrule
\textbf{Type} 			& Restriction \\
\midrule
\textbf{Origin} 			& User \\
\midrule
\textbf{Priority}		& Essential \\
\midrule
\textbf{Stability} 		& Stable \\
\midrule
\textbf{Description} 	& The simulator shall be designed for Linux operating systems. \\
\bottomrule
\end{tabular}
\caption{User requirement UR-R01.}
\label{tab:urr01}
\end{table*}
\end{center}

\begin{center}
\begin{table*}[htbp]
\centering
\begin{tabular}{@{}p{2.5cm} p{9cm}@{}} 
\toprule
\textbf{ID} 				& UR-R02\\
\midrule
\textbf{Name} 			& SimGrid toolkit \\
\midrule
\textbf{Type} 			& Restriction \\
\midrule
\textbf{Origin} 			& User \\
\midrule
\textbf{Priority}		& Essential \\
\midrule
\textbf{Stability} 		& Stable \\
\midrule
\textbf{Description} 	& The application shall use the SimGrid toolkit in order to implement the distributed computing functionalities. \\
\bottomrule
\end{tabular}
\caption{User requirement UR-R02.}
\label{tab:urr02}
\end{table*}
\end{center}

\begin{center}
\begin{table*}[htbp]
\centering
\begin{tabular}{@{}p{2.5cm} p{9cm}@{}} 
\toprule
\textbf{ID} 				& UR-R03\\
\midrule
\textbf{Name} 			& Scalability \\
\midrule
\textbf{Type} 			& Restriction \\
\midrule
\textbf{Origin} 			& User \\
\midrule
\textbf{Priority}		& Essential \\
\midrule
\textbf{Stability} 		& Stable \\
\midrule
\textbf{Description} 	& The simulator shall be scalable (carry out executions by simulating a large number of client hosts). \\
\bottomrule
\end{tabular}
\caption{User requirement UR-R03.}
\label{tab:urr03}
\end{table*}
\end{center}

\clearpage
\subsection{Requisitos Funcionales}

This subsection specifies the functional requirements.

\begin{center}
\begin{table*}[htbp]
\centering
\begin{tabular}{@{}p{2.5cm} p{9cm}@{}} 
\toprule
\textbf{ID} 				& SR-F-F01\\
\midrule
\textbf{Name} 			& Credit calculation \\
\midrule
\textbf{Type} 			& Functional \\
\midrule
\textbf{Origin} 			& UR-C01 \\
\midrule
\textbf{Priority}		& Essential \\
\midrule
\textbf{Stability} 		& Stable \\
\midrule
\textbf{Description} 	& The simulator shall calculate the number of credits granted to each volunteer client analogously to actual BOINC projects. \\
\bottomrule
\end{tabular}
\caption{Functional requirement SR-F-F01.}
\label{tab:srff01}
\end{table*}
\end{center}

\begin{center}
\begin{table*}[htbp]
\centering
\begin{tabular}{@{}p{2.5cm} p{9cm}@{}} 
\toprule
\textbf{ID} 				& SR-F-F02\\
\midrule
\textbf{Name} 			& Collection of statistics \\
\midrule
\textbf{Type} 			& Functional \\
\midrule
\textbf{Origin} 			& UR-C01 \\
\midrule
\textbf{Priority}		& Essential \\
\midrule
\textbf{Stability} 		& Stable \\
\midrule
\textbf{Description} 	& The simulator shall collect, for each project, the same statistics that actual BOINC projects (published in BOINCstats \cite{BOINC2016}). \\
\bottomrule
\end{tabular}
\caption{Functional requirement SR-F-F02.}
\label{tab:srff02}
\end{table*}
\end{center}

\begin{center}
\begin{table*}[htbp]
\centering
\begin{tabular}{@{}p{2.5cm} p{9cm}@{}} 
\toprule
\textbf{ID} 				& SR-F-F03\\
\midrule
\textbf{Name} 			& Almost identical outputs \\
\midrule
\textbf{Type} 			& Functional \\
\midrule
\textbf{Origin} 			& UR-C01 \\
\midrule
\textbf{Priority}		& Essential \\
\midrule
\textbf{Stability} 		& Stable \\
\midrule
\textbf{Description} 	& The outputs of the simulator for existing projects should be almost identical to those published in BOINCstats \cite{BOINC2016}. \\
\bottomrule
\end{tabular}
\caption{Functional requirement SR-F-F03.}
\label{tab:srff03}
\end{table*}
\end{center}

\begin{center}
\begin{table*}[htbp]
\centering
\begin{tabular}{@{}p{2.5cm} p{9cm}@{}} 
\toprule
\textbf{ID} 				& SR-F-F04\\
\midrule
\textbf{Name} 			& Multiple BOINC projects \\
\midrule
\textbf{Type} 			& Functional \\
\midrule
\textbf{Origin} 			& UR-C01 \\
\midrule
\textbf{Priority}		& Essential \\
\midrule
\textbf{Stability} 		& Stable \\
\midrule
\textbf{Description} 	& The simulator shall allow the simulation of different projects simultaneously. \\
\bottomrule
\end{tabular}
\caption{Functional requirement SR-F-F04.}
\label{tab:srff04}
\end{table*}
\end{center}

\begin{center}
\begin{table*}[htbp]
\centering
\begin{tabular}{@{}p{2.5cm} p{9cm}@{}} 
\toprule
\textbf{ID} 				& SR-F-F05\\
\midrule
\textbf{Name} 			& Client scheduler \\
\midrule
\textbf{Type} 			& Functional \\
\midrule
\textbf{Origin} 			& UR-C02 \\
\midrule
\textbf{Priority}		& Essential \\
\midrule
\textbf{Stability} 		& Stable \\
\midrule
\textbf{Description} 	& The client scheduler shall follow the actual BOINC client \gls{scheduling} (described in \cite{anderson2007}). \\
\bottomrule
\end{tabular}
\caption{Functional requirement SR-F-F05.}
\label{tab:srff05}
\end{table*}
\end{center}

\begin{center}
\begin{table*}[htbp]
\centering
\begin{tabular}{@{}p{2.5cm} p{9cm}@{}} 
\toprule
\textbf{ID} 				& SR-F-F06\\
\midrule
\textbf{Name} 			& Realistic simulation elements \\
\midrule
\textbf{Type} 			& Functional \\
\midrule
\textbf{Origin} 			& UR-C03 \\
\midrule
\textbf{Priority}		& Essential \\
\midrule
\textbf{Stability} 		& Stable \\
\midrule
\textbf{Description} 	& All simulations shall include the following elements: tasks, volunteer hosts, servers, data servers, networks, and hosts availability. \\
\bottomrule
\end{tabular}
\caption{Functional requirement SR-F-F06.}
\label{tab:srff06}
\end{table*}
\end{center}


\subsection{Requisitos No-Funcionales}

This subsection specifies the non-functional requirements.

\begin{center}
\begin{table*}[htbp]
\centering
\begin{tabular}{@{}p{2.5cm} p{9cm}@{}} 
\toprule
\textbf{ID} 				& SR-NF-PL01\\
\midrule
\textbf{Name} 			& Ubuntu 14.04 \\
\midrule
\textbf{Type} 			& Platform \\
\midrule
\textbf{Origin} 			& UR-R01 \\
\midrule
\textbf{Priority}		& Essential \\
\midrule
\textbf{Stability} 		& Stable \\
\midrule
\textbf{Description} 	& The simulator shall work on the Ubuntu Linux distribution, version 14.04. \\
\bottomrule
\end{tabular}
\caption{Non-functional requirement SR-NF-PL01.}
\label{tab:srnfpl01}
\end{table*}
\end{center}

\begin{center}
\begin{table*}[htbp]
\centering
\begin{tabular}{@{}p{2.5cm} p{9cm}@{}} 
\toprule
\textbf{ID} 				& SR-NF-PL02\\
\midrule
\textbf{Name} 			& SimGrid MSG API \\
\midrule
\textbf{Type} 			& Platform \\
\midrule
\textbf{Origin} 			& UR-R02 \\
\midrule
\textbf{Priority}		& Essential \\
\midrule
\textbf{Stability} 		& Stable \\
\midrule
\textbf{Description} 	& The implementation, setup and control of the simulations shall be carried out using the MSG API of the SimGrid toolkit. \\
\bottomrule
\end{tabular}
\caption{Non-functional requirement SR-NF-PL02.}
\label{tab:srnfpl02}
\end{table*}
\end{center}

\begin{center}
\begin{table*}[htbp]
\centering
\begin{tabular}{@{}p{2.5cm} p{9cm}@{}} 
\toprule
\textbf{ID} 				& SR-NF-PL03\\
\midrule
\textbf{Name} 			& C programming language \\
\midrule
\textbf{Type} 			& Platform\\
\midrule
\textbf{Origin} 			& UR-R03 \\
\midrule
\textbf{Priority}		& Essential \\
\midrule
\textbf{Stability} 		& Stable \\
\midrule
\textbf{Description} 	& The simulator shall be written in the C programming language. \\
\bottomrule
\end{tabular}
\caption{Non-functional requirement SR-NF-PL03.}
\label{tab:srnfpl03}
\end{table*}
\end{center}

\begin{center}
\begin{table*}[htbp]
\centering
\begin{tabular}{@{}p{2.5cm} p{9cm}@{}} 
\toprule
\textbf{ID} 				& SR-NF-S01\\
\midrule
\textbf{Name} 			& Large simulations \\
\midrule
\textbf{Type} 			& Scalability \\
\midrule
\textbf{Origin} 			& UR-R03 \\
\midrule
\textbf{Priority}		& Essential \\
\midrule
\textbf{Stability} 		& Stable \\
\midrule
\textbf{Description} 	& The application must be able to perform simulations with more than 100,000 hosts in a machine with at least 8 GB of \gls{ram}. \\
\bottomrule
\end{tabular}
\caption{Non-functional requirement SR-NF-S01.}
\label{tab:srnfs01}
\end{table*}
\end{center}

\begin{center}
\begin{table*}[htbp]
\centering
\begin{tabular}{@{}p{2.5cm} p{9cm}@{}} 
\toprule
\textbf{ID} 				& SR-NF-P01\\
\midrule
\textbf{Name} 			& Linear-time execution \\
\midrule
\textbf{Type} 			& Performance \\
\midrule
\textbf{Origin} 			& UR-R03 \\
\midrule
\textbf{Priority}		& Conditional \\
\midrule
\textbf{Stability} 		& Stable \\
\midrule
\textbf{Description} 	& Runtime of the simulator must be linear (approximately) in the number of hosts. \\
\bottomrule
\end{tabular}
\caption{Non-functional requirement SR-NF-P01.}
\label{tab:srnfp01}
\end{table*}
\end{center}

\begin{center}
\begin{table*}[htbp]
\centering
\begin{tabular}{@{}p{2.5cm} p{9cm}@{}} 
\toprule
\textbf{ID} 				& SR-NF-UI01\\
\midrule
\textbf{Name} 			& Simulation parameters \\
\midrule
\textbf{Type} 			& Interface \\
\midrule
\textbf{Origin} 			& Analyst \\
\midrule
\textbf{Priority}		& Essential \\
\midrule
\textbf{Stability} 		& Stable \\
\midrule
\textbf{Description} 	& To perform simulations, users only need to specify the simulation parameters in an \gls{xml} file. \\
\bottomrule
\end{tabular}
\caption{Non-functional requirement SR-NF-UI01.}
\label{tab:srnfui01}
\end{table*}
\end{center}

\begin{center}
\begin{table*}[htbp]
\centering
\begin{tabular}{@{}p{2.5cm} p{9cm}@{}}  
\toprule
\textbf{ID} 				& SR-NF-UI02\\
\midrule
\textbf{Name} 			& Progress bar \\
\midrule
\textbf{Type} 			& Interface \\
\midrule
\textbf{Origin} 			& Analyst \\
\midrule
\textbf{Priority}		& Conditional \\
\midrule
\textbf{Stability} 		& Stable \\
\midrule
\textbf{Description} 	& The simulations should include a progress bar. \\
\bottomrule
\end{tabular}
\caption{Non-functional requirement SR-NF-UI02.}
\label{tab:srnfui02}
\end{table*}
\end{center}

\section{Marco Regulador}
\label{sec:regulatory_framework}

This section discusses the necessary constraints taking into account the regulatory \gls{framework}. Specifically, the legal restrictions applicable to the simulator are specified.

\subsection{Restricciones Legales}
\label{sec:legal_constraints}

In the real BOINC system, users must be registered, and \gls{boinc} databases handle confidential information from users, so it is necessary to ensure that third parties can not access that information. One solution is to encrypt the information transmitted following some cryptographic \gls{protocol}. In Spain, this requirement is specified in the article 104 of the RD 1720/2007 \cite{boe2008}, which deals with the Spanish Data Protection Law. 

In contrast, the developed application does not use private data from users, and neither transmits any confidential information to third-parties, because it is just a simulator that does not even require Internet access.

On the other hand, it is crucial that our simulator be available as an \gls{opensource} software. We want it to be such that anyone can redistribute the code or modify it by the terms of the GNU Lesser General Public License (LGPL) \cite{gnulgpl}. To do this, our simulator is available on the following website: \url{https://www.arcos.inf.uc3m.es/~combos/}.

\afterpage{\blankpage} % blank page