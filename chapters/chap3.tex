\lhead[\thepage]{CHAPTER \thechapter. ANALYSIS}
\chead[]{}
\rhead[A Complete Simulator for Volunteer Computing Environments\leftmark]{\thepage}
\renewcommand{\headrulewidth}{0.5pt}

\lfoot[]{}
\cfoot[]{}
\rfoot[]{}
\renewcommand{\footrulewidth}{0pt}

%% This is an example first chapter.  You should put chapter/appendix that you
%% write into a separate file, and add a line \include{yourfilename} to
%% main.tex, where `yourfilename.tex' is the name of the chapter/appendix file.
%% You can process specific files by typing their names in at the 
%% \files=
%% prompt when you run the file main.tex through LaTeX.
\chapter{Análisis}
\label{ch:analysis}
\markboth{}{ANALYSIS}

El objetivo principal de este capitulo, es describir el proyecto mediante la obtención y especificación de los requisitos del simulador, que puede proporcionar información suficiente para un análisis detallado que, por lo tanto, puede servir para continuar diseñando e implementando (Capítulos \ref{ch:design}, \textit{\nameref{ch:design}}; and \ref{ch:implementation_and_deployment}, \textit{\nameref{ch:implementation_and_deployment}}) un software que cumpla con esos requisitos. 

Con el fin de obtener los requisitos del sistema, el tutor ha desempeñado el papel del cliente en diferentes reuniones, mientras que el alumno ha desempeñado los roles de analista, diseñador, programador y probador.

La sección \ref{sec:project_description} resume brevemente la descripción del proyecto. La sección \ref{sec:solution_selection} discute la solución elegida y la compara con las alternativas consideradas. La sección \ref{sec:requirements} especifica los requisitos del sistema, empezando con los requisitos de usuario y finalizando con los requisitos funcionales y no-funcionales. Finalmente, la sección \ref{sec:regulatory_framework} indica el conjunto de leyes y regulaciones para la gestión del software.  

\section{Descripción del proyecto}
\label{sec:project_description}

El objetivo de este proyecto es construir una herramienta que permita simular con realismo el comportamiento de un procesador basado en la arquitectura indicada por el el tutor del proyecto, de forma que sirva como única herramienta para los alumnos a lo largo de la asignatura Estructura de Computadores.

Los simuladores actuales para la enseñanza de Estructura de Computadores, están focalizados a una función concreta, como puede ser la simulación de cachés, la simulación de código ensamblador o la simulación de microcódigo entre otros, pero a la hora de unificar todas estas funcionalidades en una única herramienta, existe un vacío que genera una pérdida de tiempo en el aprendizaje de cada herramienta y la no posibilidad de ver una simulación completa.

El reto al que se enfrente cualquier simulador educativo es el de ser capaz de simular fielmente el funcionamiento de un dispositivo permitiendo al estudiante poder observar con el mayor detalle posible el comportamiento éste.

El sistema que se propone debe ser capaz de simular con realismo el comportamiento del procesador, permitiendo la definición del juego de instrucciones a utilizar para el posterior desarrollo de código ensamblador y su ejecución en el simulador con un alto nivel de detalle en cada uno de los ciclos de ejecución,.De esta forma, los estudiantes podrán comprender fácilmente los contenidos teóricos expuestos en la asignatura y serán capaces de realizar todas las prácticas en una misma herramienta, pudiendo ver de forma incremental el desarrollo de software de bajo nivel sobre un procesador elemental.


\section{Solución elegida}
\label{sec:solution_selection}

Para que los profesores de la asignatura Estructura de Computadores puedan hacer uso de una herramienta que sirva de ayuda para la explicación de los conceptos teóricos de la asignatura, y los alumnos puedan utilizarla para comprender estos conceptos y realizar posteriormente las prácticas de la asignatura, se propone el diseño e implementación de una herramienta web que simule con realismo en funcionamiento de un procesador elemental con unidad de control microprogramable.

Este simulador, será desarrollado como una herramienta web debido a la portabilidad que proporciona, ya que podrá ser ejecutado sobre un gran número de diferentes dispositivos independientemente del sistema operativo que utilice, puesto que únicamente necesita un navegador web para su correcto funcionamiento. De esta forma, los profesores y alumnos podrán hacer uso de la herramienta sin depender de su instalación en el dispositivo a utilizar, incluso pudiendo los alumnos realizar las prácticas sobre dispositivos móviles.

Además, se ha elegido que el proyecto sea desarrollado en el lenguaje de programación Javascript, debido a las facilidades que proporciona para la posterior generación automática mediante el framework de desarrollo Apache Cordova de aplicaciones móviles para las plataformas móviles Android e iOS y de aplicaciones para el sistema operativo Windows. De esta forma, con un único desarrollo, se obtiene un amplio abanico de plataformas sobre las que poder utilizar la herramienta sin tener dependencia de conexión a internet.

Por tanto, la solución elegida es capaz de unificar en una misma herramienta todas las funcionalidades requeridas para la enseñanza de Estructura de computadores con un alto nivel de detalle, con alta disponibilidad al facilitarse su como una herramienta web, y con una gran portabilidad puesto que podrá ser ejecutada sobre un gran número de diversos dispositivos.


\section{Requisitos}
\label{sec:requirements}

Esta sección proporciona una descripción detallada de los requisitos de la aplicación. Para la tarea de la especificación de requisitos, se han seguido las prácticas recomendadas por IEEE \cite{ieee1998}. De acuerdo con estas prácticas, una buena especificación debe abordar la funcionalidad del software, los problemas de rendimiento, las interfaces externas, otras características no funcionales y las limitaciones de diseño o implementación.
Además, la especificación de los requisitos debe ser:

\begin{itemize}

\item \textbf{Completa:} el documento refleja todos los requisitos de software importantes.

\item \textbf{Consistente:} los requisitos no deben generar conflictos entre sí.

\item \textbf{Correcta:} cada requisito es uno que el software se reunirá de acuerdo con las necesidades del usuario.

% every requirement is one that the software shall meet according to the user needs.

\item \textbf{Modificable:} la estructura de la especificación permite cambios en los requisitos de una manera simple, completa y consistente.

\item \textbf{Clasificación basada en la importancia y la estabilidad:} cada requisito debe indicar su importancia y su estabilidad.

\item \textbf{Trazable:} el origen de cada requisito es claro y se puede hacer referencia fácilmente en otras etapas.

\item \textbf{Inequívoco:} cada requisito tiene una sola interpretación.

\item \textbf{Verificable:} Cada requisito debe ser verificable, es decir, existe algún proceso para verificar que el software cumple con cada requisito.

\end{itemize}

Starting from the user requirements, which constitute an informal reference to the product performance that the client expects, we derived the software requirements (in this case, functional requirements and non-functional requirements) that guided the design process with specific information on the functionality of the system and other characteristics. The retrieved requirements were structured according with the following schema:

\begin{itemize}
\item[1.] \textbf{User Requirements} 
	\begin{itemize}
		\item[(a)] \textbf{Capacity:} the requirement describes the expected system functionality as in use cases.
		\item[(b)] \textbf{Restriction:} the requirement specifies constraints or conditions the system must fulfil.	
	\end{itemize}	
\end{itemize}

\begin{itemize}
\item[2.] \textbf{Software Requirements}
	\begin{itemize}
	\item[(a)] 	\textbf{Functional}
		\begin{itemize}
		\item[i.] 	\textbf{Functional:} the requirement describes the basic system functionality and purpose while minimizing ambiguity.
		\item[ii.] 	\textbf{Inverse:} the requirement limits the functionality of the application to clarify its scope.
		\end{itemize}

	\item[(b)] 	\textbf{Non-Functional}
		\begin{itemize}
		\item[i.] 	\textbf{Performance:} the requirement is related to the minimum required performance of the resulting system.
		\item[ii.] 	\textbf{Interface:} the requirement is related to the user interface of the application.
		\item[iii.] 	\textbf{Scalability:} the requirement is related to the ability of the system to adapt to increasing workloads.
		\item[iv.] 	\textbf{Platform:} the requirement specifies the underlying software and hardware platforms in which the system will operate.
		\end{itemize}			
	\end{itemize}
\end{itemize}

Table \ref{tab:requirements_template} provides the template used for requirements specification. Note that for user requirements, the ID format will be UR-XYY, where X indicates the requirement subtype: capacity requirements (C), or restrictions (R). YY corresponds to the requirement number under its subcategory. For software requirements, the ID format SR-X-YZZ will be used, where X indicates if it is a functional (F) or non-functional (NF) requirement, and Y represents its subcategory: functional (F), inverse (I), performance (P), interface (UI), scalability (S), or platform (PL). ZZ corresponds to the requirement number under its subcategory.

\begin{center}
\begin{table*}[htbp]
\centering
\begin{tabular}{@{}p{2.5cm} p{9cm}@{}} 
\toprule
\textbf{ID} 				& Requirement ID. \\
\midrule
\textbf{Name} 			& Requirement name. \\
\midrule
\textbf{Type} 			& Indicates the category in which the requirement would be placed according to the previously described schema. \\
\midrule
\textbf{Origin} 			& Constitutes the requirement source. It might be the user, another requirement or other stakeholders involved in the project. \\
\midrule
\textbf{Priority}		& Indicates the requirement priority according to its importance. A requirement can be identified either as \textit{essential}, \textit{conditional} or \textit{optional}. \\
\midrule
\textbf{Stability} 		& Indicates the requirement variability through the development process, defined as \textit{stable} or \textit{unstable}. \\
\midrule
\textbf{Description} 	& Detailed explanation of the requirement. \\
\bottomrule
\end{tabular}
\caption{Template for requirements specification.}
\label{tab:requirements_template}
\end{table*}
\end{center}

\subsection{Requisitos de Usuario}

This subsection specifies the user requirements.

\begin{center}
\begin{table*}[htbp]
\centering
\begin{tabular}{@{}p{2.5cm} p{9cm}@{}} 
\toprule
\textbf{ID} 				& UR-C01\\
\midrule
\textbf{Name} 			& BOINC projects simulation \\
\midrule
\textbf{Type} 			& Capacity \\
\midrule
\textbf{Origin} 			& User \\
\midrule
\textbf{Priority}		& Essential \\
\midrule
\textbf{Stability} 		& Stable \\
\midrule
\textbf{Description} 	& The application shall simulate real BOINC projects. \\
\bottomrule
\end{tabular}
\caption{User requirement UR-C01.}
\label{tab:urc01}
\end{table*}
\end{center}

\begin{center}
\begin{table*}[htbp]
\centering
\begin{tabular}{@{}p{2.5cm} p{9cm}@{}} 
\toprule
\textbf{ID} 				& UR-C02\\
\midrule
\textbf{Name} 			& Client \gls{scheduling} \\
\midrule
\textbf{Type} 			& Capacity \\
\midrule
\textbf{Origin} 			& User \\
\midrule
\textbf{Priority}		& Essential \\
\midrule
\textbf{Stability} 		& Stable \\
\midrule
\textbf{Description} 	& The client scheduler of the simulator shall follow the actual BOINC client \gls{scheduling}. \\
\bottomrule
\end{tabular}
\caption{User requirement UR-C02.}
\label{tab:urc02}
\end{table*}
\end{center}

\begin{center}
\begin{table*}[htbp]
\centering
\begin{tabular}{@{}p{2.5cm} p{9cm}@{}} 
\toprule
\textbf{ID} 				& UR-C03\\
\midrule
\textbf{Name} 			& Simulation components \\
\midrule
\textbf{Type} 			& Capacity \\
\midrule
\textbf{Origin} 			& User \\
\midrule
\textbf{Priority}		& Essential \\
\midrule
\textbf{Stability} 		& Stable \\
\midrule
\textbf{Description} 	& The simulations shall cover all the elements present in the BOINC infrastructure. \\
\bottomrule
\end{tabular}
\caption{User requirement UR-C03.}
\label{tab:urc03}
\end{table*}
\end{center}

\begin{center}
\begin{table*}[htbp]
\centering
\begin{tabular}{@{}p{2.5cm} p{9cm}@{}} 
\toprule
\textbf{ID} 				& UR-R01\\
\midrule
\textbf{Name} 			& Linux as underlying OS \\
\midrule
\textbf{Type} 			& Restriction \\
\midrule
\textbf{Origin} 			& User \\
\midrule
\textbf{Priority}		& Essential \\
\midrule
\textbf{Stability} 		& Stable \\
\midrule
\textbf{Description} 	& The simulator shall be designed for Linux operating systems. \\
\bottomrule
\end{tabular}
\caption{User requirement UR-R01.}
\label{tab:urr01}
\end{table*}
\end{center}

\begin{center}
\begin{table*}[htbp]
\centering
\begin{tabular}{@{}p{2.5cm} p{9cm}@{}} 
\toprule
\textbf{ID} 				& UR-R02\\
\midrule
\textbf{Name} 			& SimGrid toolkit \\
\midrule
\textbf{Type} 			& Restriction \\
\midrule
\textbf{Origin} 			& User \\
\midrule
\textbf{Priority}		& Essential \\
\midrule
\textbf{Stability} 		& Stable \\
\midrule
\textbf{Description} 	& The application shall use the SimGrid toolkit in order to implement the distributed computing functionalities. \\
\bottomrule
\end{tabular}
\caption{User requirement UR-R02.}
\label{tab:urr02}
\end{table*}
\end{center}

\begin{center}
\begin{table*}[htbp]
\centering
\begin{tabular}{@{}p{2.5cm} p{9cm}@{}} 
\toprule
\textbf{ID} 				& UR-R03\\
\midrule
\textbf{Name} 			& Scalability \\
\midrule
\textbf{Type} 			& Restriction \\
\midrule
\textbf{Origin} 			& User \\
\midrule
\textbf{Priority}		& Essential \\
\midrule
\textbf{Stability} 		& Stable \\
\midrule
\textbf{Description} 	& The simulator shall be scalable (carry out executions by simulating a large number of client hosts). \\
\bottomrule
\end{tabular}
\caption{User requirement UR-R03.}
\label{tab:urr03}
\end{table*}
\end{center}

\clearpage
\subsection{Requisitos Funcionales}

This subsection specifies the functional requirements.

\begin{center}
\begin{table*}[htbp]
\centering
\begin{tabular}{@{}p{2.5cm} p{9cm}@{}} 
\toprule
\textbf{ID} 				& SR-F-F01\\
\midrule
\textbf{Name} 			& Credit calculation \\
\midrule
\textbf{Type} 			& Functional \\
\midrule
\textbf{Origin} 			& UR-C01 \\
\midrule
\textbf{Priority}		& Essential \\
\midrule
\textbf{Stability} 		& Stable \\
\midrule
\textbf{Description} 	& The simulator shall calculate the number of credits granted to each volunteer client analogously to actual BOINC projects. \\
\bottomrule
\end{tabular}
\caption{Functional requirement SR-F-F01.}
\label{tab:srff01}
\end{table*}
\end{center}

\begin{center}
\begin{table*}[htbp]
\centering
\begin{tabular}{@{}p{2.5cm} p{9cm}@{}} 
\toprule
\textbf{ID} 				& SR-F-F02\\
\midrule
\textbf{Name} 			& Collection of statistics \\
\midrule
\textbf{Type} 			& Functional \\
\midrule
\textbf{Origin} 			& UR-C01 \\
\midrule
\textbf{Priority}		& Essential \\
\midrule
\textbf{Stability} 		& Stable \\
\midrule
\textbf{Description} 	& The simulator shall collect, for each project, the same statistics that actual BOINC projects (published in BOINCstats \cite{BOINC2016}). \\
\bottomrule
\end{tabular}
\caption{Functional requirement SR-F-F02.}
\label{tab:srff02}
\end{table*}
\end{center}

\begin{center}
\begin{table*}[htbp]
\centering
\begin{tabular}{@{}p{2.5cm} p{9cm}@{}} 
\toprule
\textbf{ID} 				& SR-F-F03\\
\midrule
\textbf{Name} 			& Almost identical outputs \\
\midrule
\textbf{Type} 			& Functional \\
\midrule
\textbf{Origin} 			& UR-C01 \\
\midrule
\textbf{Priority}		& Essential \\
\midrule
\textbf{Stability} 		& Stable \\
\midrule
\textbf{Description} 	& The outputs of the simulator for existing projects should be almost identical to those published in BOINCstats \cite{BOINC2016}. \\
\bottomrule
\end{tabular}
\caption{Functional requirement SR-F-F03.}
\label{tab:srff03}
\end{table*}
\end{center}

\begin{center}
\begin{table*}[htbp]
\centering
\begin{tabular}{@{}p{2.5cm} p{9cm}@{}} 
\toprule
\textbf{ID} 				& SR-F-F04\\
\midrule
\textbf{Name} 			& Multiple BOINC projects \\
\midrule
\textbf{Type} 			& Functional \\
\midrule
\textbf{Origin} 			& UR-C01 \\
\midrule
\textbf{Priority}		& Essential \\
\midrule
\textbf{Stability} 		& Stable \\
\midrule
\textbf{Description} 	& The simulator shall allow the simulation of different projects simultaneously. \\
\bottomrule
\end{tabular}
\caption{Functional requirement SR-F-F04.}
\label{tab:srff04}
\end{table*}
\end{center}

\begin{center}
\begin{table*}[htbp]
\centering
\begin{tabular}{@{}p{2.5cm} p{9cm}@{}} 
\toprule
\textbf{ID} 				& SR-F-F05\\
\midrule
\textbf{Name} 			& Client scheduler \\
\midrule
\textbf{Type} 			& Functional \\
\midrule
\textbf{Origin} 			& UR-C02 \\
\midrule
\textbf{Priority}		& Essential \\
\midrule
\textbf{Stability} 		& Stable \\
\midrule
\textbf{Description} 	& The client scheduler shall follow the actual BOINC client \gls{scheduling} (described in \cite{anderson2007}). \\
\bottomrule
\end{tabular}
\caption{Functional requirement SR-F-F05.}
\label{tab:srff05}
\end{table*}
\end{center}

\begin{center}
\begin{table*}[htbp]
\centering
\begin{tabular}{@{}p{2.5cm} p{9cm}@{}} 
\toprule
\textbf{ID} 				& SR-F-F06\\
\midrule
\textbf{Name} 			& Realistic simulation elements \\
\midrule
\textbf{Type} 			& Functional \\
\midrule
\textbf{Origin} 			& UR-C03 \\
\midrule
\textbf{Priority}		& Essential \\
\midrule
\textbf{Stability} 		& Stable \\
\midrule
\textbf{Description} 	& All simulations shall include the following elements: tasks, volunteer hosts, servers, data servers, networks, and hosts availability. \\
\bottomrule
\end{tabular}
\caption{Functional requirement SR-F-F06.}
\label{tab:srff06}
\end{table*}
\end{center}


\subsection{Requisitos No-Funcionales}

This subsection specifies the non-functional requirements.

\begin{center}
\begin{table*}[htbp]
\centering
\begin{tabular}{@{}p{2.5cm} p{9cm}@{}} 
\toprule
\textbf{ID} 				& SR-NF-PL01\\
\midrule
\textbf{Name} 			& Ubuntu 14.04 \\
\midrule
\textbf{Type} 			& Platform \\
\midrule
\textbf{Origin} 			& UR-R01 \\
\midrule
\textbf{Priority}		& Essential \\
\midrule
\textbf{Stability} 		& Stable \\
\midrule
\textbf{Description} 	& The simulator shall work on the Ubuntu Linux distribution, version 14.04. \\
\bottomrule
\end{tabular}
\caption{Non-functional requirement SR-NF-PL01.}
\label{tab:srnfpl01}
\end{table*}
\end{center}

\begin{center}
\begin{table*}[htbp]
\centering
\begin{tabular}{@{}p{2.5cm} p{9cm}@{}} 
\toprule
\textbf{ID} 				& SR-NF-PL02\\
\midrule
\textbf{Name} 			& SimGrid MSG API \\
\midrule
\textbf{Type} 			& Platform \\
\midrule
\textbf{Origin} 			& UR-R02 \\
\midrule
\textbf{Priority}		& Essential \\
\midrule
\textbf{Stability} 		& Stable \\
\midrule
\textbf{Description} 	& The implementation, setup and control of the simulations shall be carried out using the MSG API of the SimGrid toolkit. \\
\bottomrule
\end{tabular}
\caption{Non-functional requirement SR-NF-PL02.}
\label{tab:srnfpl02}
\end{table*}
\end{center}

\begin{center}
\begin{table*}[htbp]
\centering
\begin{tabular}{@{}p{2.5cm} p{9cm}@{}} 
\toprule
\textbf{ID} 				& SR-NF-PL03\\
\midrule
\textbf{Name} 			& C programming language \\
\midrule
\textbf{Type} 			& Platform\\
\midrule
\textbf{Origin} 			& UR-R03 \\
\midrule
\textbf{Priority}		& Essential \\
\midrule
\textbf{Stability} 		& Stable \\
\midrule
\textbf{Description} 	& The simulator shall be written in the C programming language. \\
\bottomrule
\end{tabular}
\caption{Non-functional requirement SR-NF-PL03.}
\label{tab:srnfpl03}
\end{table*}
\end{center}

\begin{center}
\begin{table*}[htbp]
\centering
\begin{tabular}{@{}p{2.5cm} p{9cm}@{}} 
\toprule
\textbf{ID} 				& SR-NF-S01\\
\midrule
\textbf{Name} 			& Large simulations \\
\midrule
\textbf{Type} 			& Scalability \\
\midrule
\textbf{Origin} 			& UR-R03 \\
\midrule
\textbf{Priority}		& Essential \\
\midrule
\textbf{Stability} 		& Stable \\
\midrule
\textbf{Description} 	& The application must be able to perform simulations with more than 100,000 hosts in a machine with at least 8 GB of \gls{ram}. \\
\bottomrule
\end{tabular}
\caption{Non-functional requirement SR-NF-S01.}
\label{tab:srnfs01}
\end{table*}
\end{center}

\begin{center}
\begin{table*}[htbp]
\centering
\begin{tabular}{@{}p{2.5cm} p{9cm}@{}} 
\toprule
\textbf{ID} 				& SR-NF-P01\\
\midrule
\textbf{Name} 			& Linear-time execution \\
\midrule
\textbf{Type} 			& Performance \\
\midrule
\textbf{Origin} 			& UR-R03 \\
\midrule
\textbf{Priority}		& Conditional \\
\midrule
\textbf{Stability} 		& Stable \\
\midrule
\textbf{Description} 	& Runtime of the simulator must be linear (approximately) in the number of hosts. \\
\bottomrule
\end{tabular}
\caption{Non-functional requirement SR-NF-P01.}
\label{tab:srnfp01}
\end{table*}
\end{center}

\begin{center}
\begin{table*}[htbp]
\centering
\begin{tabular}{@{}p{2.5cm} p{9cm}@{}} 
\toprule
\textbf{ID} 				& SR-NF-UI01\\
\midrule
\textbf{Name} 			& Simulation parameters \\
\midrule
\textbf{Type} 			& Interface \\
\midrule
\textbf{Origin} 			& Analyst \\
\midrule
\textbf{Priority}		& Essential \\
\midrule
\textbf{Stability} 		& Stable \\
\midrule
\textbf{Description} 	& To perform simulations, users only need to specify the simulation parameters in an \gls{xml} file. \\
\bottomrule
\end{tabular}
\caption{Non-functional requirement SR-NF-UI01.}
\label{tab:srnfui01}
\end{table*}
\end{center}

\begin{center}
\begin{table*}[htbp]
\centering
\begin{tabular}{@{}p{2.5cm} p{9cm}@{}}  
\toprule
\textbf{ID} 				& SR-NF-UI02\\
\midrule
\textbf{Name} 			& Progress bar \\
\midrule
\textbf{Type} 			& Interface \\
\midrule
\textbf{Origin} 			& Analyst \\
\midrule
\textbf{Priority}		& Conditional \\
\midrule
\textbf{Stability} 		& Stable \\
\midrule
\textbf{Description} 	& The simulations should include a progress bar. \\
\bottomrule
\end{tabular}
\caption{Non-functional requirement SR-NF-UI02.}
\label{tab:srnfui02}
\end{table*}
\end{center}

\section{Marco Regulador}
\label{sec:regulatory_framework}

This section discusses the necessary constraints taking into account the regulatory \gls{framework}. Specifically, the legal restrictions applicable to the simulator are specified.

\subsection{Restricciones Legales}
\label{sec:legal_constraints}

In the real BOINC system, users must be registered, and \gls{boinc} databases handle confidential information from users, so it is necessary to ensure that third parties can not access that information. One solution is to encrypt the information transmitted following some cryptographic \gls{protocol}. In Spain, this requirement is specified in the article 104 of the RD 1720/2007 \cite{boe2008}, which deals with the Spanish Data Protection Law. 

In contrast, the developed application does not use private data from users, and neither transmits any confidential information to third-parties, because it is just a simulator that does not even require Internet access.

On the other hand, it is crucial that our simulator be available as an \gls{opensource} software. We want it to be such that anyone can redistribute the code or modify it by the terms of the GNU Lesser General Public License (LGPL) \cite{gnulgpl}. To do this, our simulator is available on the following website: \url{https://www.arcos.inf.uc3m.es/~combos/}.

\afterpage{\blankpage} % blank page