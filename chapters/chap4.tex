\lhead[\thepage]{CHAPTER \thechapter. DESIGN}
\chead[]{}
\rhead[A Complete Simulator for Volunteer Computing Environments\leftmark]{\thepage}
\renewcommand{\headrulewidth}{0.5pt}

\lfoot[]{}
\cfoot[]{}
\rfoot[]{}
\renewcommand{\footrulewidth}{0pt}

%% This is an example first chapter.  You should put chapter/appendix that you
%% write into a separate file, and add a line \include{yourfilename} to
%% main.tex, where `yourfilename.tex' is the name of the chapter/appendix file.
%% You can process specific files by typing their names in at the 
%% \files=
%% prompt when you run the file main.tex through LaTeX.
\chapter{Diseño}
\label{ch:design}
\markboth{}{DESIGN}

This chapter provides a complete description of the developed simulator, including the internal architecture and the different software components. \gls{comsimboinc} is a complete simulator of BOINC infrastructures that simulates the behavior of all componentes involved: projects,  servers, network, \gls{scheduling}, redundant computing, and volunteer nodes. In this chapter we describe all the simulator components (Section \ref{sec:simulator_components}, \textit{\nameref{sec:simulator_components}}) and present the policies of the client scheduler (Section \ref{sec:local_scheduling_policies}, \textit{\nameref{sec:local_scheduling_policies}}).


\section{Componentes del Simulador}
\label{sec:simulator_components}


\subsection{Interfaz Gráfica}


\subsection{Core}


\subsection{Compilador}

