\lhead[\thepage]{CHAPTER \thechapter. DESIGN}
\chead[]{}
\rhead[A Complete Simulator for Volunteer Computing Environments\leftmark]{\thepage}
\renewcommand{\headrulewidth}{0.5pt}

\lfoot[]{}
\cfoot[]{}
\rfoot[]{}
\renewcommand{\footrulewidth}{0pt}

%% This is an example first chapter.  You should put chapter/appendix that you
%% write into a separate file, and add a line \include{yourfilename} to
%% main.tex, where `yourfilename.tex' is the name of the chapter/appendix file.
%% You can process specific files by typing their names in at the 
%% \files=
%% prompt when you run the file main.tex through LaTeX.
\chapter{Diseño}
\label{ch:design}
\markboth{}{DESIGN}

En este capítulo se realiza una descripción completa del simulador desarrollado, incluyendo la arquitectura interna y los diferentes componentes software.

La sección \ref{sec:solution_selection} discute la solución elegida y la compara con las alternativas consideradas. La sección \ref{sec:simulator_architecture} describe cada uno de los componentes que componen el simulador.

\section{Solución elegida}
\label{sec:solution_selection}

Para que los profesores de la asignatura Estructura de Computadores puedan hacer uso de una herramienta que sirva de ayuda para la explicación de los conceptos teóricos de la asignatura, y los alumnos puedan utilizarla para comprender estos conceptos y realizar posteriormente las prácticas de la asignatura, se propone el diseño e implementación de una herramienta web que simule con realismo en funcionamiento de un procesador elemental con unidad de control microprogramable.

Este simulador, será desarrollado como una herramienta web debido a la portabilidad que proporciona, ya que podrá ser ejecutado sobre un gran número de diferentes dispositivos independientemente del sistema operativo que utilice, puesto que únicamente necesita un navegador web para su correcto funcionamiento. De esta forma, los profesores y alumnos podrán hacer uso de la herramienta sin depender de su instalación en el dispositivo a utilizar, incluso pudiendo los alumnos realizar las prácticas sobre dispositivos móviles.

Para lograr dicha portabilidad, el simulador ha sido desarrollado en HTML5 (HTML + JavaScript + CSS) haciendo posible su ejecución en cualquier plataforma (smartphones, tablet, PC, etc.) que pueden ejecutar Microsoft Edge, Mozilla Firefox, Google Chrome o Safari. Además, la herramienta depende de los siguientes frameworks/bibliotecas: JQuery, JQueryUI, JQuery Mobile, Knockout y BootStrap.

Por tanto, la solución elegida es capaz de unificar en una misma herramienta todas las funcionalidades requeridas para la enseñanza de Estructura de computadores con un alto nivel de detalle, con alta disponibilidad al facilitarse su como una herramienta web, y con una gran portabilidad puesto que podrá ser ejecutada sobre un gran número de diversos dispositivos.



\section{Arquitectura de WepSIM}
\label{sec:simulator_architecture}

La arquitectura de la solución presentada en este trabajo consta de tres elementos principales:

\begin{itemize}
\item Modelo hardware: permite definir el hardware a usar.
\item Modelo software: permite definir el juego de instrucciones a utilizar.
\item Motor de simulación: simula el funcionamiento del hardware ejecutando el microcódigo/lenguaje máquina definido con anterioridad.
\end{itemize}

El modelo hardware permite definir los distintos elementos típicos de un computador (memoria principal, procesador, etc.) de una forma modular. La forma de definir estos elementos equilibra dos objetivos contrapuestos: es suficientemente completa como para imitar los principales aspectos de la realidad, pero es lo suficientemente mínima para facilitar su uso. Ante todo se persigue que sea una herramienta didáctica.

El modelo software permite definir el microcódigo y el ensamblador basado en este microcódigo de la forma tan intuitiva posible. El ensamblador a usar viene dado por un conjunto de instrucciones que puede ser definido por el usuario e intenta ser lo suficientemente flexible como para poder definir diferentes tipos y juegos de instrucciones, como por ejemplo MIPS o ARM.

El tercer elemento de la arquitectura propuesta es un motor que toma como entrada el modelo hardware descrito y el modelo software de trabajo, y se encarga de mostrar el funcionamiento del hardware con el software dado.

\subsection{Modelo hardware}


\subsection{Modelo software}


\subsection{Motor del simulador}

