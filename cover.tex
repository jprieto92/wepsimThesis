% -*-latex-*-
% 
% For questions, comments, concerns or complaints:
% thesis@mit.edu
% 
%
% $Log: cover.tex,v $
% Revision 1.8  2008/05/13 15:02:15  jdreed
% Degree month is June, not May.  Added note about prevdegrees.
% Arthur Smith's title updated
%
% Revision 1.7  2001/02/08 18:53:16  boojum
% changed some \newpages to \cleardoublepages
%
% Revision 1.6  1999/10/21 14:49:31  boojum
% changed comment referring to documentstyle
%
% Revision 1.5  1999/10/21 14:39:04  boojum
% *** empty log message ***
%
% Revision 1.4  1997/04/18  17:54:10  othomas
% added page numbers on abstract and cover, and made 1 abstract
% page the default rather than 2.  (anne hunter tells me this
% is the new institute standard.)
%
% Revision 1.4  1997/04/18  17:54:10  othomas
% added page numbers on abstract and cover, and made 1 abstract
% page the default rather than 2.  (anne hunter tells me this
% is the new institute standard.)
%
% Revision 1.3  93/05/17  17:06:29  starflt
% Added acknowledgements section (suggested by tompalka)
% 
% Revision 1.2  92/04/22  13:13:13  epeisach
% Fixes for 1991 course 6 requirements
% Phrase "and to grant others the right to do so" has been added to 
% permission clause
% Second copy of abstract is not counted as separate pages so numbering works
% out
% 
% Revision 1.1  92/04/22  13:08:20  epeisach

% NOTE:
% These templates make an effort to conform to the MIT Thesis specifications,
% however the specifications can change.  We recommend that you verify the
% layout of your title page with your thesis advisor and/or the MIT 
% Libraries before printing your final copy.

\title{WepSIM: Simulador de procesador elemental con unidad de control microprogramada}

\author{Javier Prieto Cepeda}
% If you wish to list your previous degrees on the cover page, use the 
% previous degrees command:
%       \prevdegrees{A.A., Harvard University (1985)}
% You can use the \\ command to list multiple previous degrees
%       \prevdegrees{B.S., University of California (1978) \\
%                    S.M., Massachusetts Institute of Technology (1981)}
\department{Department of Electrical Engineering and Computer Science}

% If the thesis is for two degrees simultaneously, list them both
% separated by \and like this:
% \degree{Doctor of Philosophy \and Master of Science}
\degree{Bachelor of Science in Computer Science and Engineering}

% As of the 2007-08 academic year, valid degree months are September, 
% February, or June.  The default is June.
\degreemonth{June}
\degreeyear{1990}
\thesisdate{May 18, 1990}

%% By default, the thesis will be copyrighted to MIT.  If you need to copyright
%% the thesis to yourself, just specify the `vi' documentclass option.  If for
%% some reason you want to exactly specify the copyright notice text, you can
%% use the \copyrightnoticetext command.  
%\copyrightnoticetext{\copyright IBM, 1990.  Do not open till Xmas.}

% If there is more than one supervisor, use the \supervisor command
% once for each.
\supervisor{Félix García Carballeira}{Full Professor}

% This is the department committee chairman, not the thesis committee
% chairman.  You should replace this with your Department's Committee
% Chairman.
\chairman{Arthur C. Smith}{Chairman, Department Committee on Graduate Theses}

% Make the titlepage based on the above information.  If you need
% something special and can't use the standard form, you can specify
% the exact text of the titlepage yourself.  Put it in a titlepage
% environment and leave blank lines where you want vertical space.
% The spaces will be adjusted to fill the entire page.  The dotted
% lines for the signatures are made with the \signature command.
%\maketitle

% The abstractpage environment sets up everything on the page except
% the text itself.  The title and other header material are put at the
% top of the page, and the supervisors are listed at the bottom.  A
% new page is begun both before and after.  Of course, an abstract may
% be more than one page itself.  If you need more control over the
% format of the page, you can use the abstract environment, which puts
% the word "Abstract" at the beginning and single spaces its text.

%% You can either \input (*not* \include) your abstract file, or you can put
%% the text of the abstract directly between the \begin{abstractpage} and
%% \end{abstractpage} commands.

% First copy: start a new page, and save the page number.
\afterpage{\blankpage} % blank page
\clearpage

\thispagestyle{empty}
\vspace*{\fill} 
\begin{quote}
\epigraph{\large \textit{If you would not be forgotten as soon as you are dead and rotten, either write things worth reading, or do things worth the writing.}}{\large \flushright \textbf{Benjamin Franklin}}
\end{quote}
\vspace*{\fill} 

\afterpage{\blankpage} % blank page

\chapter*{Agradecimientos}
\addcontentsline{toc}{chapter}{\textit{Agradecimientos}}%
Es muy difícil incluir en tan pocas palabras a todas aquellas personas que han puesto su granito de arena a lo largo de estos años para que a día de hoy, haya logrado llegar hasta aquí. Es por ello que si me dejo a alguien sin incluir, me disculpe y no dude en sentirse parte de estas palabras.

En primer lugar, quiero agradecer de corazón a mis padres Tomás y Raquel el haberme educado y enseñado a elegir aquel camino que me hiciera feliz de verdad, dándome la oportunidad de estudiar con su esfuerzo y sacrificio. Aquí también estás tu David, pese a llegar unos años después de mi, no paras de enseñarme día a día nuevas lecciones, de los pequeños también se aprende. Gracias.

También debo de dar las gracias a la persona que ha tenido que aguantar estos 5 años las consecuencias de estudiar esta carrera. Míriam, gracias por tu paciencia, tus consejos y tus ánimos. 

Mis abuelos también han sido parte de esas personas que han sumado su granito de arena enseñándome y aconsejándome en todo momento, ejerciendo de guías. Sé que vosotros habéis sufrido también mucho en estos años, pero aquí está el premio.

Por otro lado, debo de agradecer a dos personas el darme la oportunidad de realizar este proyecto. Félix y Álex, gracias por este tiempo en el que tanto he podido aprender y en el que tanto me habéis ayudado.

También son parte de este proyecto los grandes amigos que he hecho en la travesía por el grupo de investigación ARCOS. Uno de ellos además, después de compartir asiento en clase. Saúl, gracias por tu ayuda y por los buenísimos momentos que hemos compartido. Carlos, nos quedan más carreras por hacer juntos. Fran, Estefanía, Silvina, Cascajo, Jesús Cristina, Garci, Pablo, David, Rafa y Alfredo, gracias por acogerme tan bien en el laboratorio y por ayudarme en todo momento.

Agradecer también a Jesús, Javi Blas y al resto de personas de componen el grupo de investigación ARCOS la ayuda prestada durante este tiempo.

Por otro lado, también quiero destacar a aquellos compañeros de prácticas y amigos que he hecho a lo largo de estos años en la carrera, y que siempre recordaré: Sergio, Planet, Rubén, Álvaro, Álex, Guille, Juanlu, Sandra y Marin.



\thispagestyle{empty}

%%%%%%%%%%%%%%%%%%%%%%%%%%%%%%%%%%%%%%%%%%%%%%%%%%%%%%%%%%%%%%%%%%%%%%
% -*-latex-*-


% Uncomment the next line if you do NOT want a page number on your
% abstract and acknowledgments pages.
% \pagestyle{empty}
%\setcounter{savepage}{\thepage}
\begin{abstractpage}
\addcontentsline{toc}{chapter}{Abstract}%
% $Log: abstract.tex,v $
% Revision 1.1  93/05/14  14:56:25  starflt
% Initial revision
% 
% Revision 1.1  90/05/04  10:41:01  lwvanels
% Initial revision
% 
%
%% The text of your abstract and nothing else (other than comments) goes here.
%% It will be single-spaced and the rest of the text that is supposed to go on
%% the abstract page will be generated by the abstractpage environment.  This
%% file should be \input (not \include 'd) from cover.tex.
\thispagestyle{plain}

%Volunteer computing is a type of distributed computing in which ordinary people donate their idle computer time to science projects like SETI@home, Climateprediction.net and many others. \acrshort{boinc} provides a complete \gls{middleware} system for volunteer computing, and it became  generalized as a platform for distributed applications in areas as diverse as mathematics, medicine, molecular biology, climatology, environmental science, and astrophysics. In this document we present the whole development process of \acrshort{comsimboinc}, a complete simulator of the \acrshort{boinc} infrastructure. Although there are other \acrshort{boinc} simulators, our intention was to create a complete simulator that, unlike the existing ones, could simulate realistic scenarios taking into account the whole \acrshort{boinc} infrastructure, that other simulators do not consider: projects, servers, network, redundant computing, \gls{scheduling}, and volunteer nodes. The output of the simulations allows us to analyze a wide range of statistical results, such as the \gls{throughput} of each project, the number of jobs executed by the clients, the total credit granted and the average occupation of the \acrshort{boinc} servers. This bachelor thesis describes the design of \acrshort{comsimboinc} and the results of the validation performed. This validation compares the results obtained in \acrshort{comsimboinc} with the real ones of three different \acrshort{boinc} projects (Einstein@home, SETI@home and LHC@home). Besides, we analyze the performance of the simulator in terms of memory usage and execution time. This document also shows that our simulator can guide the design of \acrshort{boinc} projects, describing some case studies using \acrshort{comsimboinc} that could help designers verify the feasibility of \acrshort{boinc} projects.

\vspace{0.7cm}

\textbf{Keywords:} \acrshort{boinc} $\cdot$ Simulation $\cdot$ \Gls{throughput} $\cdot$ Volunteer Computing

\end{abstractpage}

% Additional copy: start a new page, and reset the page number.  This way,
% the second copy of the abstract is not counted as separate pages.
% Uncomment the next 6 lines if you need two copies of the abstract
% page.
% \setcounter{page}{\thesavepage}
% \begin{abstractpage}
% % $Log: abstract.tex,v $
% Revision 1.1  93/05/14  14:56:25  starflt
% Initial revision
% 
% Revision 1.1  90/05/04  10:41:01  lwvanels
% Initial revision
% 
%
%% The text of your abstract and nothing else (other than comments) goes here.
%% It will be single-spaced and the rest of the text that is supposed to go on
%% the abstract page will be generated by the abstractpage environment.  This
%% file should be \input (not \include 'd) from cover.tex.
\thispagestyle{plain}

%Volunteer computing is a type of distributed computing in which ordinary people donate their idle computer time to science projects like SETI@home, Climateprediction.net and many others. \acrshort{boinc} provides a complete \gls{middleware} system for volunteer computing, and it became  generalized as a platform for distributed applications in areas as diverse as mathematics, medicine, molecular biology, climatology, environmental science, and astrophysics. In this document we present the whole development process of \acrshort{comsimboinc}, a complete simulator of the \acrshort{boinc} infrastructure. Although there are other \acrshort{boinc} simulators, our intention was to create a complete simulator that, unlike the existing ones, could simulate realistic scenarios taking into account the whole \acrshort{boinc} infrastructure, that other simulators do not consider: projects, servers, network, redundant computing, \gls{scheduling}, and volunteer nodes. The output of the simulations allows us to analyze a wide range of statistical results, such as the \gls{throughput} of each project, the number of jobs executed by the clients, the total credit granted and the average occupation of the \acrshort{boinc} servers. This bachelor thesis describes the design of \acrshort{comsimboinc} and the results of the validation performed. This validation compares the results obtained in \acrshort{comsimboinc} with the real ones of three different \acrshort{boinc} projects (Einstein@home, SETI@home and LHC@home). Besides, we analyze the performance of the simulator in terms of memory usage and execution time. This document also shows that our simulator can guide the design of \acrshort{boinc} projects, describing some case studies using \acrshort{comsimboinc} that could help designers verify the feasibility of \acrshort{boinc} projects.

\vspace{0.7cm}

\textbf{Keywords:} \acrshort{boinc} $\cdot$ Simulation $\cdot$ \Gls{throughput} $\cdot$ Volunteer Computing

% \end{abstractpage}

\afterpage{\blankpage} % blank page
