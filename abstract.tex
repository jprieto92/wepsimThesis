% $Log: abstract.tex,v $
% Revision 1.1  93/05/14  14:56:25  starflt
% Initial revision
% 
% Revision 1.1  90/05/04  10:41:01  lwvanels
% Initial revision
% 
%
%% The text of your abstract and nothing else (other than comments) goes here.
%% It will be single-spaced and the rest of the text that is supposed to go on
%% the abstract page will be generated by the abstractpage environment.  This
%% file should be \input (not \include 'd) from cover.tex.
\thispagestyle{plain}

WepSIM es un simulador de un procesador elemental con unidad de control microprogramada basado en el procesador diseñado por el personal del grupo de investigación ARCOS del Departamento de Informática de la Universidad Carlos III de Madrid para la docencia en la asignatura Estructura de Computadores. 
   Con esta herramienta se ofrece una visión integrada de la microprogramación de un computador y la programación en lenguaje ensamblador, dando la posibilidad de especificar distintos juegos de instrucciones y ofreciendo un nivel de detalle a nivel de ciclo de reloj. 
   Gracias al diseño modular propuesto en el cual está basado este simulador, es posible añadir, modificar o quitar elementos existentes al modelo hardware diseñado. 
   Debido a que solo precisa de un navegador web, es posible utilizarlo en casi cualquier momento y dispositivo.
   Todas estas características, hacen de WepSIM el primer simulador docente que unifica la microprogramación y la programación en lenguaje ensamblador permitiendo una fácil personalización del modelo hardware a simular y el uso de diferentes juegos de instrucciones. Mediante este simulador se busca facilitar la enseñanza y aprendizaje en el área docente de Estructura y Arquitectura de Computadores.
\vspace{0.7cm}

\textbf{Palabras clave:} \acrshort{mips} $\cdot$ Simulación $\cdot$ Ensamblador $\cdot$ Microprogramación
