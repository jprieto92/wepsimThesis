% $Log: abstract.tex,v $
% Revision 1.1  93/05/14  14:56:25  starflt
% Initial revision
% 
% Revision 1.1  90/05/04  10:41:01  lwvanels
% Initial revision
% 
%
%% The text of your abstract and nothing else (other than comments) goes here.
%% It will be single-spaced and the rest of the text that is supposed to go on
%% the abstract page will be generated by the abstractpage environment.  This
%% file should be \input (not \include 'd) from cover.tex.
\thispagestyle{plain}

%Volunteer computing is a type of distributed computing in which ordinary people donate their idle computer time to science projects like SETI@home, Climateprediction.net and many others. \acrshort{boinc} provides a complete \gls{middleware} system for volunteer computing, and it became  generalized as a platform for distributed applications in areas as diverse as mathematics, medicine, molecular biology, climatology, environmental science, and astrophysics. In this document we present the whole development process of \acrshort{comsimboinc}, a complete simulator of the \acrshort{boinc} infrastructure. Although there are other \acrshort{boinc} simulators, our intention was to create a complete simulator that, unlike the existing ones, could simulate realistic scenarios taking into account the whole \acrshort{boinc} infrastructure, that other simulators do not consider: projects, servers, network, redundant computing, \gls{scheduling}, and volunteer nodes. The output of the simulations allows us to analyze a wide range of statistical results, such as the \gls{throughput} of each project, the number of jobs executed by the clients, the total credit granted and the average occupation of the \acrshort{boinc} servers. This bachelor thesis describes the design of \acrshort{comsimboinc} and the results of the validation performed. This validation compares the results obtained in \acrshort{comsimboinc} with the real ones of three different \acrshort{boinc} projects (Einstein@home, SETI@home and LHC@home). Besides, we analyze the performance of the simulator in terms of memory usage and execution time. This document also shows that our simulator can guide the design of \acrshort{boinc} projects, describing some case studies using \acrshort{comsimboinc} that could help designers verify the feasibility of \acrshort{boinc} projects.

\vspace{0.7cm}

\textbf{Keywords:} \acrshort{mips} $\cdot$ Simulation $\cdot$ \Gls{assembler} $\cdot$ Microprogramming
