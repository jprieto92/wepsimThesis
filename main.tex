%%            assignment before submission.  If you notice any
%%            discrepancies between these templates and the 
%%            MIT Libraries' specs, please let us know
%%            by e-mailing thesis@mit.edu

%% The documentclass options along with the pagestyle can be used to generate
%% a technical report, a draft copy, or a regular thesis.  You may need to
%% re-specify the pagestyle after you \include  cover.tex.  For more
%% information, see the first few lines of mitthesis.cls. 

%\documentclass[12pt,vi,twoside]{mitthesis}
%%
%%  If you want your thesis copyright to you instead of MIT, use the
%%  ``vi'' option, as above.
%%
%\documentclass[12pt,twoside,leftblank]{mitthesis}
%%
%% If you want blank pages before new chapters to be labelled ``This
%% Page Intentionally Left Blank'', use the ``leftblank'' option, as
%% above. 

\documentclass[11pt,twoside]{mitthesis}
\usepackage{lgrind, afterpage}
\usepackage{fancyhdr}
\usepackage{comment}
\usepackage[sc]{mathpazo}
%% These have been added at the request of the MIT Libraries, because
%% some PDF conversions mess up the ligatures.  -LB, 1/22/2014
\usepackage{cmap}
\usepackage{tabto}
\usepackage[T1]{fontenc}
\usepackage[spanish,es-tabla,es-nosectiondot]{babel} % spanish
%\usepackage[english]{babel}
\usepackage{graphicx}
\usepackage{appendix}
\usepackage{eurosym}
\usepackage{subcaption}
\usepackage{epigraph}
\usepackage{amsmath} % equations
\usepackage{mathtools}
\usepackage{algorithm}
\usepackage{algorithmicx}
\usepackage{algpseudocode}
\usepackage[textfont=it]{caption}
\usepackage[hidelinks]{hyperref}
\usepackage[acronym]{glossaries}
\usepackage{array,booktabs,makecell,rotating,ragged2e} % nice tables
\usepackage{pifont}	% check
\usepackage{listings} % linux commands
\usepackage[utf8]{inputenc}
\makeglossaries
\pagestyle{plain}

  \makeglossaries
  
% Traceability matrix
\renewcommand*\theadfont{\bfseries}
\settowidth\rotheadsize{\theadfont Infrastructure}
\renewcommand\theadgape{}
\renewcommand\theadalign{lc}
\renewcommand\rotheadgape{}

% epigraph
\setlength\epigraphwidth{8cm}
\setlength\epigraphrule{0pt}

\renewcommand{\listfigurename}{Lista de Figuras}
\renewcommand{\listtablename}{Lista de Tablas}
\renewcommand{\contentsname}{Lista de Contenidos}
\renewcommand{\figurename}{\textbf{\textit{Figura}}}

% pseudocode
\makeatletter
\renewcommand{\ALG@name}{Pseudocódigo}
\makeatother
\renewcommand{\thealgorithm}{\thechapter.\arabic{algorithm}}% Algorithm # is <chapter>.<algorithm>
\algdef{SE}[DOWHILE]{Do}{doWhile}{\algorithmicdo}[1]{\algorithmicwhile\ #1}%
\makeatletter\@addtoreset{algorithm}{chapter}\makeatother
\makeatletter
\renewcommand{\ALG@beginalgorithmic}{\normalsize}
\makeatother

% xml
\usepackage{color}
\definecolor{gray}{rgb}{0.4,0.4,0.4}
\definecolor{darkblue}{rgb}{0.0,0.0,0.6}
\definecolor{cyan}{rgb}{0.0,0.6,0.6}

\lstset{
  basicstyle=\ttfamily,
  columns=fullflexible,
  showstringspaces=false,
  commentstyle=\color{gray}\upshape
}

\lstdefinelanguage{XML}
{
  morestring=[b]",
  morestring=[s]{>}{<},
  morecomment=[s]{<?}{?>},
  stringstyle=\color{black},
  identifierstyle=\color{darkblue},
  keywordstyle=\color{cyan},
  morekeywords={xmlns,version,type}% list your attributes here
}

\usepackage{etoolbox}

% hyphenation
\hyphenation {ma-yo-res}


\makeatletter
\patchcmd{\epigraph}{\@epitext{#1}}{\itshape\@epitext{#1}}{}{}
\makeatother

\newcommand{\ra}[1]{\renewcommand{\arraystretch}{#1}}
\newcolumntype{L}[1]{>{\raggedright\let\newline\\\arraybackslash\hspace{0pt}}m{#1}}
\newcolumntype{C}[1]{>{\centering\let\newline\\\arraybackslash\hspace{0pt}}m{#1}}
\newcolumntype{R}[1]{>{\raggedleft\let\newline\\\arraybackslash\hspace{0pt}}m{#1}}

% encabezados
\lhead[\thepage]{\rightmark}
\chead[]{}
\rhead[WepSIM: Simulador de procesador elemental con unidad de control microprogramada\leftmark]{\thepage}
\renewcommand{\headrulewidth}{0.5pt}

\newcommand\blankpage{%
    \null
    \thispagestyle{empty}%
    \newpage}

% pie de pagina
\lfoot[]{}
\cfoot[]{}
\rfoot[]{}
\renewcommand{\footrulewidth}{0pt}

% primera pagina de un capitulo
\fancypagestyle{plain}{
\fancyhead[L]{}
\fancyhead[C]{}
\fancyhead[R]{}
\fancyfoot[L]{}
\fancyfoot[C]{\thepage}
\fancyfoot[R]{}
\renewcommand{\headrulewidth}{0pt}
\renewcommand{\footrulewidth}{0pt}
\cfoot{\thepage}
}

\pagestyle{fancy}

% Authors margins
\def\changemargin#1#2{\list{}{\rightmargin#2\leftmargin#1}\item[]}
\let\endchangemargin=\endlist 

% Glossary

\newglossaryentry{ensamblador}{name=ensamblador, description={lenguaje de programación de bajo nivel.}}

\newglossaryentry{computerstructure}{name=Estructura de Computadores, description={Asignatura cuyo objetivo es describir el funcionamiento básico de un computador}}

\newglossaryentry{microcodigo}{name=microcódigo, description={Instrucciones o estructuras de datos implicados en la implementación  de lenguaje máquina}}

\newglossaryentry{hardware}{name=hardware, description={Conjunto de elementos físicos o materiales que constituyen una computadora o un sistema informático}}

\newglossaryentry{software}{name=software, description={Conjunto de programas y rutinas que permiten a la computadora realizar determinadas tareas}}

\newglossaryentry{framework}{name=framework, description={Estructura real o conceptual destinada a servir de soporte o guía para la construcción de algo que expande la estructura en algo útil}}

\newglossaryentry{pipeline}{name=pipeline, description={Arquitectura basada en la transformación de un flujo de datos en un proceso comprendido por varias fases secuenciales, siendo la entrada de cada una la salida anterior}}

\newglossaryentry{lexyacc}{name=LEX and YACC, description={Programa para generar analizadores léxicos y sintácticos}}

\newglossaryentry{opensource}{name=open-source, description={Software desarrollado y distribuido libremente}}

\newglossaryentry{protocol}{name=protocolo, description={Es el conjunto especial de reglas que terminan los puntos en una conexión de telecomunicación cuando se comunican. Los protocolos especifican interacciones entre las entidades comunicantes}}

% Acronyms

\newacronym{arcos}{ARCOS}{Grupo de Aquitectura de Computadores, Universidad Carlos III de Madrid}

\newacronym{boinc}{BOINC}{Berkeley Open Infrastructure for Network Computing}

\newacronym{comsimboinc}{ComBoS}{Complete BOINC Simulator}

\newacronym{ebs}{EBS}{Elastic Block Storage}

\newacronym{edf}{EDF}{Earliest Deadline First}

\newacronym{flops}{FLOPS}{Floating-point Operations per Second}

\newacronym{ftp}{FTP}{File Transfer Protocol}

\newacronym{haas}{HaaS}{Hardware as a Service}

\newacronym{http}{HTTP}{Hypertext Transfer Protocol}

\newacronym{html}{HTML}{Hyper-Text Markup Language}

\newacronym{hpc}{HPC}{High-Performance Computing}

\newacronym{iaas}{IaaS}{Infrastructure as a Service}

\newacronym{irc}{IRC}{Internet Relay Chat}

\newacronym{iva}{IVA}{Impuesto Sobre el Valor Añadido}

\newacronym{mpi}{MPI}{Message Passing Interface}

\newacronym{nat}{NAT}{Network Address Translation}

\newacronym{paas}{PaaS}{Platform as a Service}

\newacronym{pc}{PC}{Personal Computer}

\newacronym{p2p}{P2P}{Peer-to-Peer}

\newacronym{ram}{RAM}{Random-Access Memory}

\newacronym{rpc}{RPC}{Remote Procedure Call}

\newacronym{saas}{SaaS}{Software as a Service}

\newacronym{www}{WWW}{World Wide Web}

\newacronym{xml}{XML}{Extensible Markup Language}


% Acronyms

%% This bit allows you to either specify only the files which you wish to
%% process, or `all' to process all files which you \include.
%% Krishna Sethuraman (1990).

%\typein [\files]{Enter file names to process, (chap1,chap2 ...), or `all' to
%process all files:}
\def\all{all}
%\ifx\files\all \typeout{Including all files.} \else \typeout{Including only \files.} \includeonly{\files} \fi

\begin{document}
\pagenumbering{roman} %

\begin{titlepage}
    \begin{center}
        \vspace*{1cm}
     
        \Large
        Universidad Carlos III de Madrid\\
        Escuela Politécnica Superior\\
        Grado en Ingeniería Informática
        
        \vspace{0.8cm}
        
        \centering        
        \includegraphics[width=0.3\textwidth]{figures/logo-nuevo}
        
        \vspace{1.5cm}
        
        \LARGE
        Trabajo Fin de Grado
        
        \Huge
        %\textbf{A Volunteer Computing Environments Simulator}
        \textbf{WepSIM: Simulador de un procesador elemental con unidad de control microprogramada}
        
        \vspace{0.5cm}
        
        \vspace{1.5cm}
        
        \Large
        \begin{changemargin}{3.8cm}{3.8cm}
        Autor:\hfill Javier Prieto Cepeda\\
        Tutor:\hfill Félix García Carballeira\\
        \end{changemargin}
		\begin{changemargin}{3.8cm}{3.8cm}
		\begin{flushright}		        
        \vspace{0.5cm}
        Leganés, Madrid, España\\
        Junio 2017
 		\end{flushright}
        \end{changemargin}
        
        \vfill
        
    \end{center}
\end{titlepage}

% -*-latex-*-
% 
% For questions, comments, concerns or complaints:
% thesis@mit.edu
% 
%
% $Log: cover.tex,v $
% Revision 1.8  2008/05/13 15:02:15  jdreed
% Degree month is June, not May.  Added note about prevdegrees.
% Arthur Smith's title updated
%
% Revision 1.7  2001/02/08 18:53:16  boojum
% changed some \newpages to \cleardoublepages
%
% Revision 1.6  1999/10/21 14:49:31  boojum
% changed comment referring to documentstyle
%
% Revision 1.5  1999/10/21 14:39:04  boojum
% *** empty log message ***
%
% Revision 1.4  1997/04/18  17:54:10  othomas
% added page numbers on abstract and cover, and made 1 abstract
% page the default rather than 2.  (anne hunter tells me this
% is the new institute standard.)
%
% Revision 1.4  1997/04/18  17:54:10  othomas
% added page numbers on abstract and cover, and made 1 abstract
% page the default rather than 2.  (anne hunter tells me this
% is the new institute standard.)
%
% Revision 1.3  93/05/17  17:06:29  starflt
% Added acknowledgements section (suggested by tompalka)
% 
% Revision 1.2  92/04/22  13:13:13  epeisach
% Fixes for 1991 course 6 requirements
% Phrase "and to grant others the right to do so" has been added to 
% permission clause
% Second copy of abstract is not counted as separate pages so numbering works
% out
% 
% Revision 1.1  92/04/22  13:08:20  epeisach

% NOTE:
% These templates make an effort to conform to the MIT Thesis specifications,
% however the specifications can change.  We recommend that you verify the
% layout of your title page with your thesis advisor and/or the MIT 
% Libraries before printing your final copy.

\title{WepSIM: Simulador de procesador elemental con unidad de control microprogramada}

\author{Javier Prieto Cepeda}
% If you wish to list your previous degrees on the cover page, use the 
% previous degrees command:
%       \prevdegrees{A.A., Harvard University (1985)}
% You can use the \\ command to list multiple previous degrees
%       \prevdegrees{B.S., University of California (1978) \\
%                    S.M., Massachusetts Institute of Technology (1981)}
\department{Department of Electrical Engineering and Computer Science}

% If the thesis is for two degrees simultaneously, list them both
% separated by \and like this:
% \degree{Doctor of Philosophy \and Master of Science}
\degree{Bachelor of Science in Computer Science and Engineering}

% As of the 2007-08 academic year, valid degree months are September, 
% February, or June.  The default is June.
\degreemonth{June}
\degreeyear{1990}
\thesisdate{May 18, 1990}

%% By default, the thesis will be copyrighted to MIT.  If you need to copyright
%% the thesis to yourself, just specify the `vi' documentclass option.  If for
%% some reason you want to exactly specify the copyright notice text, you can
%% use the \copyrightnoticetext command.  
%\copyrightnoticetext{\copyright IBM, 1990.  Do not open till Xmas.}

% If there is more than one supervisor, use the \supervisor command
% once for each.
\supervisor{Félix García Carballeira}{Full Professor}

% This is the department committee chairman, not the thesis committee
% chairman.  You should replace this with your Department's Committee
% Chairman.
\chairman{Arthur C. Smith}{Chairman, Department Committee on Graduate Theses}

% Make the titlepage based on the above information.  If you need
% something special and can't use the standard form, you can specify
% the exact text of the titlepage yourself.  Put it in a titlepage
% environment and leave blank lines where you want vertical space.
% The spaces will be adjusted to fill the entire page.  The dotted
% lines for the signatures are made with the \signature command.
%\maketitle

% The abstractpage environment sets up everything on the page except
% the text itself.  The title and other header material are put at the
% top of the page, and the supervisors are listed at the bottom.  A
% new page is begun both before and after.  Of course, an abstract may
% be more than one page itself.  If you need more control over the
% format of the page, you can use the abstract environment, which puts
% the word "Abstract" at the beginning and single spaces its text.

%% You can either \input (*not* \include) your abstract file, or you can put
%% the text of the abstract directly between the \begin{abstractpage} and
%% \end{abstractpage} commands.

% First copy: start a new page, and save the page number.
\afterpage{\blankpage} % blank page
\clearpage

\thispagestyle{empty}
\vspace*{\fill} 
\begin{quote}
\epigraph{\large \textit{If you would not be forgotten as soon as you are dead and rotten, either write things worth reading, or do things worth the writing.}}{\large \flushright \textbf{Benjamin Franklin}}
\end{quote}
\vspace*{\fill} 

\afterpage{\blankpage} % blank page

\chapter*{Agradecimientos}
\addcontentsline{toc}{chapter}{\textit{Agradecimientos}}%
Es muy difícil incluir en tan pocas palabras a todas aquellas personas que han puesto su granito de arena a lo largo de estos años para que a día de hoy, haya logrado llegar hasta aquí. Es por ello que si me dejo a alguien sin incluir, me disculpe y no dude en sentirse parte de estas palabras.

En primer lugar, quiero agradecer de corazón a mis padres Tomás y Raquel el haberme educado y enseñado a elegir aquel camino que me hiciera feliz de verdad, dándome la oportunidad de estudiar con su esfuerzo y sacrificio. Aquí también estás tu David, pese a llegar unos años después de mi, no paras de enseñarme día a día nuevas lecciones, de los pequeños también se aprende. Gracias.

También debo de dar las gracias a la persona que ha tenido que aguantar estos 5 años las consecuencias de estudiar esta carrera. Míriam, gracias por tu paciencia, tus consejos y tus ánimos. 

Mis abuelos también han sido parte de esas personas que han sumado su granito de arena enseñándome y aconsejándome en todo momento, ejerciendo de guías. Sé que vosotros habéis sufrido también mucho en estos años, pero aquí está el premio.

Por otro lado, debo de agradecer a dos personas el darme la oportunidad de realizar este proyecto. Félix y Álex, gracias por este tiempo en el que tanto he podido aprender y en el que tanto me habéis ayudado.

También son parte de este proyecto los grandes amigos que he hecho en la travesía por el grupo de investigación ARCOS. Uno de ellos además, después de compartir asiento en clase. Saúl, gracias por tu ayuda y por los buenísimos momentos que hemos compartido. Carlos, nos quedan más carreras por hacer juntos. Fran, Estefanía, Silvina, Cascajo, Jesús Cristina, Garci, Pablo, David, Rafa y Alfredo, gracias por acogerme tan bien en el laboratorio y por ayudarme en todo momento.

Agradecer también a Jesús, Javi Blas y al resto de personas de componen el grupo de investigación ARCOS la ayuda prestada durante este tiempo.

Por otro lado, también quiero destacar a aquellos compañeros de prácticas y amigos que he hecho a lo largo de estos años en la carrera, y que siempre recordaré. Sergio, Planet, Álex, Guille, Juanlu, Sandra y Marin.

Por último, me gustaría dar las gracias también a todas aquellas personas que forman el sistema de educación pública en España, ya que sin su esfuerzo y dedicación, no podríamos llegar hasta aquí.


\thispagestyle{empty}

%%%%%%%%%%%%%%%%%%%%%%%%%%%%%%%%%%%%%%%%%%%%%%%%%%%%%%%%%%%%%%%%%%%%%%
% -*-latex-*-


% Uncomment the next line if you do NOT want a page number on your
% abstract and acknowledgments pages.
% \pagestyle{empty}
%\setcounter{savepage}{\thepage}
\begin{abstractpage}
\addcontentsline{toc}{chapter}{Abstract}%
% $Log: abstract.tex,v $
% Revision 1.1  93/05/14  14:56:25  starflt
% Initial revision
% 
% Revision 1.1  90/05/04  10:41:01  lwvanels
% Initial revision
% 
%
%% The text of your abstract and nothing else (other than comments) goes here.
%% It will be single-spaced and the rest of the text that is supposed to go on
%% the abstract page will be generated by the abstractpage environment.  This
%% file should be \input (not \include 'd) from cover.tex.
\thispagestyle{plain}

WepSIM es un simulador de un procesador elemental con unidad de control microprogramada basado en el procesador diseñado por el personal del grupo de investigación ARCOS del Departamento de Informática de la Universidad Carlos III de Madrid para la docencia en la asignatura Estructura de Computadores. 
   Con esta herramienta se ofrece una visión integrada de la microprogramación de un computador y la programación en lenguaje ensamblador, dando la posibilidad de especificar distintos juegos de instrucciones y ofreciendo un nivel de detalle a nivel de ciclo de reloj. 
   Gracias al diseño modular propuesto en el cual está basado este simulador, es posible añadir, modificar o quitar elementos existentes al modelo hardware diseñado. 
   Debido a que solo precisa de un navegador web, es posible utilizarlo en casi cualquier momento y dispositivo.
   Todas estas características, hacen de WepSIM el primer simulador docente que unifica la microprogramación y la programación en lenguaje ensamblador permitiendo una fácil personalización del modelo hardware a simular y el uso de diferentes juegos de instrucciones. Mediante este simulador se busca facilitar la enseñanza y aprendizaje en el área docente de Estructura y Arquitectura de Computadores.
\vspace{0.7cm}

\textbf{Palabras clave:} \acrshort{mips} $\cdot$ Simulación $\cdot$ Ensamblador $\cdot$ Microprogramación

\end{abstractpage}

% Additional copy: start a new page, and reset the page number.  This way,
% the second copy of the abstract is not counted as separate pages.
% Uncomment the next 6 lines if you need two copies of the abstract
% page.
% \setcounter{page}{\thesavepage}
% \begin{abstractpage}
% % $Log: abstract.tex,v $
% Revision 1.1  93/05/14  14:56:25  starflt
% Initial revision
% 
% Revision 1.1  90/05/04  10:41:01  lwvanels
% Initial revision
% 
%
%% The text of your abstract and nothing else (other than comments) goes here.
%% It will be single-spaced and the rest of the text that is supposed to go on
%% the abstract page will be generated by the abstractpage environment.  This
%% file should be \input (not \include 'd) from cover.tex.
\thispagestyle{plain}

WepSIM es un simulador de un procesador elemental con unidad de control microprogramada basado en el procesador diseñado por el personal del grupo de investigación ARCOS del Departamento de Informática de la Universidad Carlos III de Madrid para la docencia en la asignatura Estructura de Computadores. 
   Con esta herramienta se ofrece una visión integrada de la microprogramación de un computador y la programación en lenguaje ensamblador, dando la posibilidad de especificar distintos juegos de instrucciones y ofreciendo un nivel de detalle a nivel de ciclo de reloj. 
   Gracias al diseño modular propuesto en el cual está basado este simulador, es posible añadir, modificar o quitar elementos existentes al modelo hardware diseñado. 
   Debido a que solo precisa de un navegador web, es posible utilizarlo en casi cualquier momento y dispositivo.
   Todas estas características, hacen de WepSIM el primer simulador docente que unifica la microprogramación y la programación en lenguaje ensamblador permitiendo una fácil personalización del modelo hardware a simular y el uso de diferentes juegos de instrucciones. Mediante este simulador se busca facilitar la enseñanza y aprendizaje en el área docente de Estructura y Arquitectura de Computadores.
\vspace{0.7cm}

\textbf{Palabras clave:} \acrshort{mips} $\cdot$ Simulación $\cdot$ Ensamblador $\cdot$ Microprogramación

% \end{abstractpage}

\afterpage{\blankpage} % blank page

% Some departments (e.g. 5) require an additional signature page.  See
% signature.tex for more information and uncomment the following line if
% applicable.
% \include{signature}
\pagestyle{plain}
 % -*- Mode:TeX -*-
%% This file simply contains the commands that actually generate the table of
%% contents and lists of figures and tables.  You can omit any or all of
%% these files by simply taking out the appropriate command.  For more
%% information on these files, see appendix C.3.3 of the LaTeX manual. 
\lhead[\thepage]{CONTENIDOS}
\chead[]{}
\rhead[WepSIM: Simulador de un procesador elemental con unidad de control microprogramada]{\thepage}
\renewcommand{\headrulewidth}{0.5pt}
\lfoot[]{}
\cfoot[]{}
\rfoot[]{}
\renewcommand{\footrulewidth}{0pt}
\tableofcontents
\addcontentsline{toc}{chapter}{Contenidos}
\markboth{}{CONTENTS}

\clearpage
\afterpage{\blankpage} % blank page
%\blankpage

\lhead[\thepage]{ÍNDICE DE FIGURAS}
\chead[]{}
\rhead[WepSIM: Simulador de un procesador elemental con unidad de control microprogramada\leftmark]{\thepage}
\renewcommand{\headrulewidth}{0.5pt}
\lfoot[]{}
\cfoot[]{}
\rfoot[]{}
\renewcommand{\footrulewidth}{0pt}
\listoffigures
\addcontentsline{toc}{chapter}{\listfigurename}
\markboth{}{LIST OF FIGURES}

\lhead[\thepage]{ÍNDICE DE TABLAS}
\chead[]{}
\rhead[WepSIM: Simulador de un procesador elemental con unidad de control microprogramada\leftmark]{\thepage}
\renewcommand{\headrulewidth}{0.5pt}
\lfoot[]{}
\cfoot[]{}
\rfoot[]{}
\renewcommand{\footrulewidth}{0pt}
\listoftables
\addcontentsline{toc}{chapter}{\listtablename}
\markboth{}{LIST OF TABLES}

\afterpage{\blankpage \blankpage} % blank page


\pagenumbering{arabic}

\lhead[\thepage]{CHAPTER \thechapter. INTRODUCTION}
\chead[]{}
\rhead[A Complete Simulator for Volunteer Computing Environments\leftmark]{\thepage}
\renewcommand{\headrulewidth}{0.5pt}

\lfoot[]{}
\cfoot[]{}
\rfoot[]{}
\renewcommand{\footrulewidth}{0pt}

%% This is an example first chapter.  You should put chapter/appendix that you
%% write into a separate file, and add a line \include{yourfilename} to
%% main.tex, where `yourfilename.tex' is the name of the chapter/appendix file.
%% You can process specific files by typing their names in at the 
%% \files=
%% prompt when you run the file main.tex through LaTeX.
\chapter{Introducción}
\label{ch:introduction}
\markboth{}{INTRODUCTION}

El primer capítulo introduce brevemente el objetivo del proyecto, incluyendo las características clave del proyecto y su motivación (Section \ref{sec:background_and_motivation}, \textit{\nameref{sec:background_and_motivation}}), los objetivos del proyecto (Section \ref{sec:objectives}, \textit{\nameref{sec:objectives}}), y toda la estructura del documento(Section \ref{sec:document_structure}, \textit{\nameref{sec:document_structure}}).

\section{Motivación}
\label{sec:background_and_motivation}
Explicamos la motivación de realizar un simulador interactivo.



\section{Objetivos}
\label{sec:objectives}

El objetivo principal de este proyecto, es desarrollar un simulador, que a diferencia de los existentes, pueda simular de forma completa el comportamiento de un procesador elemental permitiendo comprobar el estado de los componentes en cada ciclo de reloj, de manera que ayude a los alumnos a comprender y asimilar de forma sencilla y visual el funcionamiento de un procesador. Los objetivos secundarios son:

\begin{itemize}

\item Diseñar la especificación del juego de instrucciones que permita la creación de un lenguaje ensamblador adaptado a la arquitectura del simulador.

\item Diseñar e implementar el compilador del juego de instrucciones para la generación del firmware del simulador.

\item Diseñar e implementar el compilador genérico de ensamblador que permita la generación del binario correspondiente al juego de instrucciones diseñado.

\item Diseñar e implementar el motor del simulador permitiendo ejecuciones reales del código ensamblador correspondiente al juego de instrucciones compilado.

\item Diseñar una interfaz que proporcione en todo momento la información necesaria en relación al estado de la ejecución del código, de forma sencilla y visual.

\item Permitir que los usuarios puedan importar/exportar tanto la especificación del juego de instrucciones como el código ensamblador.

\item Crear un mecanismo de "modificación en caliente" que permita en mitad de una ejecución modificar el juego de instrucciones, de forma que se puedan realizar pruebas sin necesidad de reiniciar la ejecución.

\end{itemize}

\section{Estructura del documento}
\label{sec:document_structure}

El documento contiene los siguientes capítulos:

\begin{itemize}

\item Capítulo \ref{ch:introduction}, \textit{\nameref{ch:introduction}}, presenta una breve descripción del contenido del documento. También incluye la motivación y los objetivos del proyecto.

\item Capítulo \ref{ch:state_of_the_art}, \textit{\nameref{ch:state_of_the_art}}, incluye una descripción de los diferentes tipos de procesadores y compiladores actuales y presenta el trabajo relacionado.

\item Capítulo \ref{ch:analysis}, \textit{\nameref{ch:analysis}}, describe brevemente el proyecto, explica la solución elegida, establece los requisitos y presenta el marco regulador del proyecto.

\item Capítulo \ref{ch:design}, \textit{\nameref{ch:design}}, detalla el diseño del sistema, incluyendo todos sus componentes.

\item Capítulo \ref{ch:implementation_and_deployment}, \textit{\nameref{ch:implementation_and_deployment}}, incluye los detalles de implementación de las partes principales del software desarrollado y las características necesarias para la implementación de la aplicación.

\item Capítulo \ref{ch:verification_validation_and_evaluation}, \textit{\nameref{ch:verification_validation_and_evaluation}}, detalla una verificación y validación completa del proyecto. También muestra una evaluación de diferentes casos de prueba utilizando el simulador.

\item Capítulo \ref{ch:planning_and_budget}, \textit{\nameref{ch:planning_and_budget}}, presenta los conceptos relacionados con la planificación seguida, descompone todos los costes del proyecto y describe el entorno socio-económico.

\item Capítulo \ref{ch:conclusions_and_future_work}, \textit{\nameref{ch:conclusions_and_future_work}}, incluye las contribuciones del proyecto, explica las principales conclusiones del proyecto y presenta los trabajos futuros.

\item Appendix \ref{ch:user_manual}, \textit{\nameref{ch:user_manual}}, incluye un manual de usuario completo para la aplicación. Contiene un tutorial que guía al usuario desde la creación de un nuevo juego de instrucciones hasta una ejecución completa paso a paso, y una serie de ejemplos educativos para aprender el funcionamiento de un procesador mediante el uso de simulaciones utilizando el software desarrollado. 

\end{itemize}


\lhead[\thepage]{CHAPTER \thechapter. STATE OF THE ART}
\chead[]{}
\rhead[WepSIM: Simulador de procesador elemental con unidad de control microprogramada\leftmark]{\thepage}
\renewcommand{\headrulewidth}{0.5pt}

\lfoot[]{}
\cfoot[]{}
\rfoot[]{}
\renewcommand{\footrulewidth}{0pt}

%% This is an example first chapter.  You should put chapter/appendix that you
%% write into a separate file, and add a line \include{yourfilename} to
%% main.tex, where `yourfilename.tex' is the name of the chapter/appendix file.
%% You can process specific files by typing their names in at the 
%% \files=
%% prompt when you run the file main.tex through LaTeX.
\chapter{Estado del arte}
\label{ch:state_of_the_art}
\markboth{}{STATE OF THE ART}

Este capítulo presenta el estado del arte, la última y más avanzada etapa de las tecnologías relacionadas con nuestra aplicación. Primero, se presentan los diferentes simuladores existentes para microprogramación (Section \ref{sec:simuladores_microprogramacion}). Después, se presentan los diferentes simuladores existentes para la programación en código ensamblador (Section \ref{sec:simuladores_ensamblador}). Por último, realizamos una comparación de nuestro trabajo con el contexto actual de los distintos simuladores expuestos previamente (Section \ref{sec:propuesta_simulacion}).

\section{Simuladores para microprogramación}
\label{sec:simuladores_microprogramacion}


\section{Simuladores para programación en ensamblador}
\label{sec:simuladores_ensamblador}


\section{Propuesta de simulación unificada}
\label{sec:propuesta_simulacion}


\lhead[\thepage]{CHAPTER \thechapter. ANALYSIS}
\chead[]{}
\rhead[A Complete Simulator for Volunteer Computing Environments\leftmark]{\thepage}
\renewcommand{\headrulewidth}{0.5pt}

\lfoot[]{}
\cfoot[]{}
\rfoot[]{}
\renewcommand{\footrulewidth}{0pt}

%% This is an example first chapter.  You should put chapter/appendix that you
%% write into a separate file, and add a line \include{yourfilename} to
%% main.tex, where `yourfilename.tex' is the name of the chapter/appendix file.
%% You can process specific files by typing their names in at the 
%% \files=
%% prompt when you run the file main.tex through LaTeX.
\chapter{Análisis}
\label{ch:analysis}
\markboth{}{ANALYSIS}

El objetivo principal de este capitulo, es describir el proyecto mediante la obtención y especificación de los requisitos del simulador, que puede proporcionar información suficiente para un análisis detallado que, por lo tanto, puede servir para continuar diseñando e implementando (Capítulos \ref{ch:design}, \textit{\nameref{ch:design}}; and \ref{ch:implementation_and_deployment}, \textit{\nameref{ch:implementation_and_deployment}}) un software que cumpla con esos requisitos. 

Con el fin de obtener los requisitos del sistema, el tutor ha desempeñado el papel del cliente en diferentes reuniones, mientras que el alumno ha desempeñado los roles de analista, diseñador, programador y probador.

Section \ref{sec:project_description} briefly summarizes the project description. Section \ref{sec:solution_selection} discusses the chosen solution and compares it to the alternatives considered. Section \ref{sec:requirements} specifies the system requirements, starting with the user requirements, and ending with the functional and non-functional requirements. Finally, Section \ref{sec:regulatory_framework} indicates the set of laws and regulations for the management of the software.

\section{Descripción del proyecto}
\label{sec:project_description}

Explicamos lo que se pretende obtener del simulador.

\section{Solución elegida}
\label{sec:solution_selection}

Explicamos el porque hemos elegido esta solución, y nos comparamos.

\begin{table}[htbp]
\ra{1.2}
\centering
%\resizebox{\textwidth}{%
\resizebox{\textwidth}{!}{
\begin{tabular}{@{}llllll@{}}
\toprule
Features & SimGrid & PVMsim & Virtual-GEMS & MDCSim & SPECI-2 \\ 
\midrule
Languages				& C/C++/Java/Ruby                            & C                   & C/C++/Ruby           & C++/Java             & Java\\
Open Source				& \ding{51}                            & \ding{51}                   & \ding{51}           & \ding{55}             & \ding{51}\\
\midrule
Models & & & & &\\
\midrule
Communication		& \ding{51}               & \ding{55}           & \ding{55}  & \ding{51} & \ding{51}                  \\
Energy				& \ding{51}               & \ding{55}           & \ding{55}  & \ding{51} & \ding{55}                  \\
Hardware				& \ding{51}               & \ding{51}           & \ding{51}  & \ding{51} & \ding{51}                  \\
\Gls{scheduling}			& \ding{51}               & \ding{51}           & \ding{55}  & \ding{55} & \ding{51}                  \\
Users				& \ding{51}               & \ding{55}           & \ding{55}  & \ding{51} & \ding{55}                  \\
\bottomrule
\end{tabular}
}
\caption{Comparison of simulation \gls{framework}s for distributed computing systems.}
\label{tab:comparison_frameworks}
\end{table}


\section{Requisitos}
\label{sec:requirements}

This section provides a detailed description of the application requirements. For the requirement specification task, the IEEE recommended practices \cite{ieee1998} were followed. According to these practices, a good specification must address the software functionality, performance issues, the external interfaces, other non-functional features and design or implementation constraints. Moreover, the requirements specification must be:

\begin{itemize}

\item \textbf{Complete:} the document reflects all significant software requirements.

\item \textbf{Consistent:} requirements must not generate conflicts with each other.

\item \textbf{Correct:} every requirement is one that the software shall meet according to the user needs.

\item \textbf{Modifiable:} the structure of the specification allows changes to the requirements in a simple, complete and consistent way.

\item \textbf{Ranked based on importance and stability:} every requirement must indicate its importance and its stability.

\item \textbf{Traceable:} the origin of every requirement is clear and it can be easily referenced in further stages.

\item \textbf{Unambiguous:} every requirement has a single interpretation.

\item \textbf{Verifiable:} every requirement must be verifiable, that is, there exists some process to verify that the software complies with every single requirement.


\end{itemize}

Starting from the user requirements, which constitute an informal reference to the product performance that the client expects, we derived the software requirements (in this case, functional requirements and non-functional requirements) that guided the design process with specific information on the functionality of the system and other characteristics. The retrieved requirements were structured according with the following schema:

\begin{itemize}
\item[1.] \textbf{User Requirements} 
	\begin{itemize}
		\item[(a)] \textbf{Capacity:} the requirement describes the expected system functionality as in use cases.
		\item[(b)] \textbf{Restriction:} the requirement specifies constraints or conditions the system must fulfil.	
	\end{itemize}	
\end{itemize}

\begin{itemize}
\item[2.] \textbf{Software Requirements}
	\begin{itemize}
	\item[(a)] 	\textbf{Functional}
		\begin{itemize}
		\item[i.] 	\textbf{Functional:} the requirement describes the basic system functionality and purpose while minimizing ambiguity.
		\item[ii.] 	\textbf{Inverse:} the requirement limits the functionality of the application to clarify its scope.
		\end{itemize}

	\item[(b)] 	\textbf{Non-Functional}
		\begin{itemize}
		\item[i.] 	\textbf{Performance:} the requirement is related to the minimum required performance of the resulting system.
		\item[ii.] 	\textbf{Interface:} the requirement is related to the user interface of the application.
		\item[iii.] 	\textbf{Scalability:} the requirement is related to the ability of the system to adapt to increasing workloads.
		\item[iv.] 	\textbf{Platform:} the requirement specifies the underlying software and hardware platforms in which the system will operate.
		\end{itemize}			
	\end{itemize}
\end{itemize}

Table \ref{tab:requirements_template} provides the template used for requirements specification. Note that for user requirements, the ID format will be UR-XYY, where X indicates the requirement subtype: capacity requirements (C), or restrictions (R). YY corresponds to the requirement number under its subcategory. For software requirements, the ID format SR-X-YZZ will be used, where X indicates if it is a functional (F) or non-functional (NF) requirement, and Y represents its subcategory: functional (F), inverse (I), performance (P), interface (UI), scalability (S), or platform (PL). ZZ corresponds to the requirement number under its subcategory.

\begin{center}
\begin{table*}[htbp]
\centering
\begin{tabular}{@{}p{2.5cm} p{9cm}@{}} 
\toprule
\textbf{ID} 				& Requirement ID. \\
\midrule
\textbf{Name} 			& Requirement name. \\
\midrule
\textbf{Type} 			& Indicates the category in which the requirement would be placed according to the previously described schema. \\
\midrule
\textbf{Origin} 			& Constitutes the requirement source. It might be the user, another requirement or other stakeholders involved in the project. \\
\midrule
\textbf{Priority}		& Indicates the requirement priority according to its importance. A requirement can be identified either as \textit{essential}, \textit{conditional} or \textit{optional}. \\
\midrule
\textbf{Stability} 		& Indicates the requirement variability through the development process, defined as \textit{stable} or \textit{unstable}. \\
\midrule
\textbf{Description} 	& Detailed explanation of the requirement. \\
\bottomrule
\end{tabular}
\caption{Template for requirements specification.}
\label{tab:requirements_template}
\end{table*}
\end{center}

\subsection{Requisitos de Usuario}

This subsection specifies the user requirements.

\begin{center}
\begin{table*}[htbp]
\centering
\begin{tabular}{@{}p{2.5cm} p{9cm}@{}} 
\toprule
\textbf{ID} 				& UR-C01\\
\midrule
\textbf{Name} 			& BOINC projects simulation \\
\midrule
\textbf{Type} 			& Capacity \\
\midrule
\textbf{Origin} 			& User \\
\midrule
\textbf{Priority}		& Essential \\
\midrule
\textbf{Stability} 		& Stable \\
\midrule
\textbf{Description} 	& The application shall simulate real BOINC projects. \\
\bottomrule
\end{tabular}
\caption{User requirement UR-C01.}
\label{tab:urc01}
\end{table*}
\end{center}

\begin{center}
\begin{table*}[htbp]
\centering
\begin{tabular}{@{}p{2.5cm} p{9cm}@{}} 
\toprule
\textbf{ID} 				& UR-C02\\
\midrule
\textbf{Name} 			& Client \gls{scheduling} \\
\midrule
\textbf{Type} 			& Capacity \\
\midrule
\textbf{Origin} 			& User \\
\midrule
\textbf{Priority}		& Essential \\
\midrule
\textbf{Stability} 		& Stable \\
\midrule
\textbf{Description} 	& The client scheduler of the simulator shall follow the actual BOINC client \gls{scheduling}. \\
\bottomrule
\end{tabular}
\caption{User requirement UR-C02.}
\label{tab:urc02}
\end{table*}
\end{center}

\begin{center}
\begin{table*}[htbp]
\centering
\begin{tabular}{@{}p{2.5cm} p{9cm}@{}} 
\toprule
\textbf{ID} 				& UR-C03\\
\midrule
\textbf{Name} 			& Simulation components \\
\midrule
\textbf{Type} 			& Capacity \\
\midrule
\textbf{Origin} 			& User \\
\midrule
\textbf{Priority}		& Essential \\
\midrule
\textbf{Stability} 		& Stable \\
\midrule
\textbf{Description} 	& The simulations shall cover all the elements present in the BOINC infrastructure. \\
\bottomrule
\end{tabular}
\caption{User requirement UR-C03.}
\label{tab:urc03}
\end{table*}
\end{center}

\begin{center}
\begin{table*}[htbp]
\centering
\begin{tabular}{@{}p{2.5cm} p{9cm}@{}} 
\toprule
\textbf{ID} 				& UR-R01\\
\midrule
\textbf{Name} 			& Linux as underlying OS \\
\midrule
\textbf{Type} 			& Restriction \\
\midrule
\textbf{Origin} 			& User \\
\midrule
\textbf{Priority}		& Essential \\
\midrule
\textbf{Stability} 		& Stable \\
\midrule
\textbf{Description} 	& The simulator shall be designed for Linux operating systems. \\
\bottomrule
\end{tabular}
\caption{User requirement UR-R01.}
\label{tab:urr01}
\end{table*}
\end{center}

\begin{center}
\begin{table*}[htbp]
\centering
\begin{tabular}{@{}p{2.5cm} p{9cm}@{}} 
\toprule
\textbf{ID} 				& UR-R02\\
\midrule
\textbf{Name} 			& SimGrid toolkit \\
\midrule
\textbf{Type} 			& Restriction \\
\midrule
\textbf{Origin} 			& User \\
\midrule
\textbf{Priority}		& Essential \\
\midrule
\textbf{Stability} 		& Stable \\
\midrule
\textbf{Description} 	& The application shall use the SimGrid toolkit in order to implement the distributed computing functionalities. \\
\bottomrule
\end{tabular}
\caption{User requirement UR-R02.}
\label{tab:urr02}
\end{table*}
\end{center}

\begin{center}
\begin{table*}[htbp]
\centering
\begin{tabular}{@{}p{2.5cm} p{9cm}@{}} 
\toprule
\textbf{ID} 				& UR-R03\\
\midrule
\textbf{Name} 			& Scalability \\
\midrule
\textbf{Type} 			& Restriction \\
\midrule
\textbf{Origin} 			& User \\
\midrule
\textbf{Priority}		& Essential \\
\midrule
\textbf{Stability} 		& Stable \\
\midrule
\textbf{Description} 	& The simulator shall be scalable (carry out executions by simulating a large number of client hosts). \\
\bottomrule
\end{tabular}
\caption{User requirement UR-R03.}
\label{tab:urr03}
\end{table*}
\end{center}

\clearpage
\subsection{Requisitos Funcionales}

This subsection specifies the functional requirements.

\begin{center}
\begin{table*}[htbp]
\centering
\begin{tabular}{@{}p{2.5cm} p{9cm}@{}} 
\toprule
\textbf{ID} 				& SR-F-F01\\
\midrule
\textbf{Name} 			& Credit calculation \\
\midrule
\textbf{Type} 			& Functional \\
\midrule
\textbf{Origin} 			& UR-C01 \\
\midrule
\textbf{Priority}		& Essential \\
\midrule
\textbf{Stability} 		& Stable \\
\midrule
\textbf{Description} 	& The simulator shall calculate the number of credits granted to each volunteer client analogously to actual BOINC projects. \\
\bottomrule
\end{tabular}
\caption{Functional requirement SR-F-F01.}
\label{tab:srff01}
\end{table*}
\end{center}

\begin{center}
\begin{table*}[htbp]
\centering
\begin{tabular}{@{}p{2.5cm} p{9cm}@{}} 
\toprule
\textbf{ID} 				& SR-F-F02\\
\midrule
\textbf{Name} 			& Collection of statistics \\
\midrule
\textbf{Type} 			& Functional \\
\midrule
\textbf{Origin} 			& UR-C01 \\
\midrule
\textbf{Priority}		& Essential \\
\midrule
\textbf{Stability} 		& Stable \\
\midrule
\textbf{Description} 	& The simulator shall collect, for each project, the same statistics that actual BOINC projects (published in BOINCstats \cite{BOINC2016}). \\
\bottomrule
\end{tabular}
\caption{Functional requirement SR-F-F02.}
\label{tab:srff02}
\end{table*}
\end{center}

\begin{center}
\begin{table*}[htbp]
\centering
\begin{tabular}{@{}p{2.5cm} p{9cm}@{}} 
\toprule
\textbf{ID} 				& SR-F-F03\\
\midrule
\textbf{Name} 			& Almost identical outputs \\
\midrule
\textbf{Type} 			& Functional \\
\midrule
\textbf{Origin} 			& UR-C01 \\
\midrule
\textbf{Priority}		& Essential \\
\midrule
\textbf{Stability} 		& Stable \\
\midrule
\textbf{Description} 	& The outputs of the simulator for existing projects should be almost identical to those published in BOINCstats \cite{BOINC2016}. \\
\bottomrule
\end{tabular}
\caption{Functional requirement SR-F-F03.}
\label{tab:srff03}
\end{table*}
\end{center}

\begin{center}
\begin{table*}[htbp]
\centering
\begin{tabular}{@{}p{2.5cm} p{9cm}@{}} 
\toprule
\textbf{ID} 				& SR-F-F04\\
\midrule
\textbf{Name} 			& Multiple BOINC projects \\
\midrule
\textbf{Type} 			& Functional \\
\midrule
\textbf{Origin} 			& UR-C01 \\
\midrule
\textbf{Priority}		& Essential \\
\midrule
\textbf{Stability} 		& Stable \\
\midrule
\textbf{Description} 	& The simulator shall allow the simulation of different projects simultaneously. \\
\bottomrule
\end{tabular}
\caption{Functional requirement SR-F-F04.}
\label{tab:srff04}
\end{table*}
\end{center}

\begin{center}
\begin{table*}[htbp]
\centering
\begin{tabular}{@{}p{2.5cm} p{9cm}@{}} 
\toprule
\textbf{ID} 				& SR-F-F05\\
\midrule
\textbf{Name} 			& Client scheduler \\
\midrule
\textbf{Type} 			& Functional \\
\midrule
\textbf{Origin} 			& UR-C02 \\
\midrule
\textbf{Priority}		& Essential \\
\midrule
\textbf{Stability} 		& Stable \\
\midrule
\textbf{Description} 	& The client scheduler shall follow the actual BOINC client \gls{scheduling} (described in \cite{anderson2007}). \\
\bottomrule
\end{tabular}
\caption{Functional requirement SR-F-F05.}
\label{tab:srff05}
\end{table*}
\end{center}

\begin{center}
\begin{table*}[htbp]
\centering
\begin{tabular}{@{}p{2.5cm} p{9cm}@{}} 
\toprule
\textbf{ID} 				& SR-F-F06\\
\midrule
\textbf{Name} 			& Realistic simulation elements \\
\midrule
\textbf{Type} 			& Functional \\
\midrule
\textbf{Origin} 			& UR-C03 \\
\midrule
\textbf{Priority}		& Essential \\
\midrule
\textbf{Stability} 		& Stable \\
\midrule
\textbf{Description} 	& All simulations shall include the following elements: tasks, volunteer hosts, servers, data servers, networks, and hosts availability. \\
\bottomrule
\end{tabular}
\caption{Functional requirement SR-F-F06.}
\label{tab:srff06}
\end{table*}
\end{center}


\subsection{Requisitos No-Funcionales}

This subsection specifies the non-functional requirements.

\begin{center}
\begin{table*}[htbp]
\centering
\begin{tabular}{@{}p{2.5cm} p{9cm}@{}} 
\toprule
\textbf{ID} 				& SR-NF-PL01\\
\midrule
\textbf{Name} 			& Ubuntu 14.04 \\
\midrule
\textbf{Type} 			& Platform \\
\midrule
\textbf{Origin} 			& UR-R01 \\
\midrule
\textbf{Priority}		& Essential \\
\midrule
\textbf{Stability} 		& Stable \\
\midrule
\textbf{Description} 	& The simulator shall work on the Ubuntu Linux distribution, version 14.04. \\
\bottomrule
\end{tabular}
\caption{Non-functional requirement SR-NF-PL01.}
\label{tab:srnfpl01}
\end{table*}
\end{center}

\begin{center}
\begin{table*}[htbp]
\centering
\begin{tabular}{@{}p{2.5cm} p{9cm}@{}} 
\toprule
\textbf{ID} 				& SR-NF-PL02\\
\midrule
\textbf{Name} 			& SimGrid MSG API \\
\midrule
\textbf{Type} 			& Platform \\
\midrule
\textbf{Origin} 			& UR-R02 \\
\midrule
\textbf{Priority}		& Essential \\
\midrule
\textbf{Stability} 		& Stable \\
\midrule
\textbf{Description} 	& The implementation, setup and control of the simulations shall be carried out using the MSG API of the SimGrid toolkit. \\
\bottomrule
\end{tabular}
\caption{Non-functional requirement SR-NF-PL02.}
\label{tab:srnfpl02}
\end{table*}
\end{center}

\begin{center}
\begin{table*}[htbp]
\centering
\begin{tabular}{@{}p{2.5cm} p{9cm}@{}} 
\toprule
\textbf{ID} 				& SR-NF-PL03\\
\midrule
\textbf{Name} 			& C programming language \\
\midrule
\textbf{Type} 			& Platform\\
\midrule
\textbf{Origin} 			& UR-R03 \\
\midrule
\textbf{Priority}		& Essential \\
\midrule
\textbf{Stability} 		& Stable \\
\midrule
\textbf{Description} 	& The simulator shall be written in the C programming language. \\
\bottomrule
\end{tabular}
\caption{Non-functional requirement SR-NF-PL03.}
\label{tab:srnfpl03}
\end{table*}
\end{center}

\begin{center}
\begin{table*}[htbp]
\centering
\begin{tabular}{@{}p{2.5cm} p{9cm}@{}} 
\toprule
\textbf{ID} 				& SR-NF-S01\\
\midrule
\textbf{Name} 			& Large simulations \\
\midrule
\textbf{Type} 			& Scalability \\
\midrule
\textbf{Origin} 			& UR-R03 \\
\midrule
\textbf{Priority}		& Essential \\
\midrule
\textbf{Stability} 		& Stable \\
\midrule
\textbf{Description} 	& The application must be able to perform simulations with more than 100,000 hosts in a machine with at least 8 GB of \gls{ram}. \\
\bottomrule
\end{tabular}
\caption{Non-functional requirement SR-NF-S01.}
\label{tab:srnfs01}
\end{table*}
\end{center}

\begin{center}
\begin{table*}[htbp]
\centering
\begin{tabular}{@{}p{2.5cm} p{9cm}@{}} 
\toprule
\textbf{ID} 				& SR-NF-P01\\
\midrule
\textbf{Name} 			& Linear-time execution \\
\midrule
\textbf{Type} 			& Performance \\
\midrule
\textbf{Origin} 			& UR-R03 \\
\midrule
\textbf{Priority}		& Conditional \\
\midrule
\textbf{Stability} 		& Stable \\
\midrule
\textbf{Description} 	& Runtime of the simulator must be linear (approximately) in the number of hosts. \\
\bottomrule
\end{tabular}
\caption{Non-functional requirement SR-NF-P01.}
\label{tab:srnfp01}
\end{table*}
\end{center}

\begin{center}
\begin{table*}[htbp]
\centering
\begin{tabular}{@{}p{2.5cm} p{9cm}@{}} 
\toprule
\textbf{ID} 				& SR-NF-UI01\\
\midrule
\textbf{Name} 			& Simulation parameters \\
\midrule
\textbf{Type} 			& Interface \\
\midrule
\textbf{Origin} 			& Analyst \\
\midrule
\textbf{Priority}		& Essential \\
\midrule
\textbf{Stability} 		& Stable \\
\midrule
\textbf{Description} 	& To perform simulations, users only need to specify the simulation parameters in an \gls{xml} file. \\
\bottomrule
\end{tabular}
\caption{Non-functional requirement SR-NF-UI01.}
\label{tab:srnfui01}
\end{table*}
\end{center}

\begin{center}
\begin{table*}[htbp]
\centering
\begin{tabular}{@{}p{2.5cm} p{9cm}@{}}  
\toprule
\textbf{ID} 				& SR-NF-UI02\\
\midrule
\textbf{Name} 			& Progress bar \\
\midrule
\textbf{Type} 			& Interface \\
\midrule
\textbf{Origin} 			& Analyst \\
\midrule
\textbf{Priority}		& Conditional \\
\midrule
\textbf{Stability} 		& Stable \\
\midrule
\textbf{Description} 	& The simulations should include a progress bar. \\
\bottomrule
\end{tabular}
\caption{Non-functional requirement SR-NF-UI02.}
\label{tab:srnfui02}
\end{table*}
\end{center}

\section{Marco Regulador}
\label{sec:regulatory_framework}

This section discusses the necessary constraints taking into account the regulatory \gls{framework}. Specifically, the legal restrictions applicable to the simulator are specified.

\subsection{Restricciones Legales}
\label{sec:legal_constraints}

In the real BOINC system, users must be registered, and \gls{boinc} databases handle confidential information from users, so it is necessary to ensure that third parties can not access that information. One solution is to encrypt the information transmitted following some cryptographic \gls{protocol}. In Spain, this requirement is specified in the article 104 of the RD 1720/2007 \cite{boe2008}, which deals with the Spanish Data Protection Law. 

In contrast, the developed application does not use private data from users, and neither transmits any confidential information to third-parties, because it is just a simulator that does not even require Internet access.

On the other hand, it is crucial that our simulator be available as an \gls{opensource} software. We want it to be such that anyone can redistribute the code or modify it by the terms of the GNU Lesser General Public License (LGPL) \cite{gnulgpl}. To do this, our simulator is available on the following website: \url{https://www.arcos.inf.uc3m.es/~combos/}.

\afterpage{\blankpage} % blank page
\lhead[\thepage]{CHAPTER \thechapter. DESIGN}
\chead[]{}
\rhead[A Complete Simulator for Volunteer Computing Environments\leftmark]{\thepage}
\renewcommand{\headrulewidth}{0.5pt}

\lfoot[]{}
\cfoot[]{}
\rfoot[]{}
\renewcommand{\footrulewidth}{0pt}

%% This is an example first chapter.  You should put chapter/appendix that you
%% write into a separate file, and add a line \include{yourfilename} to
%% main.tex, where `yourfilename.tex' is the name of the chapter/appendix file.
%% You can process specific files by typing their names in at the 
%% \files=
%% prompt when you run the file main.tex through LaTeX.
\chapter{Diseño}
\label{ch:design}
\markboth{}{DESIGN}

En este capítulo se realiza una descripción completa del simulador desarrollado, incluyendo la arquitectura interna y los diferentes componentes software.

La sección \ref{sec:solution_selection} discute la solución elegida y la compara con las alternativas consideradas. La sección \ref{sec:simulator_components} describe cada uno de los componentes que componen el simulador.

\section{Solución elegida}
\label{sec:solution_selection}

Para que los profesores de la asignatura Estructura de Computadores puedan hacer uso de una herramienta que sirva de ayuda para la explicación de los conceptos teóricos de la asignatura, y los alumnos puedan utilizarla para comprender estos conceptos y realizar posteriormente las prácticas de la asignatura, se propone el diseño e implementación de una herramienta web que simule con realismo en funcionamiento de un procesador elemental con unidad de control microprogramable.

Este simulador, será desarrollado como una herramienta web debido a la portabilidad que proporciona, ya que podrá ser ejecutado sobre un gran número de diferentes dispositivos independientemente del sistema operativo que utilice, puesto que únicamente necesita un navegador web para su correcto funcionamiento. De esta forma, los profesores y alumnos podrán hacer uso de la herramienta sin depender de su instalación en el dispositivo a utilizar, incluso pudiendo los alumnos realizar las prácticas sobre dispositivos móviles.

Para lograr dicha portabilidad, el simulador ha sido desarrollado en HTML5 (HTML + JavaScript + CSS) haciendo posible su ejecución en cualquier plataforma (smartphones, tablet, PC, etc.) que pueden ejecutar Microsoft Edge, Mozilla Firefox, Google Chrome o Safari. Además, la herramienta depende de los siguientes frameworks/bibliotecas: JQuery, JQueryUI, JQuery Mobile, Knockout y BootStrap.

Por tanto, la solución elegida es capaz de unificar en una misma herramienta todas las funcionalidades requeridas para la enseñanza de Estructura de computadores con un alto nivel de detalle, con alta disponibilidad al facilitarse su como una herramienta web, y con una gran portabilidad puesto que podrá ser ejecutada sobre un gran número de diversos dispositivos.



\section{Arquitectura de WepSIM}
\label{sec:simulator_architecture}

La arquitectura de la solución presentada en este trabajo consta de tres elementos principales:

\begin{itemize}
\item Modelo hardware: permite definir el hardware a usar.
\item Modelo software: permite definir el juego de instrucciones a utilizar.
\item Moto de simulación: simula el funcionamiento del hardware ejecutando el microcódigo/lenguaje máquina definido con anterioridad.
\end{itemize}

El modelo hardware permite definir los distintos elementos típicos de un computador (memoria principal, procesador, etc.) de una forma modular. La forma de definir estos elementos equilibra dos objetivos contrapuestos: es suficientemente completa como para imitar los principales aspectos de la realidad, pero es lo suficientemente mínima para facilitar su uso. Ante todo se persigue que sea una herramienta didáctica.

El modelo software permite definir el microcódigo y el ensamblador basado en este microcódigo de la forma tan intuitiva posible. El ensamblador a usar viene dado por un conjunto de instrucciones que puede ser definido por el usuario e intenta ser lo suficientemente flexible como para poder definir diferentes tipos y juegos de instrucciones, como por ejemplo MIPS o ARM.

El tercer elemento de la arquitectura propuesta es un motor que toma como entrada el modelo hardware descrito y el modelo software de trabajo, y se encarga de mostrar el funcionamiento del hardware con el software dado.

\subsection{Modelo hardware}


\subsection{Modelo software}


\subsection{Motor del simulador}


\lhead[\thepage]{CAPÍTULO \thechapter. IMPLEMENTACIÓN Y DESPLIEGUE}
\chead[]{}
\rhead[WepSIM: Simulador de procesador elemental con unidad de control microprogramada\leftmark]{\thepage}
\renewcommand{\headrulewidth}{0.5pt}

\lfoot[]{}
\cfoot[]{}
\rfoot[]{}
\renewcommand{\footrulewidth}{0pt}

%% This is an example first chapter.  You should put chapter/appendix that you
%% write into a separate file, and add a line \include{yourfilename} to
%% main.tex, where `yourfilename.tex' is the name of the chapter/appendix file.
%% You can process specific files by typing their names in at the 
%% \files=
%% prompt when you run the file main.tex through LaTeX.
\chapter{Implementación y despliegue}
\label{ch:implementation_and_deployment}
\markboth{}{IMPLEMENTATION}


Este capítulo trata de la implementación y despliegue del software. En cuanto a la implementación del sistema, se explican las partes más complicadas del código en (Sección  \ref{sec:implementation}, \textit{\nameref{sec:implementation}}). Por otro lado, explicamos los pasos necesarios para desplegar el sistema final (Sección \ref{sec:deployment}, \textit{\nameref{sec:deployment}})


\section{Implementación}
\label{sec:implementation}


Como hemos explicado en el capítulo \ref{ch:analysis}, \textit{\nameref{ch:analysis}}, hemos implementado el simulador utilizando el lenguaje de programación JavaScript junto con HTML5, CSS y las bibliotecas/frameworks JQuery, JQueryUI, JQuery Mobile, Knockout y BootStrap. El motor de simulación es el encargado de ejecutar cada uno de los ciclos de reloj del simulador, tomando como entradas tanto el modelo hardware como el modelo software, pero el desarrollador ha debido de diseñar e implementar el algoritmo que posibilita esta ejecución.

Además, hemos trabajado en conseguir una herramienta que sea capaz de generar la memoria de control mediante la definición del juego de instrucciones por parte del usuario, y de generar el código binario asociado al código ensamblador definido por el usuario; el cual depende del juego de instrucciones definido previamente. Para ello, se han diseñado e implementado dos compiladores diferentes, capaces de generar los binarios correspondientes además de las estructuras de datos necesarias para ayudar al motor de simulación a lo largo de la ejecución.

De esta forma, en \ref{alg:core_simulator_pseudocode} podemos ver el pseudocódigo de como en cada ciclo de reloj se realiza la actualización de cada módulo hardware mediante la activación de señales y la propagación de los resultados de cada operación. En \ref{alg:firmware_compiler_pseudocode}, podemos observar el pseudocódigo del proceso de generación de la memoria de control, mientras que en \ref{alg:assembly_compiler_pseudocode} podemos ver el proceso de compilación del código ensamblador definido por el usuario en función de la memoria de control previamente generada.

\vspace{1cm}

\begin{algorithm}[h]
	\caption{Proceso de ejecución de ciclo de reloj}
	\label{alg:core_simulator_pseudocode}
  	\scriptsize
  	\setstretch{1.35}
	\begin{algorithmic}[1]
		\Function{client\_main}{ }
		\While {time < max\_time}
		\State increase $wall\_cpu\_time$ to the running project
		\State \Call{update\_debt}{}
		\State \Call{update\_deadline\_missed}{}
		\State \Call{cpu\_scheduling}{}
		\State \Call{signal}{} Work fetch process
		\State \Call{wait}{} $scheduling\_interval$
		\EndWhile	
		\State Return
		\EndFunction
	\end{algorithmic}
\end{algorithm}

\clearpage

\begin{algorithm}[h]
	\caption{Proceso de compilación del juego de instrucciones}
	\label{alg:firmware_compiler_pseudocode}
  	\scriptsize
  	\setstretch{1.35}
	\begin{algorithmic}[1]
		\Function{work\_fetch}{ }
		\State $project = null$
		\While {time < max\_time}
		\For {each project $p$ in $projects$}
		\If {$p$ meets the requirements}
		\State $project = p$
		\EndIf
		\EndFor
		\If {$project$ and not $deadlines\_missed$}
		\State \Call{ask\_for\_work}{$project$}
		\EndIf		
		\State \Call{wait}{} $work\_fetch\_period$
		\EndWhile	
		\State \Call{signal}{} Client main process
		\State Return
		\EndFunction
	\end{algorithmic}
\end{algorithm}

\begin{algorithm}[h]
	\caption{Proceso de compilación de código ensamblador}
	\label{alg:assembly_compiler_pseudocode}
  	\scriptsize
  	\setstretch{1.35}
	\begin{algorithmic}[1]
		\Function{work\_fetch}{ }
		\State $project = null$
		\While {time < max\_time}
		\For {each project $p$ in $projects$}
		\If {$p$ meets the requirements}
		\State $project = p$
		\EndIf
		\EndFor
		\If {$project$ and not $deadlines\_missed$}
		\State \Call{ask\_for\_work}{$project$}
		\EndIf		
		\State \Call{wait}{} $work\_fetch\_period$
		\EndWhile	
		\State \Call{signal}{} Client main process
		\State Return
		\EndFunction
	\end{algorithmic}
\end{algorithm}


\section{Despliegue}
\label{sec:deployment}

En esta sección se presenta el despliegue de la herramienta. Para ello, indicamos las especificaciones técnicas recomendadas para que el usuario final obtenga la mejor experiencia posible con la herramienta:

\begin{itemize}

\item \textbf{Sistema Operativo}: Ubuntu 16.04.2 LTS (Linux distribution) /Windows 10 / MacOS 10.12.5.

\item \textbf{Procesador}: Intel(R) Core(TM) i3 CPU 6300 @3.8GHz or higher.

\item \textbf{\gls{ram}}: 4 GB or higher.

\item \textbf{Almacenamiento}: 1 GB of free space in the Hard Disk Drive (recomendado para el navegador web).

\item \textbf{Red}: La conexión a internet es necesaria para el acceso a la herramienta web.

\item \textbf{Software}: Los siguientes navegadores web son los recomendados para el uso de la herramienta:

	\begin{itemize}

	\item[1.] Mozilla Firefox.
	
	\item[2.] Google Chrome.
	
	\item[3.] Microsoft Edge.
	
	\item[4.] Safari.

	\end{itemize}

\end{itemize}

\begin{figure}[htbp]
 	\centering
 	\includegraphics[width=10cm]{figures/folder_diagram}
 	\caption{Estructura de ficheros.}
	\label{fig:folder_structure}
\end{figure}

En caso de desear el usuario descargar el código fuente de la herramienta para realizar cualquier modificación en la definición del modelo hardware o cualquier otro módulo, es necesario explicar la estructura de ficheros que componen el simulador y las dependencias que existen entre sí. De esta forma, en \ref{fig:folder_structure} podemos ver los ficheros que componen la herramienta web, los cuales tienen una serie de dependencias indicadas en \ref{fig:files_dependencies}.

\begin{figure}[htbp]
 	\centering
 	\includegraphics[width=15.5cm]{figures/dependencies_diagram}
 	\caption{Dependencias entre ficheros.}
	\label{fig:files_dependencies}
\end{figure}

Los ficheros que componen WepSIM son detallados a continuación:

\begin{itemize}

\item \textbf{index.html: } este fichero se encarga de la vista de la herramienta, generando la interfaz de usuario de la herramienta.

\item \textbf{gpl.txt y lgpl.txt: } estos ficheros contienen la especificación de las licencias del software.

\item \textbf{sim\char`_cfg.js} este fichero contiene las estructuras de datos de la configuración de la herramienta.

\item \textbf{sim\char`_core\char`_ctrl.js:}  este fichero contiene la implementación del motor de simulación de la herramienta.

\item \textbf{sim\char`_core\char`_ui.js: } este fichero contiene la implementación del motor de la interfaz de usuario de la herramienta.

\item \textbf{sim\char`_hw\char`_cpu.js: } este fichero contiene la definición del modelo hardware de la cpu del simulador.

\item \textbf{sim\char`_hw\char`_io.js: } este fichero contiene la definición del modelo hardware del módulo de generación de interrupciones del simulador.

\item \textbf{sim\char`_hw\char`_kbd.js: } este fichero contiene la definición del modelo hardware del módulo del teclado del simulador.

\item \textbf{sim\char`_hw\char`_mem.js: } este fichero contiene la definición del modelo hardware de la memoria principal del simulador.

\item \textbf{sim\char`_hw\char`_mem.js: } este fichero contiene la definición del modelo hardware de la pantalla del simulador.

\item \textbf{sim\char`_lang.js: } este fichero contiene la implementación de las funciones principales del parser de ficheros de la herramienta.

\item \textbf{sim\char`_lang\char`_asm.js: } este fichero contiene la implementación del compilador de código ensamblador de la herramienta.

\item \textbf{sim\char`_lang\char`_firm.js: } este fichero contiene la implementación del compilador de firmware de la herramienta.

\item \textbf{images/cpu.svg: } este fichero contiene la definición de la imagen vectorial de la cpu de la herramienta.

\item \textbf{images/cpu.svg: } este fichero contiene la definición de la imagen vectorial de la unidad de control de la herramienta.

\item \textbf{external folder: } este directorio contiene las bibliotecas y frameworks que necesita el simulador para su correcto funcionamiento, como son JQuery, JQueryUI, JQuery Mobile, Knockout y BootStrap.

\end{itemize}

En el apéndice \ref{ch:user_manual} se presenta el manual completo de usuario de la herramienta, que incluye la especificación del modelo hardware implementado y la explicación de uso del simulador, indicando algunos ejemplos docentes para aprender el uso de esta herramienta.

\afterpage{\blankpage} % blank page
\lhead[\thepage]{CHAPTER \thechapter. VERIFICATION, VALIDATION AND EVALUATION}
\chead[]{}
\rhead[A Complete Simulator for Volunteer Computing Environments\leftmark]{\thepage}
\renewcommand{\headrulewidth}{0.5pt}

\lfoot[]{}
\cfoot[]{}
\rfoot[]{}
\renewcommand{\footrulewidth}{0pt}

%% This is an example first chapter.  You should put chapter/appendix that you
%% write into a separate file, and add a line \include{yourfilename} to
%% main.tex, where `yourfilename.tex' is the name of the chapter/appendix file.
%% You can process specific files by typing their names in at the 
%% \files=
%% prompt when you run the file main.tex through LaTeX.
\chapter{Verificación, validación y evaluación}
\label{ch:verification_validation_and_evaluation}
\markboth{}{VERIFICATION, VALIDATION AND EVALUATION}

This chapter details the verification, validation and evaluation of the project. First, we present the verification and validation of the simulator (Section \ref{sec:verification_and_validation}, \textit{\nameref{sec:verification_and_validation}}), and we detail a series of tests that allowed us to verify that we had met all the requirements set in Chapter \ref{ch:analysis} (\textit{\nameref{ch:analysis}}). After this, we show the validation of the outputs of the simulations, demonstrating that the simulator performs accurate and realistic simulations. We also display a study of the performance of the simulator (Section \ref{sec:performance_study}, \textit{\nameref{sec:performance_study}}), in which we show that \gls{comsimboinc} is efficient and scalable. Finally, we present several case studies of the simulator usage (Section \ref{sec:case_studies}, \textit{\nameref{sec:case_studies}}), with the corresponding analysis and evaluation of the results.

We have used the drand48 Linux functions \cite{drand48} as random number generator in our simulations. These functions generate pseudo-random numbers using the linear congruential algorithm and 48-bit integer arithmetic. Each simulation result presented in this chapter is based on the average of 20 runs. For a 95\% interval, the error is less than $\pm$ 3\% for all values.

\section{Verificación y validación}
\label{sec:verification_and_validation}

The main objective of this section is to verify that all the requirements set out in Chapter \ref{ch:analysis} (\textit{\nameref{ch:analysis}}) have been fulfilled. In addition, we validate the results provided by \gls{comsimboinc}, comparing them to the results of SimBOINC and to the statistical results of the official \gls{boinc} webpage.

In software engineering, verification and validation are the processes of checking that a software system meets specifications and that it fulfills its intended purpose. As explained in Chapter \ref{ch:analysis} (\textit{\nameref{ch:analysis}}), the customer initially sets the requirements desired for the final product (user requirements). From there, analysts specify software requirements (functional and non-functional requirements). In order to verify that the project requirements are met, verification and validation processes are needed (see Figure \ref{fig:verification_validation}).

\vspace{1cm}

\begin{figure}[htb]
 	\centering
 	\includegraphics[width=12cm]{figures/verification_validation}
 	\caption{Software verification and validation.}
	\label{fig:verification_validation}
\end{figure}

\vspace{1cm}

\textit{Software verification} is the process of evaluating work-products (not the actual final product) of a development phase to determine whether they meet the specified requirements for that phase (the software requirements). \textit{Software validation} is the process of evaluating the final product at the end of the development process to determine whether it satisfies the requirements specified by the user at the beginning of the project \cite{verification}.


\subsection{Pruebas de verificación}

In order to perform the verification tests, we have followed a dynamic process during the development phase of the software. With these tests we wanted to answer the question: ``Are we building the product right?''. Table \ref{tab:verification_tests} provides the template used for the verification tests. Note that the ID format is VET-XX, where XX indicates the verification test number.

\clearpage

\begin{center}
\begin{table*}[htb]
\centering
\begin{tabular}{@{}p{2.5cm} p{9cm}@{}} 
\toprule
\textbf{ID} 					& Test ID. \\
\midrule
\textbf{Name} 				& Test name. \\
\midrule
\textbf{Requirements} 		& Software requirements fulfilled with this test. \\
\midrule
\textbf{Description} 		& Test description. \\
\midrule
\textbf{Preconditions}		& Predicates that must always be true before performing the test. \\
\midrule
\textbf{Procedure}			& A fixed, step-by-step sequence of activities performed by the test. \\
\midrule
\textbf{Postconditions} 		& Predicates that must always be true just after performing the test. \\
\midrule
\textbf{Evaluation} 			& \textit{Passed} or \textit{Failed}. \\
\bottomrule
\end{tabular}
\caption{Template for verification tests.}
\label{tab:verification_tests}
\end{table*}
\end{center}

Then, we specify the verification tests.

\vspace{0.7cm}

\begin{center}
\begin{table*}[htb]
\centering
\begin{tabular}{@{}p{2.5cm} p{13cm}@{}} 
\toprule
\textbf{ID} 					& VET-01 \\
\midrule
\textbf{Name} 				& Platform. \\
\midrule
\textbf{Requirements} 		& SR-NF-PL01, SR-NF-PL02, SR-NF-PL03, SR-NF-UI02. \\
\midrule
\textbf{Description} 		& Verify that the software can be used on the platform and is developed with the tools specified in the requirements. \\
\midrule
\textbf{Preconditions}		& 1. Use a machine with Ubuntu 14.04 operating system.\\
							& 2. GCC (GNU Compiler) 5.1 or higher must be installed on the machine. \\
							& 3. The user must be located in the main directory of the \gls{comsimboinc} application. \\
\midrule
\textbf{Procedure}			& 1. Check the code of the simulator source files (all these files are inside the /Files folder). \\
							& 2. Run the generator script with the default parameters to create the simulation files.\\
							& 3. Run the simulation.\\
\midrule
\textbf{Postconditions} 		& 1. All source files must be written in C programming language.\\
							& 2. The implementation, setup and control of the simulations must be carried out using the MSG API of the SimGrid toolkit.\\
							& 3. Simulations run while a progress bar indicates the percentage of execution. \\			
							& 4. The simulator must successfully finish its execution in the specified operating system. \\
\midrule
\textbf{Evaluation} 			& Passed \\
\bottomrule
\end{tabular}
\caption{Verification test VET-01.}
\label{tab:vet01}
\end{table*}
\end{center}


\begin{center}
\begin{table*}[htb]
\centering
\begin{tabular}{@{}p{2.5cm} p{13cm}@{}} 
\toprule
\textbf{ID} 					& VET-02 \\
\midrule
\textbf{Name} 				& Realistic BOINC elements in simulations. \\
\midrule
\textbf{Requirements} 		& SR-F-F06, SR-NF-UI01, SR-NF-UI02. \\
\midrule
\textbf{Description} 		& Verify that the simulator allows to simulate all the BOINC actual elements. \\
\midrule
\textbf{Preconditions}		& 1. The user must be located in the main directory of the \gls{comsimboinc} application.\\
\midrule
\textbf{Procedure}			& 1. Specify the following elements in the simulation parameters: tasks, volunteer hosts, servers, data servers, networks, and hosts availability.\\
							& 2. Run the generator script to create the simulation files.\\
							& 3. Run the simulation. \\
\midrule
\textbf{Postconditions} 		& 1. Simulations run while a progress bar indicates the percentage of execution. \\		
							& 2. The simulator must successfully finish its execution. \\
\midrule
\textbf{Evaluation} 			& Passed \\
\bottomrule
\end{tabular}
\caption{Verification test VET-02.}
\label{tab:vet02}
\end{table*}
\end{center}


\begin{center}
\begin{table*}[htb]
\centering
\begin{tabular}{@{}p{2.5cm} p{13cm}@{}} 
\toprule
\textbf{ID} 					& VET-03 \\
\midrule
\textbf{Name} 				& Statistics of BOINC projects. \\
\midrule
\textbf{Requirements} 		& SR-F-F02, SR-NF-UI01, SR-NF-UI02. \\
\midrule
\textbf{Description} 		& Verify that the outputs of the simulator are the same as those published by BOINCstats \cite{BOINC2016}. The outputs are: credits, hosts, active hosts, and \acrshort{flops}.\\
\midrule
\textbf{Preconditions}		& 1. The user must be located in the main directory of the \gls{comsimboinc} application. \\
\midrule
\textbf{Procedure}			& 1. Specify the simulation parameters.\\
							& 2. Run the generator script to create the simulation files.\\
							& 3. Run the simulation.\\
\midrule
\textbf{Postconditions} 		& 1. Simulations run while a progress bar indicates the percentage of execution. \\
							& 2. The simulator must successfully finish its execution. \\
							& 3. The outputs of the simulator contain at least: credits, hosts, active hosts, and \acrshort{flops} (The same as those published by BOINCstats \cite{BOINC2016}). \\
\midrule
\textbf{Evaluation} 			& Passed \\
\bottomrule
\end{tabular}
\caption{Verification test VET-03.}
\label{tab:vet03}
\end{table*}
\end{center}


\begin{center}
\begin{table*}[htb]
\centering
\begin{tabular}{@{}p{2.5cm} p{13cm}@{}} 
\toprule
\textbf{ID} 					& VET-04 \\
\midrule
\textbf{Name} 				& Multiple BOINC proyects simultaneously. \\
\midrule
\textbf{Requirements} 		& SR-F-F04, SR-NF-UI01, SR-NF-UI02. \\
\midrule
\textbf{Description} 		& Verify that the simulator allows multiple project simulations simultaneously. \\
\midrule
\textbf{Preconditions}		& 1. The user must be located in the main directory of the \gls{comsimboinc} application. \\
\midrule
\textbf{Procedure}			& 1. Specify the simulation parameters of three different projects (for example, the SETI@home, Einstein@home, and LHC@home projects).\\
							& 2. Run the generator script to create the simulation files.\\
							& 3. Run the simulation.\\
\midrule
\textbf{Postconditions} 		& 1. Simulations run while a progress bar indicates the percentage of execution. \\
							& 2. The simulator must successfully finish its execution. \\
\midrule
\textbf{Evaluation} 			& Passed \\
\bottomrule
\end{tabular}
\caption{Verification test VET-04.}
\label{tab:vet04}
\end{table*}
\end{center}


\begin{center}
\begin{table*}[htb]
\centering
\begin{tabular}{@{}p{2.5cm} p{13cm}@{}} 
\toprule
\textbf{ID} 					& VET-05 \\
\midrule
\textbf{Name} 				& BOINC client scheduler. \\
\midrule
\textbf{Requirements} 		& SR-F-F05, SR-NF-UI01, SR-NF-UI02. \\
\midrule
\textbf{Description} 		& Verify that the client scheduler implemented produces the same results as the actual BOINC scheduler. \\
\midrule
\textbf{Preconditions}		& 1. The user must be located in the main directory of the \gls{comsimboinc} application. \\
\midrule
\textbf{Procedure}			& 1. Specify the client side simulation parameters of three different projects: SETI@home, Einstein@home, and LHC@home. \\
							& 2. Run the generator script to create the simulation files.\\
							& 3. Run the simulation.\\
\midrule
\textbf{Postconditions} 		& 1. Simulations run while a progress bar indicates the percentage of execution. \\
							& 2. The simulator must successfully finish its execution. \\
							& 3. The outputs of the simulation are the same as the real BOINC client scheduler (it is detailed in \ref{subsec:validation_of_the_client_scheduler}, \textit{\nameref{subsec:validation_of_the_client_scheduler}}).  \\
\midrule
\textbf{Evaluation} 			& Passed \\
\bottomrule
\end{tabular}
\caption{Verification test VET-05.}
\label{tab:vet05}
\end{table*}
\end{center}


\begin{center}
\begin{table*}[htb]
\centering
\begin{tabular}{@{}p{2.5cm} p{13cm}@{}} 
\toprule
\textbf{ID} 					& VET-06 \\
\midrule
\textbf{Name} 				& Accurate simulations of BOINC projects. \\
\midrule
\textbf{Requirements} 		& SR-F-F01, SR-F-F03, SR-NF-UI01, SR-NF-UI02. \\
\midrule
\textbf{Description} 		& Verify that the outputs of the simulator for existing projects (SETI@home, Einstein@home, and LHC@home) should be almost identical to those published in BOINCstats \cite{BOINC2016}. \\
\midrule
\textbf{Preconditions}		& 1. The user must be located in the main directory of the \gls{comsimboinc} application. \\
\midrule
\textbf{Procedure}			& 1. Specify the simulation parameters of three different projects: SETI@home, Einstein@home, and LHC@home. \\
							& 2. Run the generator script to create the simulation files.\\
							& 3. Run the simulation.\\
\midrule
\textbf{Postconditions} 		& 1. Simulations run while a progress bar indicates the percentage of execution. \\
							& 2. The simulator must successfully finish its execution. \\
							& 3. The outputs of the simulation are the same as the actual BOINC projects, in terms of \acrshort{flops} and credit (it is detailed in \ref{subsec:validation_of_the_whole_simulator}, \textit{\nameref{subsec:validation_of_the_whole_simulator}}). \\
\midrule
\textbf{Evaluation} 			& Passed \\
\bottomrule
\end{tabular}
\caption{Verification test VET-06.}
\label{tab:vet06}
\end{table*}
\end{center}


\begin{center}
\begin{table*}[htb]
\centering
\begin{tabular}{@{}p{2.5cm} p{13cm}@{}} 
\toprule
\textbf{ID} 					& VET-07 \\
\midrule
\textbf{Name} 				& Large simulations. \\
\midrule
\textbf{Requirements} 		& SR-NF-S01, SR-NF-UI01, SR-NF-UI02. \\
\midrule
\textbf{Description} 		& Verify the application is able to perform simulations
with more than 100,000 hosts in a machine with at least 8 GB of \gls{ram}. \\
\midrule
\textbf{Preconditions}		& 1. Use a machine with at least 8GB or \gls{ram}. \\
							& 2. The user must be located in the main directory of the \gls{comsimboinc} application. \\
\midrule
\textbf{Procedure}			& 1. Specify the simulation parameters with more than 100,000 hosts. \\
							& 2. Run the generator script to create the simulation files.\\
							& 3. Run the simulation.\\
\midrule
\textbf{Postconditions} 		& 1. Simulations run while a progress bar indicates the percentage of execution. \\
							& 2. The simulator must successfully finish its execution. \\
\midrule
\textbf{Evaluation} 			& Passed \\
\bottomrule
\end{tabular}
\caption{Verification test VET-07.}
\label{tab:vet07}
\end{table*}
\end{center}


\begin{center}
\begin{table*}[htb]
\centering
\begin{tabular}{@{}p{2.5cm} p{13cm}@{}} 
\toprule
\textbf{ID} 					& VET-08 \\
\midrule
\textbf{Name} 				& Execution time. \\
\midrule
\textbf{Requirements} 		& SR-NF-P01, SR-NF-UI01, SR-NF-UI02. \\
\midrule
\textbf{Description} 		& Check that simulations follow a linear execution time. \\
\midrule
\textbf{Preconditions}		& 1. The user must be located in the main directory of the \gls{comsimboinc} application. \\
\midrule
\textbf{Procedure}			& 1. Specify the simulation parameters with different workloads. \\
							& 2. Run the generator script to create the simulation files.\\
							& 3. Run the simulation.\\
							
							& 4. Go to 2 specifying different simulation parameters.\\
\midrule
\textbf{Postconditions} 		& 1. Simulations run while a progress bar indicates the percentage of execution. \\
							& 2. The simulator must successfully finish its execution. \\
							& 3. Check that executions follow a linear execution time when increasing the workload (it is detailed in \ref{sec:performance_study}, \textit{\nameref{sec:performance_study}}). \\
\midrule
\textbf{Evaluation} 			& Passed \\
\bottomrule
\end{tabular}
\caption{Verification test VET-08.}
\label{tab:vet08}
\end{table*}
\end{center}


\clearpage


The verification test traceability matrix (Table \ref{tab:verification_matrix}) determines that all the software requirements have been verified during the development phase of the project.

\vspace{2cm}


\begin{table}[htb]
\ra{1.3}
  \centering
  \begin{tabular}{@{}L{3cm}C{0.7cm}C{0.7cm}C{0.7cm}C{0.7cm}C{0.7cm}C{0.7cm}C{0.7cm}C{0.7cm}@{}}
    \toprule
     \thead{Requirements} & \rothead{VET-01} & \rothead{VET-02} & \rothead{VET-03} & \rothead{VET-04} & \rothead{VET-05} & \rothead{VET-06} & \rothead{VET-07} & \rothead{VET-08}\\
    \midrule
    SR-F-F01 & & & & & & \ding{51} & &\\
    SR-F-F02 & & & \ding{51} & & & & &\\
    SR-F-F03 & & & & & & \ding{51} & &\\
    SR-F-F04 & & & & \ding{51} & & & &\\
    SR-F-F05 & & & & & \ding{51} & & &\\
    SR-F-F06 & & \ding{51} & & & & & &\\
    SR-NF-PL01 & \ding{51} & & & & & & & \\
    SR-NF-PL02 & \ding{51} & & & & & & &\\
    SR-NF-PL03 & \ding{51} & & & & & & &\\
    SR-NF-S01 & & & & & & & \ding{51} &\\
    SR-NF-P01 & & & & & & & & \ding{51}\\
    SR-NF-UI01 & & \ding{51} & \ding{51} & \ding{51} & \ding{51} & \ding{51} & \ding{51} & \ding{51}\\
    SR-NF-UI02 & \ding{51} & \ding{51} & \ding{51} & \ding{51} & \ding{51} & \ding{51} & \ding{51} & \ding{51}\\
    \bottomrule
\end{tabular}
\caption{Verification test traceability matrix.}
\label{tab:verification_matrix}
\end{table}    

\clearpage

\subsection{Validation Tests}

To perform the validation tests, we have checked the final software, comparing it with the user needs specified in Chapter \ref{ch:analysis} (\textit{\nameref{ch:analysis}}). With these tests we want to answer the question: ``Have we built the right product?''. Table \ref{tab:validation_tests} provides the template used for the validation tests. Note that the ID format is VAT-XX, where XX indicates the validation test number.

\begin{center}
\begin{table*}[htb]
\centering
\begin{tabular}{@{}p{2.5cm} p{9cm}@{}} 
\toprule
\textbf{ID} 					& Test ID. \\
\midrule
\textbf{Name} 				& Test name. \\
\midrule
\textbf{Requirements} 		& User requirements fulfilled with this test. \\
\midrule
\textbf{Verification tests} 	& Verification tests that help us to validate this test. \\
\midrule
\textbf{Description} 		& Test description. \\
\midrule
\textbf{Preconditions}		& Predicates that must always be true before performing the test. \\
\midrule
\textbf{Procedure}			& A fixed, step-by-step sequence of activities performed by the test. \\
\midrule
\textbf{Postconditions} 		& Predicates that must always be true just after performing the test. \\
\midrule
\textbf{Evaluation} 			& \textit{Passed} or \textit{Failed}. \\
\bottomrule
\end{tabular}
\caption{Template for validation tests.}
\label{tab:validation_tests}
\end{table*}
\end{center}


Then, we specify the validation tests.


\begin{center}
\begin{table*}[htb]
\centering
\begin{tabular}{@{}p{2.5cm} p{13cm}@{}} 
\toprule
\textbf{ID} 					& VAT-01 \\
\midrule
\textbf{Name} 				& BOINC projects simulation. \\
\midrule
\textbf{Requirements} 		& UR-C01. \\
\midrule
\textbf{Verification tests} 	& VET-03, VET-06. \\
\midrule
\textbf{Description} 		& Validate that the simulator is able to simulate the behavior of BOINC projects. \\
\midrule
\textbf{Preconditions}		&  1. The user must be located in the main directory of the \gls{comsimboinc} application. \\
\midrule
\textbf{Procedure}			& 1. Specify the simulation parameters of three different BOINC projects: SETI@home, Einstein@home, and LHC@home. \\
							& 2. Run the generator script to create the simulation files. \\
							& 3. Run the simulation. \\ 
\midrule
\textbf{Postconditions} 		& 1. The simulator must successfully finish its execution. \\
							& 2. The outputs of the simulation are the same as the actual BOINC projects, in terms of \acrshort{flops} and credit (it is detailed in \ref{subsec:validation_of_the_whole_simulator}, \textit{\nameref{subsec:validation_of_the_whole_simulator}}). \\
\midrule
\textbf{Evaluation} 			& Passed. \\
\bottomrule
\end{tabular}
\caption{Validation test VAT-01.}
\label{tab:vat-01}
\end{table*}
\end{center}


\begin{center}
\begin{table*}[htb]
\centering
\begin{tabular}{@{}p{2.5cm} p{13cm}@{}} 
\toprule
\textbf{ID} 					& VAT-02 \\
\midrule
\textbf{Name} 				& Client \gls{scheduling}. \\
\midrule
\textbf{Requirements} 		& UR-C02. \\
\midrule
\textbf{Verification tests} 	& VET-05. \\
\midrule
\textbf{Description} 		& Validate that the client scheduler implemented produces the same results as the actual BOINC scheduler. \\
\midrule
\textbf{Preconditions}		&  1. The user must be located in the main directory of the \gls{comsimboinc} application. \\
\midrule
\textbf{Procedure}			& 1. Specify the client side simulation parameters of three different projects: SETI@home, Einstein@home, and LHC@home. \\
							& 2. Run the generator script to create the simulation files. \\
							& 3. Run the simulation. \\ 
\midrule
\textbf{Postconditions} 		& 1. The simulator must successfully finish its execution. \\
							& 2. The outputs of the simulation are the same as the real BOINC client scheduler (it is detailed in \ref{subsec:validation_of_the_client_scheduler}, \textit{\nameref{subsec:validation_of_the_client_scheduler}}). \\
\midrule
\textbf{Evaluation} 			& Passed. \\
\bottomrule
\end{tabular}
\caption{Validation test VAT-02.}
\label{tab:vat-02}
\end{table*}
\end{center}


\begin{center}
\begin{table*}[htb]
\centering
\begin{tabular}{@{}p{2.5cm} p{13cm}@{}} 
\toprule
\textbf{ID} 					& VAT-03 \\
\midrule
\textbf{Name} 				& Simulation components. \\
\midrule
\textbf{Requirements} 		& UR-C03. \\
\midrule
\textbf{Verification tests} 	& VET-02. \\
\midrule
\textbf{Description} 		& Validate that the simulations cover all the elements of the BOINC infrastructure. \\
\midrule
\textbf{Preconditions}		&  1. The user must be located in the main directory of the ComBoS application. \\
\midrule
\textbf{Procedure}			& 1. Specify the following elements in the simulation parameters: tasks, volunteer hosts, servers, data servers, networks, and hosts availability. \\
							& 2. Run the generator script to create the simulation files. \\
							& 3. Run the simulation. \\ 
\midrule
\textbf{Postconditions} 		& 1. The simulator must successfully finish its execution. \\
\midrule
\textbf{Evaluation} 			& Passed. \\
\bottomrule
\end{tabular}
\caption{Validation test VAT-03.}
\label{tab:vat-03}
\end{table*}
\end{center}


\begin{center}
\begin{table*}[htb]
\centering
\begin{tabular}{@{}p{2.5cm} p{13cm}@{}} 
\toprule
\textbf{ID} 					& VAT-04 \\
\midrule
\textbf{Name} 				& Platform. \\
\midrule
\textbf{Requirements} 		& UR-R01, UR-R02. \\
\midrule
\textbf{Verification tests} 	& VET-01. \\
\midrule
\textbf{Description} 		& Validate that the software can be used on the platform and is developed with the tools specified in the requirements. \\
\midrule
\textbf{Preconditions}		& 1. Use a machine with a Linux operating system. \\
							& 2. The user must be located in the main directory of the \gls{comsimboinc} application. \\
\midrule
\textbf{Procedure}			& 1. Check the code of the simulator source files (all these files are inside the /Files folder). \\
							& 2. Run the generator script with the default parameters to create the simulation files. \\
							& 3. Run the simulation. \\ 
\midrule
\textbf{Postconditions} 		& 1. The simulator must successfully finish its execution. \\
							& 2. The implementation, setup and control of the simulations must be carried out using the MSG API of the SimGrid toolkit. \\
\midrule
\textbf{Evaluation} 			& Passed. \\
\bottomrule
\end{tabular}
\caption{Validation test VAT-04.}
\label{tab:vat-04}
\end{table*}
\end{center}

\clearpage

\begin{center}
\begin{table*}[htb]
\centering
\begin{tabular}{@{}p{2.5cm} p{13cm}@{}} 
\toprule
\textbf{ID} 					& VAT-05 \\
\midrule
\textbf{Name} 				& Scalability. \\
\midrule
\textbf{Requirements} 		& UR-R03. \\
\midrule
\textbf{Verification tests} 	& VET-07. \\
\midrule
\textbf{Description} 		& Validate that the application is able to perform large simulations. \\
\midrule
\textbf{Preconditions}		&  1. The user must be located in the main directory of the ComBoS application. \\
\midrule
\textbf{Procedure}			& 1. Specify the simulation parameters with more than 100,000 hosts. \\
							& 2. Run the generator script to create the simulation files. \\
							& 3. Run the simulation. \\ 
\midrule
\textbf{Postconditions} 		& 1. The simulator must successfully finish its execution. \\
\midrule
\textbf{Evaluation} 			& Passed. \\
\bottomrule
\end{tabular}
\caption{Validation test VAT-05.}
\label{tab:vat-05}
\end{table*}
\end{center}


The validation test traceability matrix (Table \ref{tab:validation_matrix}) determines that all the user needs have been validated in the final product.

\vspace{2cm}


\begin{table}[htb]
\ra{1.3}
  \centering
  \begin{tabular}{@{}L{3cm}C{0.7cm}C{0.7cm}C{0.7cm}C{0.7cm}C{0.7cm}@{}}
    \toprule
     \thead{Requirements} & \rothead{VAT-01} & \rothead{VAT-02} & \rothead{VAT-03} & \rothead{VAT-04} & \rothead{VAT-05}\\
    \midrule
    UR-C01 & \ding{51} & & & & \\
    UR-C02 & & \ding{51} & & & \\
    UR-C03 & & & \ding{51} & & \\
    UR-R01 & & & & \ding{51} & \\
    UR-R02 & & & & \ding{51} & \\
    UR-R03 & & & & & \ding{51} \\
    \bottomrule
\end{tabular}
\caption{Validation test traceability matrix.}
\label{tab:validation_matrix}
\end{table}    


\clearpage


\subsection{Validation of the Client Scheduler}
\label{subsec:validation_of_the_client_scheduler}

To validate the client scheduler of \gls{comsimboinc}, we have compared the results of different executions of the simulator with the equivalent SimBOINC simulations. Of course, as we only want to validate the client scheduler (individually), we have simulated scenarios with no delay caused by network or servers.

All scenarios considered are based on a single client host with three associated projects (Einstein@home, SETI@home and LHC@home). Through the different tests we have varied the priorities of the projects and the time of each simulation. When using hosts with the same power, our goal is to compare the number of tasks executed in each simulator.

As explained in Section \ref{sec:related_work}, \textit{\nameref{sec:related_work}} (Chapter \ref{ch:state_of_the_art}, \textit{\nameref{ch:state_of_the_art}}), SimBOINC simulates the \gls{boinc} client scheduler and its simulations are highly accurate, because it uses almost exactly the \gls{boinc} client's CPU scheduler source code. Tables \ref{tab:validation1}, \ref{tab:validation2} and \ref{tab:validation3} show different test cases:

\begin{itemize}
\item 
Table \ref{tab:validation1} presents the number of tasks executed by a client host of 1.4 · $10^{9}$ \acrshort{flops} on simulations of 100, 500, 1,000, 5,000, and 10,000 hours. The priorities of the three projects are the same, so that each project uses the same runtime (33\% CPU). The results of \gls{comsimboinc} and SimBOINC are almost identical.

\begin{table}[htbp]
\centering
\ra{1.2}
\begin{tabular}{@{}p{1.3cm}p{2.4cm}p{1.8cm}p{1.8cm}p{0.1cm}p{2.4cm}p{1.8cm}p{2.0cm}@{}}
\toprule
& \multicolumn{3}{c}{$SimBOINC$} & & \multicolumn{3}{c}{$ComBoS$}\\
\cmidrule{2-4} \cmidrule{6-8}
Time in hours & Einstein@home (33\%) & SETI@home (33\%) & LHC@home (33\%) && Einstein@home (33\%) & SETI@home (33\%) & LHC@home (33\%)\\ 
\midrule
\multicolumn{1}{r}{100}			& \multicolumn{1}{r}{1}			& \multicolumn{1}{r}{21}			& \multicolumn{1}{r}{33}			&& \multicolumn{1}{r}{1}			& \multicolumn{1}{r}{22}			& \multicolumn{1}{r}{28}			\\
\multicolumn{1}{r}{500}			& \multicolumn{1}{r}{7}			& \multicolumn{1}{r}{108}	& \multicolumn{1}{r}{166}		&& \multicolumn{1}{r}{7}			& \multicolumn{1}{r}{112}		& \multicolumn{1}{r}{163}		\\ 
\multicolumn{1}{r}{1,000}		& \multicolumn{1}{r}{14}			& \multicolumn{1}{r}{220}		& \multicolumn{1}{r}{331}		&& \multicolumn{1}{r}{13}		& \multicolumn{1}{r}{223}		& \multicolumn{1}{r}{333}		\\
\multicolumn{1}{r}{5,000}		& \multicolumn{1}{r}{70}			& \multicolumn{1}{r}{1,103}		& \multicolumn{1}{r}{1,652}		&& \multicolumn{1}{r}{70}		& \multicolumn{1}{r}{1,106}		& \multicolumn{1}{r}{1,659}		\\
\multicolumn{1}{r}{10,000}		& \multicolumn{1}{r}{139}		& \multicolumn{1}{r}{2,214}		& \multicolumn{1}{r}{3,319}		&& \multicolumn{1}{r}{139}		& \multicolumn{1}{r}{2,221}		& \multicolumn{1}{r}{3,331}		\\
\bottomrule
\end{tabular}
\caption{Executed tasks (three projects running on a single host of 1.4 Giga\acrshort{flops}).}
\label{tab:validation1}
\end{table}

\item Table \ref{tab:validation2} proposes a case similar to the previous test. In this case, the host has a power of 5.5 · $10^{9}$ \acrshort{flops} and the priorities of the projects differ. The tasks of the LHC@home project consume 50\% of CPU usage, while the tasks of the Einstein@home and SETI@home projects consume 25\% of the CPU usage each. As in the previous case, the number of tasks executed in \gls{comsimboinc} is practically the same as in the case of SimBOINC.

\begin{table}[htbp]
\centering
\ra{1.2}
\begin{tabular}{@{}p{1.3cm}p{2.4cm}p{1.8cm}p{1.8cm}p{0.1cm}p{2.4cm}p{1.8cm}p{2.0cm}@{}}
\toprule
& \multicolumn{3}{c}{$SimBOINC$} & & \multicolumn{3}{c}{$ComBoS$}\\
\cmidrule{2-4} \cmidrule{6-8}
Time in hours & Einstein@home (25\%) & SETI@home (25\%) & LHC@home (50\%) && Einstein@home (25\%) & SETI@home (25\%) & LHC@home (50\%)\\
\midrule
\multicolumn{1}{r}{100}		& \multicolumn{1}{r}{4}		& \multicolumn{1}{r}{67}		& \multicolumn{1}{r}{181}	&& \multicolumn{1}{r}{4}		& \multicolumn{1}{r}{64}		& \multicolumn{1}{r}{182}	\\
\multicolumn{1}{r}{500}		& \multicolumn{1}{r}{21}		& \multicolumn{1}{r}{332}	& \multicolumn{1}{r}{975}	&& \multicolumn{1}{r}{21}	& \multicolumn{1}{r}{333}		& \multicolumn{1}{r}{975}	\\
\multicolumn{1}{r}{1,000}	& \multicolumn{1}{r}{42}		& \multicolumn{1}{r}{662}	& \multicolumn{1}{r}{1,955}	&& \multicolumn{1}{r}{40}	& \multicolumn{1}{r}{662}		& \multicolumn{1}{r}{1,981}	\\
\multicolumn{1}{r}{5,000}	& \multicolumn{1}{r}{208}	& \multicolumn{1}{r}{3,297}	& \multicolumn{1}{r}{9,831}	&& \multicolumn{1}{r}{206}	& \multicolumn{1}{r}{3,297}	& \multicolumn{1}{r}{9,889}	\\
\multicolumn{1}{r}{10,000}	& \multicolumn{1}{r}{416}	& \multicolumn{1}{r}{6,581}	& \multicolumn{1}{r}{19,637}	&& \multicolumn{1}{r}{413}	& \multicolumn{1}{r}{6,593}	& \multicolumn{1}{r}{19,784}	\\
\bottomrule
\end{tabular}
\caption{Executed tasks (three projects running on a single host of 5.5 Giga\acrshort{flops}).}
\label{tab:validation2}
\end{table}

\item Table \ref{tab:validation3} includes three different test cases. In each test case, a host of \acrshort{flops} 5.5 · $10^{9}$ runs a unique project (100\% of the CPU time). In the first case, the host performs tasks of Einstein@home project and the results are exactly the same in both simulators. In the case of SETI@home and LHC@home projects the results vary minimally.

\begin{table}[htbp]
\centering
\ra{1.2}
\resizebox{\textwidth}{!}{
\begin{tabular}{@{}rrrcrrcrr@{}}
\toprule
& \multicolumn{2}{c}{$Einstein@home(100\%)$} & \phantom{a}& \multicolumn{2}{c}{$SETI@home(100\%)$} & \phantom{a}& \multicolumn{2}{c}{$LHC@home(100\%)$}\\
\cmidrule{2-3} \cmidrule{5-6} \cmidrule{8-9}
Time in hours & SimBOINC & ComBoS && SimBOINC & ComBoS && SimBOINC & ComBoS\\ \midrule
100			& 16			& 16			&& 263			& 263			&& 395			& 394		\\
500			& 82			& 82			&& 1,318			& 1,319			&& 1,978			& 1,972		\\
1,000		& 164		& 164		&& 2,637			& 2,639			&& 3,956			& 3,945		\\
5,000		& 824		& 824		&& 13,177		& 13,195			&& 19,780		& 19,728		\\
10,000		& 1,649		& 1,649		&& 26,315		& 26,390			&& 39,473		& 39,457		\\
\bottomrule
\end{tabular}
}
\caption{Executed tasks (single project running on a single host of 5.5 Giga\acrshort{flops}).}
\label{tab:validation3}
\end{table}

\end{itemize}

If we consider only the client scheduler, \gls{comsimboinc} results match those of SimBOINC, demonstrating the proper functioning of the simulator in this regard.

\subsection{Validation of Whole Simulator}
\label{subsec:validation_of_the_whole_simulator}

To validate the complete simulator, we have relied on data from the BOINCstats website \cite{BOINC2016}, which provides official statistical results of \gls{boinc} projects. In this section, we analyze the behavior of \gls{comsimboinc} considering the simulation results of the SETI@home, Einstein@home and LHC@home projects.

We have used the CPU power traces of the client hosts that make up the VN of each project \cite{SETIflops, EINSTEINflops, LHCflops}. We have not used any other traces. In order to model the availability and unavailability of the hosts, we used the results obtained in \cite{Javadi2011}. This research analyzed about 230,000 hosts' availability traces obtained from the SETI@home project. According to this paper, 21\% of the hosts exhibit truly random availability intervals, and it also measured the goodness of fit of the resulting distributions using standard probability-probability (PP) plots. For availability, the authors saw that in most cases the Weibull distribution is a good fit. For unavailability, the distribution that offers the best fit is the log-normal. The parameters used for the Weibull distribution are $shape=0.393$ and $scale=2.964$. For the log-normal, the parameters obtained and used in \gls{comsimboinc} are a distribution with mean $\mu = -0.586$ and standard deviation $\sigma=2.844$. All these parameters were obtained from \cite{Javadi2011} too. For the network parameters, we have used the bandwidth and latency values of current ADSL networks, and 10 Gbps for the network backbone. SimGrid's models allow us to adjust this network values. We have obtained all the other parameters of the simulations from the official websites of the SETI@home, Einstein@home, and LHC@home projects.

\begin{table}[htbp]
\centering
\ra{1.2}
\resizebox{\textwidth}{!}{
\begin{tabular}{@{}lrrrrcrr@{}}
\toprule
&&& \multicolumn{2}{c}{$BOINCstats$} & \phantom{} & \multicolumn{2}{c}{$ComBoS$}\\
\cmidrule{4-5} \cmidrule{7-8}
Project & Total hosts & Active hosts & $GigaFLOPS$ & $Credit/day$ && $GigaFLOPS$ & $Credit/day$\\
\midrule
SETI@home		& 3,970,427 &	175,220	& 864,711		& 171,785,234		&& 865,001	 & 168,057,478	\\
Einstein@home	& 1,496,566	&	68,338	& 1,044,515		& 208,902,921		&& 1,028,172	 & 205,634,486	\\
LHC@home			& 356,942 	&	15,814	& 7,521			& 1,504,214			&& 7,392		 & 1,393,931		\\
\bottomrule
\end{tabular}
}
\caption{Validation of the whole simulator.}
\label{tab:validation4}
\end{table}

Table~\ref{tab:validation4} compares the actual results of the SETI@home, Einstein@home and LHC@home projects with those obtained with \gls{comsimboinc} in terms of Giga\acrshort{flops} and credits. The error obtained is 2.2\% for credit/day and 0.03\% for Giga\acrshort{flops} compared to the SETI@home project; 1.6\% for credit/day and for Giga\acrshort{flops} compared to the Einstein@home project; and 7.3\% for credit/day and 1.7\% for Giga\acrshort{flops} compared to the LHC@home project. We consider that these results allow us to validate the whole simulator.

\clearpage

\section{Performance Study}
\label{sec:performance_study}

In this section we analyze the performance of the simulator in terms of memory usage and execution time. All measurements were made on a computer with 32 GB of RAM and 8 Intel Core i7 processors running at 2.67 Ghz each. The server runs the Linux 3.13.0-85-generic kernel. It runs the 3.10 version of the SimGrid toolkit.  In spite of the computer has eight cores, each simulation was performed individually in a single core. 

Figure \ref{fig:memory} shows the memory usage of the simulator and Figure \ref{fig:time} shows the execution time by increasing the number of client hosts in each simulation. Note that the tests have been carried out up to 1 million hosts, the same number of active hosts of all \gls{boinc} projects together. Figure \ref{fig:memory} shows a linear ($O(n)$) memory footprint. Figure \ref{fig:time} shows the execution time of the simulator for four different simulation times: 1 day, 2 days, 3 days and 4 days. Both metrics demonstrate that the simulator is highly scalable. This has been possible due to the high performance \cite{Legrand2015} of the SimGrid toolkit.

\vspace{1cm}

\begin{figure}[htbp] 
	\begin{subfigure}{0.5\textwidth}
		\includegraphics[width=\linewidth]{figures/memory}
		\caption{Memory usage.} 
		\label{fig:memory}
	\end{subfigure}
	\hspace*{\fill} % separation between the subfigures
	\begin{subfigure}{0.5\textwidth}
		\includegraphics[width=\linewidth]{figures/time}
		\caption{Execution time.} 
		\label{fig:time}
	\end{subfigure}
	\caption{Performance study.}
	\label{fig:performance}
\end{figure}

\clearpage

\section{Case Studies}
\label{sec:case_studies}




\subsection{Combined Results}


\lhead[\thepage]{CHAPTER \thechapter. PLANNING AND BUDGET}
\chead[]{}
\rhead[A Complete Simulator for Volunteer Computing Environments\leftmark]{\thepage}
\renewcommand{\headrulewidth}{0.5pt}

\lfoot[]{}
\cfoot[]{}
\rfoot[]{}
\renewcommand{\footrulewidth}{0pt}

%% This is an example first chapter.  You should put chapter/appendix that you
%% write into a separate file, and add a line \include{yourfilename} to
%% main.tex, where `yourfilename.tex' is the name of the chapter/appendix file.
%% You can process specific files by typing their names in at the 
%% \files=
%% prompt when you run the file main.tex through LaTeX.
\chapter{Planificación y presupuesto}
\label{ch:planning_and_budget}
\markboth{}{PLANNING AND BUDGET}

This chapter presents a detailed planning of the project (Section \ref{sec:planning}, \textit{\nameref{sec:planning}}). Then, we explain the project costs (Section \ref{sec:budget}, \textit{\nameref{sec:budget}}). At the end of the chapter, we comment on the socio-economic environment of the project ({Section \ref{sec:socioeconomic_environment}, \textit{\nameref{sec:socioeconomic_environment}}}).

\section{Planificación}
\label{sec:planning}

This section includes the complete project planning. First, we describe the software development methodology used. After that, we detail the time duration of each phase of the project, collecting all times in a Gantt chart.

\subsection{Justificación de la Metodología}

Due to its characteristics, we have divided our project into three iterations:

\begin{itemize}

\item \textbf{Basic functionality:} the first iteration has been to achieve the simulation of a simple distributed computing system. The aim of this phase has been to simulate client machines that exchange messages with a server through the a network.

\item \textbf{Client side:} this phase has been to incorporate all the necessary functionality on the client side (described in Chapter \ref{ch:design}, \textit{\nameref{ch:design}}).

\item \textbf{Server side:} this phase has been to incorporate all the necessary functionality on the server side (described in Chapter \ref{ch:design}, \textit{\nameref{ch:design}}).

\end{itemize}

It was necessary to have an iterative methodology used to develop each of the phases independently to join all together in the last stage and obtain the final product. For this purpose, we have analyzed three different software development methodologies: Software prototyping \cite{grimm1998}, the Waterfall model \cite{hebert1983} and the Spiral model \cite{boehm1988}. Software prototyping did not fit well because it requires building a prototype of the software in a short time. The Waterfall model is a sequential design process, used in software development processes, in which progress is seen as flowing steadily downwards (like a waterfall) through different phases. The problem with this methodology is that it does not allow iterations within the software development. Finally, the Spiral model allowed fragmenting the project into different iterations. The model combines the strengths of the other two models (simplicity and flexibility), and uses an iterative process. Although this model is slower than the other two, it allowed us to apply different iterations so we decided to apply it to the whole process.

\subsection{Ciclo de Vida}

The life cycle development process of the project has followed the Spiral lifecycle model \cite{boehm1988}. Figure \ref{fig:spiral_model} shows the Spiral model using a scheme.

\begin{figure}[htbp]
 	\centering
 	\includegraphics[width=12cm]{figures/spiral_model}
 	\caption{Spiral model (Boehm, 2000).}
	\label{fig:spiral_model}
\end{figure}

The Spiral model has four phases, which are repeated during the different iterations of the model. These phases are:

\begin{itemize}

\item \textbf{Planning} (Determine objectives in Figure \ref{fig:spiral_model}): the user requirements are gathered, a feasibility study of the system is performed, and the iteration objectives are determined. 

\item \textbf{Analysis} (Identify and resolve risks in Figure \ref{fig:spiral_model}): a full analysis of requirements is done and the potential risks are identified. This phase ends with a basic design.

\item \textbf{Development and Test}: Code implementation is done. Test cases and test results are performed.

\item \textbf{Evaluation} (Plan the next iteration in Figure \ref{fig:spiral_model}): Customers evaluate the software and provide their feedback. In this case, the student tries to get the supervisor's approval. This is  the \textit{critical task} of the life cycle, since we can only move on to the next iteration of the Spiral lifecycle model if this task is approved.

\end{itemize}

Each phase starts with a design goal and ends with the customer (the supervisor) reviewing the progress so far. As previously explained, we have divided the software development into three iterations: basic functionality, client side, and server side. In the last iteration, the complete software must undergo extensive testing in order to validate the simulator.


\subsection{Tiempo Estimado}

The Gantt chart (Figure \ref{fig:gantt}) shows all the tasks carried out during the project development. This project has been developed within a Collaboration in University Departments Scholarship \cite{colaboracion}, funded by the Spanish Ministry of Education, Culture, and Sport. The project began on November 2st, 2015, and ended on June 22, 2016, making a total of almost eight months of work. During this time, I have worked from Monday to Friday, four hours a day.

The Gantt chart shows all the tasks performed in each iteration of the spiral lifecycle model. Recall that the three iterations were: Basic functionality, Client side and Server side. In addition to the tasks (phases) mentioned above (Planning, Analysis, Development and Test, and Evaluation), we have included the Documentation task at the end of each iteration. The Documentation task has consisted mainly in drafting this bachelor thesis.


\begin{figure}[htbp]
 	\centering
 	\includegraphics[width=16.5cm]{figures/gantt}
 	\caption{Gantt chart.}
	\label{fig:gantt}
\end{figure}

\section{Presupuesto}
\label{sec:budget}

This section details the overall project budget. On the one hand, we present the project costs and, on the other hand, we disclose the offer presented to the customer.

\subsection{Coste del Proyecto}

Table \ref{tab:project_information} summarizes the main features of the project including the total budget. 

\begin{center}
\ra{1.2}
\begin{table*}[htbp]
\centering
\begin{tabular}{@{}p{3.5cm} p{9cm}@{}} 
\toprule
\multicolumn{2}{c}{\textbf{\textit{Project Information}}}\\
\midrule
\textbf{Title} 					& A Complete Simulator for Volunteer Computing Environments \\
\midrule
\textbf{Author} 					& Saúl Alonso Monsalve \\
\midrule
\textbf{Department} 				& Computer Science and Engineering Department \\
\midrule
\textbf{Start date}				& 2nd of November of 2015 \\
\midrule
\textbf{End date}				& 22nd of June of 2016 \\
\midrule
\textbf{Duration} 				& 8 months \\
\midrule
\textbf{Indirect costs ratio} 	& 20 \% \\
\midrule
\textbf{Total budget} 			& 30,526.49 \\
\bottomrule
\end{tabular}
\caption{Project Information.}
\label{tab:project_information}
\end{table*}
\end{center}

Then the total budget of the project is broken down below.

\subsubsection{Costes Directos}

In this part, the direct costs of the project are presented. Table \ref{tab:dhrc} shows the direct costs caused by personnel costs, based on the planning presented in the previous section. The supervisor and the student have played the following roles:

\begin{itemize}

\item \textbf{Supervisor:} Project manager.

\item \textbf{Student:} Analyst, Developer, Tester.

\end{itemize} 

\begin{center}
\ra{1.2}
\begin{table*}[htbp]
\centering
\begin{tabular}{@{}p{3cm} R{3.5cm} R{2.2cm} R{2.4cm}@{}} 
\toprule
\textbf{Category} & \textbf{Cost per hour (\euro)} & \textbf{Hours} & \textbf{Total (\euro)} \\
\midrule
Project manager					& 60 						& 56			& 3,360 \\
Analyst			 				& 35							& 188		& 6,580 \\
Developer		 				& 35							& 316		& 11,060 \\
Tester		 					& 25							& 112		& 2,800 \\
\midrule
\textbf{\textit{Total}}			&							&			& \textbf{23,800.00}\\
\bottomrule
\end{tabular}
\caption{Human resources costs.}
\label{tab:dhrc}
\end{table*}
\end{center}

Table \ref{tab:dec} shows the direct costs caused by equipment acquisition and usage. The chargeable cost, C, is calculated using the following formula:

\begin{equation}
  C = \frac{d \cdot c \cdot u}{D}
\label{eq:costs}
\end{equation}

Where:

\begin{itemize}

\item \textbf{C:} Chargeable cost. It is equivalent to the depreciated value.

\item \textbf{d:} Time the equipment has been used.

\item \textbf{c:} Equipment cost. 

\item \textbf{u:} Project dedication. Percentage of time the equipment has been used.

\item \textbf{D:} Equipment depreciation period.

\end{itemize}

\begin{center}
\ra{1.2}
\begin{table*}[htbp]
\centering
\begin{tabular}{@{}p{2.5cm} C{1.8cm} C{2.1cm} C{2.1cm} C{2.7cm} C{2.3cm}@{}} 
\toprule
\textbf{Concept} & \textbf{Cost, c (\euro)} & \textbf{Dedication, u (\%)} & \textbf{Dedication, d (months)} & \textbf{Depreciation, D (months)} & \textbf{Chargeable cost, C (\euro)}\\
\midrule
Desktop \acrshort{pc}		 			& \multicolumn{1}{r}{799.99}		& \multicolumn{1}{r}{100}		& \multicolumn{1}{r}{8} 		& 	\multicolumn{1}{r}{36}	& 	\multicolumn{1}{r}{177.78} \\
Laptop 						& \multicolumn{1}{r}{529.99} 	& \multicolumn{1}{r}{25}			& \multicolumn{1}{r}{8} 		& 	\multicolumn{1}{r}{36}	& 	\multicolumn{1}{r}{29.44} \\
\acrshort{arcos} Tucan					& \multicolumn{1}{r}{89,501.60}	& \multicolumn{1}{r}{10}			& \multicolumn{1}{r}{6} 		& 	\multicolumn{1}{r}{60}	& 	\multicolumn{1}{r}{895.02} \\
\acrshort{arcos} Mirlo					& \multicolumn{1}{r}{2,469.99}	& \multicolumn{1}{r}{70}			& \multicolumn{1}{r}{6} 		& 	\multicolumn{1}{r}{60}	& 	\multicolumn{1}{r}{172.90} \\
Printer						& \multicolumn{1}{r}{399.24}		& \multicolumn{1}{r}{5}			& \multicolumn{1}{r}{3}		& 	\multicolumn{1}{r}{60}	& 	\multicolumn{1}{r}{1.00} \\
\midrule
\textbf{\textit{Total}}		&			&			& 			& &  \multicolumn{1}{r}{\textbf{1,276.14}}\\
\bottomrule
\end{tabular}
\caption{Equipment costs.}
\label{tab:dec}
\end{table*}
\end{center}

Furthermore, the equipment presented in Table \ref{tab:dec} is detailed below:

\begin{itemize}

\item \textbf{Desktop \acrshort{pc}:} All in One - Asus Z220ICUK, 21.5'', i5-6400T, 8GB, 1TB)		

\item \textbf{Laptop:} Toshiba L50D-C-19D, A10-8700P, 8GB \gls{ram} and 1TB.

\item \textbf{\acrshort{arcos} Tucan:} Cluster used by the research group \acrshort{arcos}.

\item \textbf{\acrshort{arcos} Mirlo:} Server used by the research group \acrshort{arcos}. 32GB \gls{ram} and eight i7 processors of 2.67GHz each.

\item \textbf{Printer:} HP LaserJet Enterprise P3015.

\end{itemize}

Other direct costs are shown in Table \ref{tab:odc}. These costs consist of office material, a toner for the printer, and the monthly travel pass. Office material includes: pencils, pens, notebooks, paper, tipex, and markers.

\begin{center}
\ra{1.2}
\begin{table*}[htbp]
\centering
\begin{tabular}{@{}p{5cm} R{3.5cm}@{}} 
\toprule
\textbf{Concept} & \textbf{Cost (\euro)} \\
\midrule
Office material					& 112.98				\\
Toner (x1) 			 			& 89.62				\\
Monthly travel pass (x8) 		& 160				\\
\midrule
\textbf{\textit{Total}}			& \textbf{362.60} 	\\
\bottomrule
\end{tabular}
\caption{Other direct costs.}
\label{tab:odc}
\end{table*}
\end{center}

\subsubsection{Resumen de Costes}

Table \ref{tab:cs} shows the complete summary of the project costs. Indirect costs (20\% of direct costs) consist of the electricity and water bills, telephone, Internet access, etc.

\begin{center}
\ra{1.2}
\begin{table*}[htbp]
\centering
\begin{tabular}{@{}p{5cm} R{5cm}@{}} 
\toprule
\multicolumn{2}{c}{\textbf{\textit{Costs summary}}}\\
\midrule
\textbf{Human resources} 				& 23,800.00 \\
\textbf{Equipment} 						& 1,276.14 \\
\textbf{Other direct costs} 				& 362.60 \\
\textbf{Indirect costs}					& 5,087.75 \\
\midrule
\textbf{\textit{Total budget}}			& \textbf{30,526.49} \\
\bottomrule
\end{tabular}
\caption{Costs summary.}
\label{tab:cs}
\end{table*}
\end{center}

The total budget for this project amounts to \textbf{30,526.49 \euro \ (thirty thousand five hundred twenty-six euro and forty-nine cent)}.

\subsection{Oferta de Proyecto Propuesta}

Table \ref{tab:offer} shows a detailed offer proposal. This offer includes the estimated risks (20\%), the expected benefits (15\%), and the Value Added Tax (Spanish \gls{iva}), which corresponds to 21\% \cite{iva2012}. After applying all theses concepts, the final amount for this project in case of sale to a third-party client is \textbf{50,973.14 \euro \ (fifty thousand nine hundred seventy-three euro and fourteen cent).}

\begin{center}
\ra{1.2}
\begin{table*}[htbp]
\centering
\begin{tabular}{@{}p{2.5cm} R{2.6cm} R{3.1cm} R{3.5cm}@{}} 
\toprule
\multicolumn{4}{c}{\textbf{\textit{Offer proposal}}}\\
\midrule
\textbf{Concept} & \textbf{Increment (\%)} & \textbf{Partial value (\euro)} & \textbf{Aggregated cost (\euro)} \\
\midrule
Project costs				& - 			& 30,526.49		& 30,526.49 \\
Risk			 				& 20			& 6,105.30		& 36,631.79 \\
Benefits		 				& 15			& 5,494.77		& 42,126.56 \\
IVA		 					& 21			& 8,846.58		& 50,973.14 \\
\midrule
\textbf{\textit{Total}}		&			&			& \textbf{50,973.14}\\
\bottomrule
\end{tabular}
\caption{Offer proposal.}
\label{tab:offer}
\end{table*}
\end{center}

\section{Entorno Socio-Económico}
\label{sec:socioeconomic_environment}

As commented in previous chapters, \gls{comsimboinc} can guide the design of \gls{boinc} projects. This means that \gls{boinc} project designers can perform accurate simulations using \gls{comsimboinc} before deploying the system. Thanks to this, designers can save money and resources, because they will know the performance of the system before deploying it. In addition, it can also save energy because designers will not need to perform tests using the original infrastructure, as they will only need to use \gls{comsimboinc} in order to analyze the functioning of different alternatives.

Moreover, \gls{boinc} operates as a platform for distributed applications in areas as diverse as mathematics, medicine, molecular biology, climatology, environmental science, and astrophysics. On the one hand, there are projects that help the scientific community, such as the SETI@home project \cite{Anderson2002SETI@home}, of which the purpose is to analyze radio signals, searching for signs of extraterrestrial intelligence; or the Citizen Science Grid project \cite{csgproject}, which is dedicated to supporting a wide range of research and educational projects. On the other hand, there are projects dedicated to the environmental care, such as the Climateprediction.net project \cite{climateprediction}, which studies climate. Therefore, our simulator indirectly contributes to both science and the environment.



\lhead[\thepage]{CAPÍTULO \thechapter. CONCLUSIONES Y TRABAJOS FUTUROS}
\chead[]{}
\rhead[WepSIM: Simulador de un procesador elemental con unidad de control microprogramada\leftmark]{\thepage}
\renewcommand{\headrulewidth}{0.5pt}

\lfoot[]{}
\cfoot[]{}
\rfoot[]{}
\renewcommand{\footrulewidth}{0pt}

%% This is an example first chapter.  You should put chapter/appendix that you
%% write into a separate file, and add a line \include{yourfilename} to
%% main.tex, where `yourfilename.tex' is the name of the chapter/appendix file.
%% You can process specific files by typing their names in at the 
%% \files=
%% prompt when you run the file main.tex through LaTeX.
\chapter{Conclusiones y trabajos futuros}
\label{ch:conclusions_and_future_work}
\markboth{}{CONCLUSIONS AND FUTURE WORK}

En este capítulo se presentan las conclusiones del trabajo, se revisan los objetivos establecidos al principio de este documento, y se incluyen algunas conclusiones personales. Además, se discuten las principales contribuciones de nuestro trabajo, indicando también las publicaciones resultantes de este trabajo. Finalmente, se discute el trabajo futuro.

\section{Conclusiones}

En este trabajo se ha descrito el diseño de WepSIM, un simulador de un procesador elemental con unidad de control microprogramada. Este trabajo presenta un nuevo simulador que resulta intuitivo, portable y extensible, sirviendo como complemento docente para la docencia en Estructura de Computadores. Este simulador, permite definir diferentes juegos de instrucciones y ejecutar y depurar código fuente que use el conjunto de instrucciones definido. También permite definir el comportamiento del procesador mediante microprogramación.

WepSIM, permite a los estudiante entender el funcionamiento de un procesador elemental de una forma sencilla, pudiendo ser usado desde un dispositivo móvil o un ordenador con un navegador Web moderno, sin la necesidad de ser instalado. De esta forma, los estudiantes pueden interactuar con el simulador aprendiendo y comprendiendo el funcionamiento del procesador elemental WepSIM, incluyendo los mecanismos de interacción con el software de sistema e integrando en una misma herramienta tanto la microprogramación como la programación en ensamblador.

El objetivo principal de este proyecto era desarrollar un simulador, que a diferencia de los existentes, pudiera simular de forma completa el comportamiento de un procesador elemental permitiendo comprobar el estado de los componentes en cada ciclo de reloj, de manera que ayudáse a los estudiantes a comprender y asimilar de forma sencilla y visual el funcionamiento de un procesador.  También hemos cumplido con todos los demás objetivos presentados en la introducción del documento:

\begin{itemize}

\item \textbf{O1}, Se ha diseñado una herramienta que simula la ejecución del juego de instrucciones especificado en un computador llamado WepSIM, desde el punto de vista de la microprogramación y la programación en ensamblador.

\item \textbf{O2}, La herramienta permite la especificación de diferentes juegos de instrucciones.

\item \textbf{O3}, La herramienta unifica la microprogramación de un computador y la programación en lenguaje ensamblador.

\item \textbf{O4}, La herramienta permite al usuario visualizar en cada ciclo de reloj el estado y el comportamiento del computador simulado.

\end{itemize}

A nivel personal, este trabajo me ha ayudado a adentrarme en el mundo de la investigación científica. He logrado aplicar una gran cantidad de los conocimientos adquiridos a lo largo del grado. Además, he aprendido importantes técnicas de modelado de hardware, compilación y simulación, las cuáles tienen una gran utilidad y complejidad y me han servido parar profundizar aún más en los conocimientos adquiridos en el grado. Por todo ello, es muy satisfactorio ver el resultado final obtenido, puesto que he logrado superar todos los problemas que han surgido a lo largo del proyecto.

\subsection{Contribuciones}

El proyecto llevado a cabo durante este Trabajo Fin de Grado encaja con muchas de las asignaturas estudiadas en el Grado en Ingeniería Informática de la Universidad Carlos III de Madrid, destacando los siguientes temas en particular:

\begin{itemize}

\item \textbf{Tecnología de Computadores} (asignatura obligatoria, Primer curso) en donde se introducen los componentes hardware y la lógica binaria.

\item \textbf{Estructura de Computadores} (asignatura obligatoria, Segundo curso) en donde se introducen las bases de la estructura y funcionamiento de un computador.

\item \textbf{Teoría de Autómatas y Lenguajes Formales} (asignatura obligatoria, Segundo curso) en donde se introducen las bases acerca de los lenguajes y gramáticas formales.

\item \textbf{Sistemas Operativos} (asignatura obligatoria, Segundo curso) en donde se introducen las bases del funcionamiento del sistema operativo.

\item \textbf{Arquitectura de Computadores} (asignatura obligatoria, Tercer curso) en donde se introducen las bases de la arquitectura de un computador.

\item \textbf{Diseño de Sistemas Operativos} (asignatura obligatoria, Tercer curso) en donde se introducen las bases del diseño de los distintos módulos de un sistema operativo.

\item \textbf{Dirección de proyectos de desarrollo de software} (asignatura obligatoria, Tercer curso) en donde se introducen las bases para la dirección y gestión de un proyecto de desarrollo de software.

\end{itemize}

\subsection{Publicaciones}

Este Trabajo Fin de Grado ha permitido realizar una importante contribución al campo de la docencia en Estructura y Arquitectura de Computadores. Además, se han conseguido publicar los siguientes artículos científicos:

\begin{itemize}

\item \textbf{A. Calderón, F. García-Carballeira, and J. Prieto}, “WepSIM: Simulador modular e interactivo de un procesador elemental para facilitar una visión integrada de la microprogramación y la programación en ensamblador”, \textit{Enseñanza y aprendizaje de ingeniería de computadores}, vol. 6, 35-53,2016. \cite{mateos2016wepsim}

\item \textbf{J. Prieto, A. Calderón, F. García-Carballeira, and S. Alonso-Monsalve}, “WepSIM: simulador integrado de microprogramación y programación en ensamblador”, \textit{Jornadas sarteco 2016}. \cite{arcos2032}

\end{itemize}

Además, en el momento de la entrega de este documento, también otro artículo enviado a la espera de su aceptación.

\vspace{1cm}

\section{Trabajos futuros}

Actualmente, hay varias líneas de trabajos futuros en las cuáles estamos trabajando.

\begin{itemize}

\item En cuanto a mejoras en el modelo hardware:

\begin{itemize}

\item[1.] Introducir más elementos hardware, como por ejemplo una caché, de forma que se amplíen los contenidos de la asignatura incluídos en la herramienta.

\item[2.] Introducir un modelo hardware basado en \textit{pipeline}, permitiendo el uso de la herramienta en aquellas asignaturas que utilizan este modelo de arquitectura. 

\end{itemize}

\item En cuanto a mejoras en el modelo software:

\begin{itemize}

\item[3.] Añadir revisión semántica del código, permitiendo identificar y notificar los errores de programación al usuario.

\item[4.] Añadir nuevos juegos de instrucciones a la herramienta como por ejemplo el ensamblador ARM, permitiendo la utilización de diferentes lenguajes en la herramienta.

\item[5.] Estudiar el ensamblador de MIPS/ARM generado con GCC/Clang de forma que pueda ser usado directamente en WepSIM.

\end{itemize}

\item En cuanto a mejoras en la herramienta:

\begin{itemize}

\item[6.] Añadir un módulo de corrección automática de prácticas a la herramienta, de forma que los estudiantes puedan practicar con ella y comprobar la validez de sus ejercicios.

\item[7.] Migrar la herramienta como aplicación móvil mediante el \emph{plugin} Apache Cordova, de forma que la herramienta no quede ligada al uso mediante navegador web.

\end{itemize}

\end{itemize}


\lhead[\thepage]{APPENDIX A. USER MANUAL}
\chead[]{}
\rhead[A Complete Simulator for Volunteer Computing Environments]{\thepage}
\renewcommand{\headrulewidth}{0.5pt}

\lfoot[]{}
\cfoot[]{}
\rfoot[]{}
\renewcommand{\footrulewidth}{0pt}

\begin{appendices}
\chapter{User Manual}
\label{ch:user_manual}

This appendix presents a detailed user manual of \gls{comsimboinc}. First we indicate the basic requirements to deploy the application and a detailed tutorial for the installation of SimGrid. Finally, we present an example of the simulator usage.

\section{Basic Requirements}

The technical specifications recommended for the final user to obtain the best experience from the application are:

\begin{itemize}

\item \textbf{Operating System}: Ubuntu 14.04.4 LTS (Linux distribution) or higher.

\item \textbf{Processor}: Intel(R) Core(TM) i7 CPU 920 @2.67GHz or higher.

\item \textbf{\gls{ram}}: 8 GB or higher.

\item \textbf{Storage}: 1 GB of free space in the Hard Disk Drive.

\item \textbf{Network}: Internet connection is not required.

\item \textbf{Software}: The following software must be installed in order to run the application:

	\begin{itemize}

	\item[1.] GCC (GNU Compiler) 5.1 or higher.
	
	\item[2.] SimGrid toolkit 3.10 or higher.
	
	\end{itemize}

\end{itemize}


\section{SimGrid Installation}

We will present a tutorial for the installation of the SimGrid toolkit (version 3.10). First, you have to download the official binary package from the \textit{download page} (\url{http://simgrid.gforge.inria.fr/download.php}). In this case you will download the file \textit{SimGrid-3.10.tar.gz}.

Then, you have to recompile de archive. This should be done in a few lines:

\vspace{0.6cm}

\begin{lstlisting}[language=bash]
  $ tar xf SimGrid-3.10.tar.gz
  $ cd SimGrid-3.10
  $ cmake -DCMAKE_INSTALL_PREFIX=/opt/simgrid $HOME
  $ make
  $ make install
\end{lstlisting}

\vspace{0.6cm}

After following these steps, you will have the SimGrid toolkit installed in your computer.


\section{Usage Example}

In order to use \gls{comsimboinc}, you must download the corresponding files from the following website: \url{https://www.arcos.inf.uc3m.es/~combos/}. After unzipping the downloaded file, the unzipped files will follow the folder structure presented in Figure \ref{fig:folder_structure} (Section \ref{sec:deployment}, \textit{\nameref{sec:deployment}} (Chapter \ref{ch:implementation_and_deployment}, \textit{\nameref{ch:implementation_and_deployment}})). To perform simulations using \gls{comsimboinc}, it is necessary to model the platform to be simulated. Once you know the environment to simulate, you must specify all simulation parameters in the parameters \gls{xml} file.

Figure \ref{fig:interaction} shows an example of a potential simulation that can be carried out by \gls{comsimboinc}. The figure shows a simplified platform with two \gls{boinc} projects and 350,000 clients. The first project is represented by two \gls{scheduling} servers (SS0 and SS1) and two data servers (DS0 and DS1). The second project consists of a single \gls{scheduling} server (SS2) and three data servers (DS2, DS3 and DS4). Clients are grouped into three sets. The first group (G0) consists of 100,000 hosts and has a route to the first project. The second group (G1), has 200,000 hosts and a route to both projects. The third group (G2) consists of 50,000 computers and has route to the second project. The rest of the figure shows the links among the elements of the environment (from L0 to L7). In each of the links, latency and bandwidth are indicated.

\begin{figure}[htbp]
    \centering
    \includegraphics[width=13.5cm]{figures/interaction}
    \caption{Simulator platform example.}
    \label{fig:interaction}
\end{figure}

To create a simulation, \gls{comsimboinc} requires to specify all the parameters described in Tables \ref{tab:project_parameters} and \ref{tab:vngroup_parameters} in the parameters.xml file (see Listing \ref{lis:parameters}). Users can define the power and availability of the volunteer hosts via either a traces file or distribution functions. For example, in the case of the SETI@home project, we have analyzed the 3,900,000 hosts that participate in this project. The CPU performance of the hosts can be modeled according to an exponential function, as shown Figure \ref{fig:exponential}, which has a mean of 5.871 Giga\acrshort{flops} per host.

\begin{figure}[htbp] 
	\begin{subfigure}{0.5\textwidth}
		\includegraphics[width=\linewidth]{figures/density}
		\captionsetup{width=.9\textwidth}
		\caption{Probability density function of SETI@home hosts power.} 
		\label{fig:pdf}
	\end{subfigure}
	\hspace*{\fill} % separation between the subfigures
	\begin{subfigure}{0.5\textwidth}
		\includegraphics[width=\linewidth]{figures/distribution}
		\captionsetup{width=.9\textwidth}
		\caption{Cumulative distribution function of SETI@home hosts power.} 
		\label{fig:cdf}
	\end{subfigure}
	\caption{CPU performance modeling for SETI@home hosts.}
	\label{fig:exponential}
\end{figure}


Our software first processes the \gls{xml} file, creating the necessary deployment and platform files for subsequent simulations. In  \gls{comsimboinc}, all this is transparent to the user. The user only has to specify all the parameters of the simulation in the \gls{xml} file (Listing \ref{lis:parameters}), and run the generator script using the following command:

\vspace{0.6cm}

\begin{lstlisting}[language=bash]
  $ ./generator
\end{lstlisting}

\vspace{0.6cm}

The above command generates the platform and deployment files. The platform file contemplates all the necessary elements in the simulation: hosts, clusters, links, etc. The deployment file indicates the processes that should be created during the simulation. In addition, the generator script also compiles all source files needed for the simulation and generates the executable file. Finally, to run the simulation you just need to run the execution script:

\vspace{0.6cm}

\begin{lstlisting}[language=bash]
  $ ./execute
\end{lstlisting}

\vspace{0.6cm}

The execution results are composed by multiple statistical results (see Listings \ref{lis:output}): the execution time, the memory usage of the simulator, the load of the \gls{scheduling} and data servers, the total number of work requests received in the \gls{scheduling} servers, the job statistics (number of jobs created, sent, received, analyzed, success, fail, too late, etc), the credit granted to the clients, the number of \acrshort{flops}, the average power of the volunteer nodes, and the percentage of time the volunteer nodes were available during the simulation.

\clearpage

    \lstset{
    language=xml,
    morekeywords={Server, Client, side},
    tabsize=3,
    frame=lines,
    label=code:sample,
    frame=shadowbox,
    rulesepcolor=\color{gray},
    xleftmargin=20pt,
    framexleftmargin=15pt,
    numbers=left,
    numberstyle=\tiny,
    numbersep=5pt,
    breaklines=true,
    showstringspaces=false,
    basicstyle=\footnotesize,
    captionpos=b,
    caption=parameters.xml file filled with the parameters of the example.,
    label=lis:parameters}
    
    \lstinputlisting{figures/parameters.xml}
    
\clearpage

    \lstset{
    %language=xml,
    morekeywords={PROJECT1, PROJECT2},
    tabsize=3,
    frame=lines,
    label=code:sample,
    frame=shadowbox,
    rulesepcolor=\color{gray},
    xleftmargin=20pt,
    framexleftmargin=15pt,
    numbers=left,
    numberstyle=\tiny,
    numbersep=5pt,
    breaklines=true,
    showstringspaces=false,
    basicstyle=\footnotesize,
    captionpos=b,
    caption=Simulation execution results.,
    label=lis:output}
    
    \lstinputlisting{figures/output}
    
\end{appendices}

%\lhead[\thepage]{APPENDIX A. USER MANUAL}
\chead[]{}
\rhead[A Complete Simulator for Volunteer Computing Environments]{\thepage}
\renewcommand{\headrulewidth}{0.5pt}

\lfoot[]{}
\cfoot[]{}
\rfoot[]{}
\renewcommand{\footrulewidth}{0pt}

\begin{appendices}
\chapter{User Manual}
\label{ch:user_manual}

This appendix presents a detailed user manual of \gls{comsimboinc}. First we indicate the basic requirements to deploy the application and a detailed tutorial for the installation of SimGrid. Finally, we present an example of the simulator usage.

\section{Basic Requirements}

The technical specifications recommended for the final user to obtain the best experience from the application are:

\begin{itemize}

\item \textbf{Operating System}: Ubuntu 14.04.4 LTS (Linux distribution) or higher.

\item \textbf{Processor}: Intel(R) Core(TM) i7 CPU 920 @2.67GHz or higher.

\item \textbf{\gls{ram}}: 8 GB or higher.

\item \textbf{Storage}: 1 GB of free space in the Hard Disk Drive.

\item \textbf{Network}: Internet connection is not required.

\item \textbf{Software}: The following software must be installed in order to run the application:

	\begin{itemize}

	\item[1.] GCC (GNU Compiler) 5.1 or higher.
	
	\item[2.] SimGrid toolkit 3.10 or higher.
	
	\end{itemize}

\end{itemize}


\section{SimGrid Installation}

We will present a tutorial for the installation of the SimGrid toolkit (version 3.10). First, you have to download the official binary package from the \textit{download page} (\url{http://simgrid.gforge.inria.fr/download.php}). In this case you will download the file \textit{SimGrid-3.10.tar.gz}.

Then, you have to recompile de archive. This should be done in a few lines:

\vspace{0.6cm}

\begin{lstlisting}[language=bash]
  $ tar xf SimGrid-3.10.tar.gz
  $ cd SimGrid-3.10
  $ cmake -DCMAKE_INSTALL_PREFIX=/opt/simgrid $HOME
  $ make
  $ make install
\end{lstlisting}

\vspace{0.6cm}

After following these steps, you will have the SimGrid toolkit installed in your computer.


\section{Usage Example}

In order to use \gls{comsimboinc}, you must download the corresponding files from the following website: \url{https://www.arcos.inf.uc3m.es/~combos/}. After unzipping the downloaded file, the unzipped files will follow the folder structure presented in Figure \ref{fig:folder_structure} (Section \ref{sec:deployment}, \textit{\nameref{sec:deployment}} (Chapter \ref{ch:implementation_and_deployment}, \textit{\nameref{ch:implementation_and_deployment}})). To perform simulations using \gls{comsimboinc}, it is necessary to model the platform to be simulated. Once you know the environment to simulate, you must specify all simulation parameters in the parameters \gls{xml} file.

Figure \ref{fig:interaction} shows an example of a potential simulation that can be carried out by \gls{comsimboinc}. The figure shows a simplified platform with two \gls{boinc} projects and 350,000 clients. The first project is represented by two \gls{scheduling} servers (SS0 and SS1) and two data servers (DS0 and DS1). The second project consists of a single \gls{scheduling} server (SS2) and three data servers (DS2, DS3 and DS4). Clients are grouped into three sets. The first group (G0) consists of 100,000 hosts and has a route to the first project. The second group (G1), has 200,000 hosts and a route to both projects. The third group (G2) consists of 50,000 computers and has route to the second project. The rest of the figure shows the links among the elements of the environment (from L0 to L7). In each of the links, latency and bandwidth are indicated.

\begin{figure}[htbp]
    \centering
    \includegraphics[width=13.5cm]{figures/interaction}
    \caption{Simulator platform example.}
    \label{fig:interaction}
\end{figure}

To create a simulation, \gls{comsimboinc} requires to specify all the parameters described in Tables \ref{tab:project_parameters} and \ref{tab:vngroup_parameters} in the parameters.xml file (see Listing \ref{lis:parameters}). Users can define the power and availability of the volunteer hosts via either a traces file or distribution functions. For example, in the case of the SETI@home project, we have analyzed the 3,900,000 hosts that participate in this project. The CPU performance of the hosts can be modeled according to an exponential function, as shown Figure \ref{fig:exponential}, which has a mean of 5.871 Giga\acrshort{flops} per host.

\begin{figure}[htbp] 
	\begin{subfigure}{0.5\textwidth}
		\includegraphics[width=\linewidth]{figures/density}
		\captionsetup{width=.9\textwidth}
		\caption{Probability density function of SETI@home hosts power.} 
		\label{fig:pdf}
	\end{subfigure}
	\hspace*{\fill} % separation between the subfigures
	\begin{subfigure}{0.5\textwidth}
		\includegraphics[width=\linewidth]{figures/distribution}
		\captionsetup{width=.9\textwidth}
		\caption{Cumulative distribution function of SETI@home hosts power.} 
		\label{fig:cdf}
	\end{subfigure}
	\caption{CPU performance modeling for SETI@home hosts.}
	\label{fig:exponential}
\end{figure}


Our software first processes the \gls{xml} file, creating the necessary deployment and platform files for subsequent simulations. In  \gls{comsimboinc}, all this is transparent to the user. The user only has to specify all the parameters of the simulation in the \gls{xml} file (Listing \ref{lis:parameters}), and run the generator script using the following command:

\vspace{0.6cm}

\begin{lstlisting}[language=bash]
  $ ./generator
\end{lstlisting}

\vspace{0.6cm}

The above command generates the platform and deployment files. The platform file contemplates all the necessary elements in the simulation: hosts, clusters, links, etc. The deployment file indicates the processes that should be created during the simulation. In addition, the generator script also compiles all source files needed for the simulation and generates the executable file. Finally, to run the simulation you just need to run the execution script:

\vspace{0.6cm}

\begin{lstlisting}[language=bash]
  $ ./execute
\end{lstlisting}

\vspace{0.6cm}

The execution results are composed by multiple statistical results (see Listings \ref{lis:output}): the execution time, the memory usage of the simulator, the load of the \gls{scheduling} and data servers, the total number of work requests received in the \gls{scheduling} servers, the job statistics (number of jobs created, sent, received, analyzed, success, fail, too late, etc), the credit granted to the clients, the number of \acrshort{flops}, the average power of the volunteer nodes, and the percentage of time the volunteer nodes were available during the simulation.

\clearpage

    \lstset{
    language=xml,
    morekeywords={Server, Client, side},
    tabsize=3,
    frame=lines,
    label=code:sample,
    frame=shadowbox,
    rulesepcolor=\color{gray},
    xleftmargin=20pt,
    framexleftmargin=15pt,
    numbers=left,
    numberstyle=\tiny,
    numbersep=5pt,
    breaklines=true,
    showstringspaces=false,
    basicstyle=\footnotesize,
    captionpos=b,
    caption=parameters.xml file filled with the parameters of the example.,
    label=lis:parameters}
    
    \lstinputlisting{figures/parameters.xml}
    
\clearpage

    \lstset{
    %language=xml,
    morekeywords={PROJECT1, PROJECT2},
    tabsize=3,
    frame=lines,
    label=code:sample,
    frame=shadowbox,
    rulesepcolor=\color{gray},
    xleftmargin=20pt,
    framexleftmargin=15pt,
    numbers=left,
    numberstyle=\tiny,
    numbersep=5pt,
    breaklines=true,
    showstringspaces=false,
    basicstyle=\footnotesize,
    captionpos=b,
    caption=Simulation execution results.,
    label=lis:output}
    
    \lstinputlisting{figures/output}
    
\end{appendices}

%\printglossary[type=\acronymtype,title=Abbreviations]
%\printglossary

%Some text between the list of acronyms and the glossary.

\printglossary
\addcontentsline{toc}{chapter}{Glosario}

\printglossary[type=\acronymtype]
\addcontentsline{toc}{chapter}{Acrónimos}

\cleardoublepage
\phantomsection
\lhead[\thepage]{BIBLIOGRAFÍA}
\chead[]{}
\rhead[WepSIM: Simulador de procesador elemental con unidad de control microprogramada]{\thepage}
\renewcommand{\headrulewidth}{0.5pt}

\lfoot[]{}
\cfoot[]{}
\rfoot[]{}
\renewcommand{\footrulewidth}{0pt}

%% This defines the bibliography file (main.bib) and the bibliography style.
%% If you want to create a bibliography file by hand, change the contents of
%% this file to a `thebibliography' environment.  For more information 
%% see section 4.3 of the LaTeX manual.
\begin{singlespace}
\addcontentsline{toc}{chapter}{Bibliography}
\bibliography{main}
\bibliographystyle{ieeetr}
\end{singlespace}

\end{document}