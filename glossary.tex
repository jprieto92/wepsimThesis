% Glossary

\newglossaryentry{ensamblador}{name=ensamblador, description={lenguaje de programación de bajo nivel.}}

\newglossaryentry{computerstructure}{name=Estructura de Computadores, description={Asignatura cuyo objetivo es describir el funcionamiento básico de un computador}}

\newglossaryentry{microcodigo}{name=microcódigo, description={Instrucciones o estructuras de datos implicados en la implementación  de lenguaje máquina}}

\newglossaryentry{hardware}{name=hardware, description={Conjunto de elementos físicos o materiales que constituyen una computadora o un sistema informático}}

\newglossaryentry{software}{name=software, description={Conjunto de programas y rutinas que permiten a la computadora realizar determinadas tareas}}

\newglossaryentry{framework}{name=framework, description={Estructura real o conceptual destinada a servir de soporte o guía para la construcción de algo que expande la estructura en algo útil}}

\newglossaryentry{pipeline}{name=pipeline, description={Arquitectura basada en la transformación de un flujo de datos en un proceso comprendido por varias fases secuenciales, siendo la entrada de cada una la salida anterior}}

\newglossaryentry{lexyacc}{name=LEX and YACC, description={Programa para generar analizadores léxicos y sintácticos}}

\newglossaryentry{opensource}{name=open-source, description={Software desarrollado y distribuido libremente}}

\newglossaryentry{protocol}{name=protocolo, description={Es el conjunto especial de reglas que terminan los puntos en una conexión de telecomunicación cuando se comunican. Los protocolos especifican interacciones entre las entidades comunicantes}}
